%!TEX TS-program = xelatex
% vim: set fenc=utf-8

% -*- coding: UTF-8; -*-
%!TEX encoding = UTF-8 Unicode 
\documentclass[master]{cugthesis}

% Some shortcuts
\newcommand\N{\ensuremath{\mathbb{N}}}
\newcommand\R{\ensuremath{\mathbb{R}}}
\newcommand\Z{\ensuremath{\mathbb{Z}}}
\renewcommand\O{\ensuremath{\emptyset}}
\newcommand\T{\ensuremath{\mathbb{T}}}
\renewcommand\d{\ensuremath{\,\mathrm{d}}}
\newcommand\Q{\ensuremath{\mathbb{Q}}}
\newcommand\C{\ensuremath{\mathbb{C}}}
\renewcommand\thesubfigure{(\alph{subfigure})}

% Some theorem environment settings
\usepackage{ntheorem,tikz,mathabx,tkz-euclide,caption,subcaption}
\theoremseparator{.}
\newtheorem{theorem}{定理}[chapter]
\newenvironment{proof}{{\noindent\itshape 证明}.}{\hfill $\Box$\par}
\newtheorem*{question}{Question}

\newtheorem{proposition}{性质}[chapter]
\newtheorem{corollary}{推论}[chapter]
\newtheorem{lemma}{引理}[chapter]
\theorembodyfont{\upshape}
\newtheorem{remark}{注}
\newtheorem{definition}{定义}[chapter]
\newtheorem{exercise}{Exercise}[section]
\newtheorem*{solution}{Solution}
\newtheorem{example}{例}


\cugthesistitle{KdV方程能观测不等式}{KdV方程能观测不等式}
\cugthesisauthor{王允磊}{Zhenyu Wang}
\studentid{1201910835}
\cugthesismajor{数学与应用数学}{Mathematics and Applied Mathematics}
\cugthesisteacher{王明}{Ming Wang}
\educatingunit{数学与物理学院}
\cugthesisdate{2022}{1}

\cugabstract{本文主要研究线性 KdV 方程的能观测不等式, 即用某些时刻或时间段内函数在局部空间上的信息给出整个函数在某些赋范空间中的上界.   

对于分布在全空间的线性 KdV 方程,  我们受到 Amrein-Berthier 不确定性原理证明的启发, 借助不依赖于非观测区域有界性的希尔伯特空间方法给出了两点时刻能观测不等式的一般判据.  由该判据, 我们首次得到了观测区域为可测集外的能观测不等式. 通过与唯一延拓性相结合, 我们还得到了非观测区域为一类满足一定密度条件且测度大小可以无限的集合类的两点时刻能观测不等式. 


对于分布在周期情形下的线性 KdV 方程, 受到薛定谔方程关于移动观测点可观测性研究的启发, 我们首先得到了 KdV 方程关于单个移动观测点的可观测性, 并建立了相应的充分必要判据. 在此基础上, 我们对两个移动观测点的情形进行了详细地讨论, 这里的情形与薛定谔方程完全不同, 我们利用图论的描述方法 给出了两个移动观测点可观测性的等价陈述. 最后, 我们利用数的素因式分解和函数图像的性质给出了 KdV 方程关于两个移动观测点可观测性的判定定理

最后, 我们利用HUM方法将上述能观测不等式中的结果转化为方程的可控性定理.}{
We mainly consider the linear KdV equations on $\R$ and $\T$. In this paper, we present new observability inequalities, which give the upper bounds of the norm of the solutions through information obtained by some part of the region in some time.

For the linear KdV equations on $\R$, inspired by the proof of the Amrein-Berthier uncertainty principle, we give the general criterion of observability inequalities by means of the Hilbert space method, which does not depend on the boundedness of the non-observable region. By this criterion, we obtained new observability inequalities at two time points from measuable sets. Combined with the unique continuation property, , we obtained new observability inequalities at two time points from some sets which satisfy some density condition, and these sets may have infinite measure.


For the linear KdV equation on $\R$, inspired by the study of the Schrödinger equation on the observability of moving observation points, we give the observability of the KdV equation with respect to a single moving observation point, and established the corresponding sufficient and necessary criteria . Further, we discuss the situation of two moving observation points in detail. This situation is completely different from the Schrödinger equation, and we give the equivalent statement of the observability of two moving observation points through the language of graph theory. Finally, we use the prime factorization and the properties of the function to give the criterions for the observability of two moving points.

Finally, we use the HUM method to convert the results above into the controllability theorems.
}
\cugkeywords{KdV 方程; 能观测不等式; 不确定性原理; 唯一延拓性; 紧算子; 可控性.}{KdV equations; observability inequality; uncertainty principle; unique continuation property; controllability.} 
\begin{document}
    \makefrontpages 
    % 你的正文
    \chapter{前言}
    本文主要研究线性KdV方程, 给出全空间和周期情形下新的能观测不等式, 并讨论其在控制理论中的应用.
    \section{KdV方程概述}
    Korteweg-de Vries 方程 (简称 KdV方程), 于1877年 Boussinesq\cite{Bouss1877} 首先给出, 并在1895年 Korteweg-de Vries \cite{Kort1895} 研究管道中液体浅水波时被重新发现. 该方程写作
    \begin{equation}
        \partial_t u +u \partial_x u +\partial_x^3 u =0,\label{kdv}
    \end{equation}
    其中 $u=u(t,x)$是一个关于时间变量 $t$和空间变量$x$ 的实值函数. 它同时包含了色散项和非线性项, 是人们研究色散与非线性之间的相互作用的重要模型.
    
    KdV 方程有许多变种, 其中最简单的便是不带非线性项 $u\partial_xu$ 的线性 KdV 方程:
    \begin{equation}
        \partial_tu+\partial_x^3u=0.
    \end{equation}
    由于其舍去了非线性项, 通常情况下方程解的性质更易于得到, 然后在此基础上进一步研究方程 \eqref{kdv} 是否仍保持相应的性质. 另一方面, 许多时候非线性项并不局限于 $u\partial_x u$ 一种形式, 例如下面的方程
    \begin{equation}
        \partial_tu +\partial_x^3u+\partial_x f(u)=0,
    \end{equation}
    其中 $f(u)$ 为 $u$ 的多项式, 具有该形式的 $KdV$ 方程被称作广义 KdV 方程(gKdV). 在研究 KdV 方程的某些性质时, 非线性项 $\partial_x f(u)$ 的具体形式可能不会起到本质作用.
    
    
   \iffalse KdV 方程的非线性项 $u\partial_x u$让它在各个函数空间中适定性的证明变得不平凡. 关于 KdV 方程在Sobolev空间 $H^s(\R)$ 和 $H^s(\T)$ 中适定性的研究, 早期的结果见 \cite{bona1975initial, kato1979cauchy,cohen1979existence,ginibre1990existence,kenig1993well}. Bourgain 在1993年发表的论文\cite{bourgain1993fourier} 中, 利用新空间 $X^{s,b}$ 有效地刻画了方程的色散特性以及它与线性方程间的偏移量, 由此证明了方程在 $s\ge 0$ 时两种空间下的局部适定性. Kenig, Ponce 和 Vega 在此结果的基础上结合他们发现的双线性估计 \cite{kenig1996bilinear}, 将局部适定性在 $\R$ 和 $\T$ 情形下的范围分别扩大到 $s\ge -\frac{3}{4}$ 和 $s\ge -\frac{1}{2}$. Bourgain 建立了\cite{bourgain1997periodic} 初值具有有界傅里叶变换的全局适定性, 证明中用到了 KdV 系统的可积性. Tao 等人用 $I$--方法证明了 $s>-\frac{3}{4}$ 的全局适定性, 并要求解映射在 $H^s$上一致连续时 $-\frac{3}{4}$ 是临界点. Guo 则给出 \cite{guo2009global} 了全空间情形 $s=-\frac{3}{4}$ 的全局适定性.
    \fi
    
    
    
    KdV方程自发现以来便被广泛研究, 在方程的适定性, 孤立波的存在性和稳定性, 可积性, 唯一延拓性, 长时间性态上都有丰富的结果. 
    
    本论文所研究的能观测不等式便属于唯一延拓性这一范畴.
    
    
    \section{能观测不等式与唯一延拓性}
    对于一般的定义在区域 $\Omega$ 上的发展方程, 可写作
    \begin{equation}\label{evo}
        \partial_t u = L u, \quad u(0,x)=u_0\in D(L).
    \end{equation}
    很多时候, 存在 $\omega\subset \Omega$, $T>0$, 以及相应的常数 $C(T,\omega)>0$, 使得对任意满足方程 \eqref{evo} 的解 $u(t,x)$ 都有下述不等式成立
    \begin{equation}\label{init}
        \int_{\Omega}|u_0(x)|^2\d x\le C(T,\omega)\int_0^T\int_\omega|u(t,x)|^2\d x\d t.
    \end{equation}
    这样的不等式称为色散方程 \eqref{evo} 的能观测不等式. 

    
    由能观测不等式 \eqref{init} 的形式可得到相应的唯一延拓性: 当解函数 $u(t,x)$ 在 $(0,T)\times \omega$ 上为零时, $u(t,x)\equiv 0$, 见图 \ref{fig2}. 因此我们可以粗略地认为:
    \begin{center}
        \itshape 能观测不等式就是定量的唯一延拓性.
    \end{center}
    \begin{figure}[ht]
    \centering
    \begin{tikzpicture}[scale=1.3]
      \draw plot[smooth cycle] coordinates{(0,0) (4,0) (2.5,2.5) (-1,2.5)};
      \draw[magenta] (0.5,0.5) ellipse (0.5 and 0.375);
      \draw[magenta] (0.9,2.1) ellipse (0.5 and 0.375);
      \draw[magenta] (2.5,0.8) ellipse (0.5 and 0.375);
      \draw (2,-0.3) node[below]{$\Omega$};
      \draw[->] (4-0.15,2) to[out=180,in=40]  (0.5,0.5);
      \draw[->] (4-0.1,2+0.1) to[out=150, in=30] (0.9,2.1);
      \draw[->] (4-0.15,2-0.1) to[out=-120, in=10] (2.5,0.8);
      \draw (4.1,2) node{$\color{magenta}{\omega}$};
    \end{tikzpicture}
    \caption{唯一延拓性: 若$u(t,x)$ 在时间 $0$到 $T$, 区域$\omega$ 内为零, 则$u(t,x)\equiv 0$.}
    \label{fig2}
  \end{figure}
    
    近五十年来, 人们对色散方程唯一延拓性和能观测不等式的建立做了大量的研究, 并发展出许多方法. 特别地, 这些方法被广泛运用于两类最重要的色散方程 -- 薛定谔方程和 KdV 方程当中. 在唯一延拓性和能观测不等式的研究进展中, 这两大类方程所运用的方法相互影响, 所得到的的结果亦有许多相似之处. 一方面, 由于两类方程有许多共同点, 同一种形式的唯一延拓性或能观测不等式往往能够同时对两类方程建立. 另一方面, 它们在色散程度以及非线性项上的区别, 使得即使我们建立的是形式上相同的结果, 有时却必须采用完全不同的方法. 
    因此, 在接下来对唯一延拓性和能观测不等式发展历史的回顾中, 我们不仅讨论 KdV 方程, 也将薛定谔方程相应的发展历程一并对比讨论.
\subsection{全空间情形}    
    上世纪八十年代, Saut 和 Scheurer \cite{saut1980theoreme,Saut1987UniqueCF} 考虑了一维全空间中, 形如
    \begin{equation*}
        L=iD_t +\alpha i^{2k+1}D^{2k+1}+R(x,t,D)
    \end{equation*}
    的色散算子, 其中 $\alpha\neq 0, D=\frac{1}{i}\partial_x, D_t =\frac{1}{i}\partial_t$, 以及
    \begin{equation*}
        R(t,x,D)=\sum_{j=0}^{2k}r_j(t,x)D^j,\quad r_j\in L_{\rm{loc}}^\infty(R,L^2_{\rm{loc}}(\R)).
    \end{equation*}
    他们用 Carleman 估计证明了: 若 $u\in L^2_{\rm{loc}}(\R,H_{\rm{loc}}^{2k+1}(\R))$ 是方程 $Lu=0$ 的解, 且在 $\R_t\times\R_x $ 的一个开子集 $\Omega$ 内为零, 则其在 $\Omega$ 的纵向区域内均为零. 该结论用于 KdV 方程可得, 当 $u\in L_{\rm{loc}}^\infty (\R, H^3(\R))$ 是方程 \eqref{kdv} 的解且在 $\R_t\times \R_x$ 的开子集内为零时, $u$ 恒为零. 另一方面, Tataru \cite{tataru1995carleman} 也用 Carleman 估计证明了一定条件下薛定谔方程的唯一延拓性. 进一步地, Isakov \cite{isakov1993carleman} 用 Carleman 估计证明了一大类具有非齐次主项发展方程的唯一延拓性. 
    1992 年, Zhang \cite{Zhang1992UniqueCF} 利用傅里叶变换的逆散射方法以及 Miura 变换证明了 KdV 方程在一定条件下两点时刻同一半轴上的唯一延拓性: $u(t,x)$ 在 $\lbrace t_0,t_1\rbrace\times (c,\infty)$ (或 $\lbrace t_0,t_1\rbrace \times (-\infty,c)$) 上为零, $c$ 为一常数, 则 $u$ 恒为零. 1997年, Bourgain  \cite{bourgain1997compactness} 利用实轴上具有紧支撑函数的傅里叶变换在复平面的解析性质, 证明了形如 
    \begin{equation*}
        \partial_t u+ \partial_x^3u+\partial_x F(u) = 0, \quad F(u) \text{ 为多项式}
    \end{equation*}
  的方程不存在具有紧支撑的非平凡解, 即一段时间内解函数在一紧集外为零时, 解恒为零. 2002 年, Kenig, Ponce 和 Vega \cite{kenig2002support} 进一步完善 Zhang \cite{Zhang1992UniqueCF} 的结果并证明了: 对于广义 KdV 方程 $u(t,x)$, 若其足够光滑且满足一定指且在两个不同时刻只分布在同一半轴上, 则 $u$ 恒为零.
  
  以上关于唯一延拓性的进展均为定性结果.   
    
    2009年, Rosier \cite{Rosier2009ExactBC} 对薛定谔方程
    \begin{equation}
        i\partial_t u +\Delta u =0,\quad u(0,x)=u_0\in L^2(\R^n)\label{sch}
    \end{equation}
    的解 $u(t,x)$, 建立了能观测不等式
    \begin{equation}
        \int_{\R^n}|u_0(x)|^2\d x\le C(n,T,E)\int_0^T\int_E|u(t,x)|^2\d x\mathrm{d}t,\label{schobs-g}
    \end{equation}
    其中 $T>0$, $E$是 $\R^n$中的子集, $C(n,T,E)>0$是仅依赖于 $n,T,E$ 的常数. 实际上, 不等式 \eqref{schobs-g} 在任意的 $n\ge 1$ 中都可以取 $E=\left\lbrace x\in \R^n: |x|\ge r\right\rbrace, $ 并且在 $n=1$ 时可取 $E$ 为厚集\cite{Huang2020ObservableSP} (一种更为一般的集合类). 
    
     Gengsheng Wang, Ming Wang 和 Yubiao Zhang在2019年发表的文章\cite{Wang2019ObservabilityAU}中证明了下述新的能观测不等式: 存在一个常数 $C=C(n)>0$ 使得对于所有的 $t>0$, 所有的 $r_1,r_2>0$ 以及所有满足 \eqref{sch} 的解 $u(t,x)$, 都有
    \begin{equation}
        \int_{\R^n}|u_0(x)|^2\d x\le C e^{\frac{Cr_1r_2}{t}}\left(\int_{|x|\ge r_1}|u_0(x)|^2\d x+\int_{|x|\ge r_2}|u(t,x)|^2\d x\right).\label{schobs}
    \end{equation}这一结果将不等式 \eqref{schobs-g} 右边的观测时间从一段时间改进到两点时刻,    因此我们称不等式 \eqref{schobs} 为两点能观测不等式. 另外, 在\cite[5.2节]{Wang2019ObservabilityAU}中还得到了不等式 \eqref{schobs} 与两点时刻限制在球外的脉冲控制的精确控制性. 进一步地, Ming Wang, Ze Li 和 Shanlin Huang\cite{Wang2021Indiana}得到了带有某种位势项的非线性薛定谔方程两点能观测不等式.
    
    不等式 \eqref{schobs} 的证明基于调和分析中的不确定性原理, 该原理描述的是任意一个非零函数和它的傅里叶变换不可能同时具有紧支撑, 它的一个定量版本是下述的不等式: 对任意的 $r_1,r_2>0$和任意的 $f\in L^2(\R^n)$, 都存在一个只依赖于 $n$ 的常数 $C=C(n)>0$ 使得
    \begin{equation}
        \int_{\R^n}|f(x)|^2\d x\le C e^{Cr_1r_2}\left(\int_{|x|\ge r_1}|f|^2\d x+\int_{|x|\ge r_2}|\widehat{f}(\xi)|^2\d \xi\right)\label{uncertainty}
    \end{equation}
    成立, 更详细的介绍见\cite{Havin2012, Jaming2007NazarovsUP,Nazarov1993}. 注意到薛定谔方程 \eqref{sch} 的解满足关系式\cite{Linares2014Ponce}
    \begin{equation}
        (2it)^{\frac{n}{2}}e^{-i|x|^2 / 4t} u(x,t) = \widehat{e^{i|\cdot|^2 /4t}u_0}(x /2t), \quad\forall t>0.\label{idn}
    \end{equation}
    该等式说明解函数 $u(t,x)$ 在 $t$ 时刻与初始函数 $u_0$的傅里叶变换只相差一个模为 $1$ 的因子. 有了 \eqref{idn} 式, 能观测不等式 \eqref{schobs} 可以由 \eqref{uncertainty} 推出. 同样的方法可以得到如下观测区域为可测集外的能观测不等式: 设 $A,B\subset \R^n$为测度有限的可测集, 即 $|A|,|B|<\infty$, 则对任意的 $t>0$和任意满足方程 \eqref{sch} 的解都有
    \begin{equation}
        \int_{\R^n}|u_0(x)|^2\d x\le C(t,|A|,|B|)\left(\int_{A^c}|u_0(x)|^2\d x+\int_{B^c}|u(t,x)|^2\d x\right)\label{schobs-2}
    \end{equation}
    成立, 其中 $A^c$表示集合 $A$ 在全空间 $\R^n$ 中的补集, $C(t,|A|,|B|)$ 是常数.
    
    KdV方程作为最重要的色散方程之一,  一个自然的想法便是对其也建立类似的能观测不等式. 特别地, 我们的前两个主要结果便是对下述的线性KdV方程
    \begin{equation*}
        \partial_t u+\partial_x^3 u=0,\quad u(0,x)=u_0(x)\in L^2(\R),
    \end{equation*} 
    尝试建立能观测不等式 \eqref{schobs-2}. 但是KdV方程 \eqref{kdv-r} 不同于薛定谔方程, 没有类似于式 \eqref{idn} 的关系式, 所以上述关于薛定谔方程的能观测不等式 \eqref{schobs} (亦或 \eqref{schobs-2})对KdV方程不适用. 然而就在不久之前, Ze Li 和 Ming Wang 证明了下述关于线性KdV方程 \eqref{kdv-r} 形如式 \eqref{schobs-2} 能观测不等式: 存在一个常数 $C>0$ 使得对于任意的 $r_1,r_2,t>0$ 和任意满足方程 \eqref{kdv-r} 的解 $u(t,x)\in C([0,\infty);L^2(\R))$, 都有
    \begin{equation}
        \int_{\R}|u_0(x)|^2\d x\le Ce^{Ct^{-\frac{4}{3}}\left(r_1^4+r_2^4\right)}\left(\int
        _{|x|\ge r_1}|u_0(x)|^2\d x+\int _{|x|\ge r_2}|u(t,x)|^2\d x\right).\label{kdvobs-1}
    \end{equation}
    该能观测不等式的证明利用了初始函数为紧支撑的KdV方程解的定量解析光滑效应, 以及解析函数的某种定量唯一延拓不等式. 
   
   \subsection{有界情形}
   在线性色散方程中, 由 HUM 方法可得精确可控性与能观测不等式的等价性. 而方程控制理论中精确可控性的研究早于能观测不等式的研究, 并且有着丰富的结果. 故接下来我们首先介绍能观测不等式早期以精确可控性的形式出现的研究进展. Russell 和 Zhang 首先考虑    \cite{russell1993controllability} 了线性 KdV 方程
   \begin{equation}
       \left\lbrace\begin{array}{ll}
           \partial_t u+\partial_{x}^3u=0, & 0<x<2\pi,0<t<T, \\
           \partial_x^k u(t,0)=\partial_x^ku(t,2\pi), & k\in \lbrace0,2\rbrace,\\
           \partial_x u(t,2\pi)-\partial_x u(t,0)=h(t), & 0<t<T,
       \end{array}\right.
   \end{equation}
   并在解满足 $[u]:=\int_0^{2\pi}u\d x=0$ 的情形下证明了某种精确可控性. Roiser \cite{rosier1997exact}在此基础上考虑了线性 KdV 方程
   \begin{equation}
       \left\lbrace
       \begin{array}{ll}
           \partial_t u+\partial_x u+\partial_x^3 u=0, &  \\
           u(t,0)=u(t,L)=0, & \\
           u_x(t,L)=h(t),&\\
           u(0,x)=u_0,
       \end{array}
       \right.
   \end{equation}
   并证明了当 $L\notin \mathcal{N}:=\lbrace 2\pi \sqrt{\frac{k^2+kl+l^2}{3}}: k,l\in \N_+\rbrace$ 时, 控制项 $h(t)$ 的精确可控性.  在 \cite{coron2004exact} 中, 对临界值 $\mathcal{N}$ 中的情形进行了补充讨论, 并得到了初值和目标函数小范围内的精确可控性. 
   
   上述结果都建立在控制函数所在区域不随时间变化的情形下, 即能观测不等式的观测区域不随时间变化. 实际上, 人们发现当控制区域 $\omega=\omega(t)$ 随时间变化时, 可以减少观测区域的维度. Khapalov 首先给出了 \cite{khapalov1995exact,khapalov2017mobile}  抛物和双曲偏微分方程中观测点随时间移动的能观测性定理. Jaming 利用动力学方法给出了\cite{jaming2018dynamical} 几大类常见线性偏微分方程中观测点随时间移动的能观测性定理.
   最近, Jaming 和 Komornik 考虑了\cite{jaming2020moving} 薛定谔方程
   \begin{equation}
   \left\lbrace
   \begin{array}{ll}
              \partial_t u+ i\partial_x^2 u = 0,  & (t,x)\in \R \times (0,2\pi),  \\
       u(t,0)=u(t,2\pi), & t\in \R,\\
       \partial_x(t,0)=\partial_x(t,2\pi), & t\in \R,\\
       u(0,x)=u_0(x), & x\in (0,2\pi),
   \end{array}
   \right.
   \end{equation}
   并用非调和的傅里叶方法 (Ingham 不等式) 证明了当 $a\notin \Z$时, 对任意的 $T>0$ 和方程的解 $u(t,x)$ 都有能观测不等式
   \begin{equation}
       \int_0^{2\pi}|u_0|^2\d x \le C \int_0^T u(t_0+t,x_0-at) \d t
   \end{equation}
   成立. 在此之前, 人们给出的薛定谔有界区域情形下能观测不等式的观测区域都是一段区间, 并且不随时间移动. 
 
 \section{主要结果}
 论文分成两大部分, 第一部分 (第二章) 主要介绍全空间下线性 KdV 方程两点时刻能观测不等式. 第二部分 (第三章) 主要介绍有界区域下线性 KdV 方程观测点随时间移动的可观测性.
 
 在第二章的全空间情形中, 第一节首先给出了两个新的两点时刻能观测不等式, 即定理 \ref{thm-1} 和定理 \ref{thm-2}. 定理 \ref{thm-1} 给出了观测区域为可测集外的两点时刻能观测不等式, 定理 \ref{thm-2} 给出了观测区域为满足某一密度条件 (见定义 \ref{def-1}) 和半轴条件的两点时刻能观测不等式. 第一节中用具体的例子表明前者突破了非观测区域有界的限制, 而后者则突破了非观测区域测度有限的限制, 所以这两个结果不互相包含. 由于两个定理的证明在大的框架下思想相同, 所以 第二节先给出了两点时刻能观测不等式的一般建立方法, 即性质 \ref{prop-T}. 第三节给出了定理 \ref{thm-1} 的完整证明, 第四节给出了定理 \ref{thm-2} 的完整证明.
 
\iffalse 在第三章的有界区域情形中, 第一节首先介绍了 Ingham 不等式及其推广形式, 第二节介绍并证明观测点随时间移动的能观测性定理的主要结果, 即定理 \ref{thm3-2-1}, 定理 \ref{thm-3-2-2} 和定理 \ref{thm-3-2-3}. 定理 \ref{thm3-2-1}
 给出了单个观测点随时间移动的能观测不等式, 证明遵循了 \cite{jaming2020moving} 中对薛定谔方程的处理方法, 即利用观测线段所构成的函数其傅里叶系数的一致分离性. 定理 \ref{thm-3-2-2} 利用Eisenstein 整环的唯一分解性给出了观测区域为两个不同速率移动观测点的能观测不等式判定方法, 并给出了此情形下最简单的反例, 两个移动观测点观测值为零无法推出解函数为零. 由于基于整环唯一分解性的判定方法在大数情形下缺乏可行性, 定理 \ref{thm-3-2-3} 利用 KdV 方程的色散特性给出了更加容易的判定方法.
 \fi 

    


    
    

    \iffalse 论文的第二部分考虑圆周 $\T:= \R / \Z$ 上的线性KdV方程.\cite{Ionescu2006UniquenessPO}
    \fi
    
    
    \section{符号说明}
    
    本文中所提到的可测概念均为勒贝格可测, 例如可测函数即勒贝格可测函数, 可测集即勒贝格可测集. 
    
    在全空间 $\R$ 中, 可测函数 $f\in L^1(\R)$, 其傅里叶变换表示为
    \begin{equation*}
        \widehat{f}(\xi)=\int_{\R} e^{-i\xi x} f(x)\d x,
    \end{equation*}
    对可测函数 $g\in L^1(\widehat{\R}) $, 其傅里叶逆变换表示为
    \begin{equation*}
        \widecheck{g}(x)=\frac{1}{2\pi}\int_{\R} e^{i\xi x} g(\xi) \d \xi.
    \end{equation*}
    
    \begin{definition}
    设 $f$ 为 $\R$ 上的可测函数, 则定义集合
    \begin{equation*}
        \mathrm{supp}\, f= \lbrace x: f(x)\neq 0 \rbrace,
    \end{equation*}
    并称 $\mathrm{supp}\, f$ 为可测函数 $f$ 的支撑.
    \end{definition}
    
    设 $A, B$ 为两个表达式, 我们规定:
    \begin{enumerate}
        \item 若存在一个正常数 $\alpha$ 使得 $A\le \alpha B$, 则记 $A\lesssim B$.
        \item 若 $A\lesssim B$ 和 $B\lesssim A$ 均成立, 则记 $A\asymp B$.
    \end{enumerate}
    
    对集合 $A$, $|A|$ 表示 $A$ 的测度或者元素个数, 可结合上下文确定. 
    
    


    % 更多内容
    \chapter{全空间情形的可观测性}
    本章考虑全空间 $\R $上的线性KdV方程
        \begin{equation}
        \partial_t u+\partial_x^3 u=0,\quad u(0,x)=u_0(x)\in L^2(\R),\label{kdv-r}
    \end{equation} 
    并建立相应的两点时刻能观测不等式能观测不等式. 
    \section{两点时刻能观测不等式}
  由于本章的两个结果所用的证明方法在结构上相同, 我们在本节先陈述这两个结果, 比较他们的异同, 它们的证明被放在本章的第二, 三, 四节. 
  
  下述定理是全空间情形的第一个结果.
    \begin{theorem}\label{thm-1}
     设 $A, B$为 $\R$ 上的测度有限可测集. 则对任意 $t>0$以及任意满足方程 \eqref{kdv-r} 的解$u(t,x)$, 存在常数 $C=C(t,|A|,|B|)>0$使得
     \begin{equation}
         \int_{\R} |u_0|^2 \d x\le C\left(\int_{A^c}|u_0|^2\d x+\int_{B^c}|u(t,x)|^2\d x\right)\label{kdvobs-2}
     \end{equation}
     成立.
    \end{theorem}
    显然, 同不等式 \eqref{sch} 相比, 定理 \ref{thm-1} 在适用范围上显然更加广泛, 即去掉了集合 $A$和$B$有界的限制, 只要求测度有限. 具体地, 取$E=F=\bigcup_{k\in \Z,k\neq 0} \left[k,k+\frac{1}{2k^2}\right]$, 该集合是无界且测度有限的, 见图 \ref{fig3}.
    \begin{figure}
    \centering
   \begin{tikzpicture}
     \draw [very thick] (-3.5,0) -- (3.5,0);
     \draw (3.6,0) node[right] {$\cdots $} ;
     \draw [very thick,->] (4.5,0) -- (7.0,0) node [right] {$x$};
     \draw (-3.6,0) node[left] {$\cdots $};
     \draw [very thick] (-5.5,0) -- (-4.5,0);
     \foreach \x in {-3,...,3} \draw (\x,0.05) -- (\x,-0.05) node[below] {\x};
     \draw (5,0.05) -- (5,-0.05) node[below] {$k$};
     \draw (-5,0.05) -- (-5, -0.05) node[below] {$-k$};
     \draw (6,0.05) -- (6, -0.05) node[below] {$k+1$};
     \draw [ultra thick,color = magenta] (1,0) -- (1.5,0);
     \draw [ultra thick,color =magenta] (-1,0) -- (-0.5,0);
     \draw [ultra thick,color =magenta] (2,0) -- (2+1/4,0);
     \draw [ultra thick,color = magenta] (-2,0) -- (-2+1/4,0);
     \draw [ultra thick,color =magenta] (3,0) -- (3+1/8,0);
     \draw [ultra thick,color =magenta] (-3,0) -- (-3+1/8,0);
     \draw [ultra thick,color =magenta] (5,0)--(5.2,0);
     \draw [ultra thick,color =magenta] (-5,0) -- (-4.8,0);
     \draw [ultra thick,color =magenta] (6,0) -- (6.1,0);
     \foreach \x in {-3, -2,-1,1,2,3} \draw[->,rounded corners] (0,1.4) --(0,1+\x*\x /50) -- (\x,1+\x*\x /50) -- (\x, 0.1);
     \draw (0,1.4) node[above] {$\color{magenta}{A=B}$};
     \draw[->,rounded corners] (0,1.4) -- (0,1+4*4 /50) -- (5,1+4*4/50) -- (5,0.1);
     \draw[->,rounded corners] (0,1.4) -- (0,1+4*4/50)--(-5, 1+4*4/50) -- (-5,0.1);
   \end{tikzpicture}
   \caption{$A=B=\bigcup_{k\in \Z,k\neq 0} \left[k,k+\frac{1}{2k^2}\right]$}
   \label{fig3}
 \end{figure}
 该条件满足定理 \ref{thm-1} 但不满足Ze Li和Ming Wang关于 $A$ 和 $B$ 取有界集的结果 \eqref{kdvobs-1}.
    另一方面, 我们注意到这里的常数 $C$与它的相关项$t$, $A$和$B$之间没有显式的依赖关系, 原因在于我们的证明方法不同于 \eqref{kdvobs-1} 中利用解析估计, 而是用反证法进行了存在性证明. 证明定理 \ref{thm-1} 的核心思路最初\cite{Amrein1977OnSP}被用于 Amrein-Berthier 不确定性原理的证明, 这是该方法第一次被用来证明色散方程的能观测不等式. 证明总共分成三个步骤:
    \begin{enumerate}
        \item[(1)] 证明不等式 \eqref{kdvobs-2} 在
        \begin{equation}
            \|T\|_{L^{2}(\R)\to L^2(\R)}<1\label{sl1}
        \end{equation}
        时成立, 其中 $T=\chi_BS(t)\chi_A$, $S(t)=e^{-t\partial_x^3}$ 表示由式 \eqref{kdv-r} 生成的群.
        \item[(2)] 证明算子 $T$ 是从 $L^2(\R)$ 到 $L^2(\R)$ 的紧算子.
        \item[(3)] 利用(2) 的结论将式 \eqref{sl1} 的证明归结为 $\| T\|_{L^2(\R)\to L^2(\R)}\neq 1$ 的证明.
    \end{enumerate}
    
    在定理 \ref{thm-1} 的证明中, $A$ 和 $B$ 均为测度有限可测集这一假设是必要的, 它在步骤(2)和(3)的证明中都需要被用到. 通过对步骤(2) 
    
    更进一步地观察, 我们发现, 如果 $A$ 和 $B$ 满足某种密度条件, 仍然可以保证算子 $T$ 是紧算子. 由于在该密度条件下, $A$ 和 $B$ 的测度可以不是有限的, 我们便可以建立KdV方程新的两点能观测不等式, 这就得到了我们的第二个主要结果. 为了陈述该结果, 我们首先给出关于一个关于集合密度的新定义.
    \begin{definition}\label{def-1}
    设 $A\subset \R$ 为可测集, 若其满足条件
    \begin{equation*}
         \varlimsup_{x\to \infty}|A\cap[x,x+1]|\cdot|x|^{\alpha}\lesssim 1,
    \end{equation*}
    则称 $A$ 具有 $|x|^{-\alpha}$密度.
    \end{definition}
    \begin{remark}\label{rem-1}
    该集合密度的定义可以等价地陈述为: 存在常数 $L>0$ 使得对于任意满足 $|x|\ge L$ 的 $x$, 都有
    \begin{equation*}
        |A\cap [x,x+1]|\cdot |x|^\alpha\lesssim |x|^{-\alpha}.
     \end{equation*}
    在第二个结果亦即定理 \ref{thm-2} 的证明中(第二章, 第三节), 我们主要利用该集合密度定义的等价陈述.
    \end{remark}
    
    有了上述定义, 我们就可以将第二个结果陈述为下述定理.
    \begin{theorem}\label{thm-2}
     设 $A$ 和 $B$ 为密度 $|x|^{-\alpha}, \alpha>\frac{5}{6}$的可测集, 并且对某个常数 $c\in\R$ 我们有 $A, B \subset (c,\infty)$ 或者 $A, B\subset (-\infty ,c)$. 则存在常数 $C=C(t,A,B)>0$, 使得对任意的 $t>0$以及任意满足方程 \eqref{kdv-r} 的解 $u(t,x)$,  不等式 \eqref{kdvobs-2} 仍然成立.
    \end{theorem}
    在定理 \ref{thm-2} 的条件中, 我们最感兴趣的情形是 $\alpha\in \left(\frac{5}{6},1\right]$. 事实上, 当 $\alpha>1$ 时 $A$和$B$ 都是具有有限测度的可测集, 相应的能观测不等式可以直接由定理 \ref{thm-1} 得到. 为了说明定理 \ref{thm-2} 与定理 \ref{thm-1} 的区别, 我们令 $\alpha\in \left(\frac{5}{6},1\right]$ 并取
    \begin{equation*}
        A=B=\bigcup_{k=1}^\infty [k,k+2k^{-\alpha}].
    \end{equation*}
    该情形下 $A$ 和 $B$ 满足定理 \ref{thm-2} 的假设,  然而它们都具有无限测度, 见图 \ref{fig4}.
    \begin{figure}[ht]
    \centering
   \begin{tikzpicture}
     \draw [very thick] (-0.5,0) -- (3.5,0);
     \draw (3.6,0) node[right] {$\cdots $};
     \draw [very thick,->] (4.5,0) --(7,0) node [right] {$x$};
     \foreach \x in {0,1,2,3} \draw (\x,0.05) -- (\x,-0.05) node[below] {\x};
     \draw (5,0.05)--(5,-0.05) node[below]{$k$};
     \draw (6,0.05)--(6,-0.05) node[below]{$k+1$};
     \draw [ultra thick,color=magenta] (1,0) -- (1.5,0);
     \draw [ultra thick, color=magenta] (2,0) -- (2.4,0);
     \draw [ultra thick,color=magenta] (3,0) -- (3.2,0);
     \draw [ultra thick,color=magenta] (5,0) -- (5.15,0);
     \draw [ultra thick,color=magenta] (6,0) -- (6.1,0);
     \draw [->, rounded corners] (2.5,1.4) -- (2.5,1+16 /50) -- (1,1+16/50) -- (1,0.1);
     \draw [->,rounded corners] (2.5,1.4) -- (2.5,1+8 /50) -- (2,1+8/50) -- (2,0.1);
     \draw [->, rounded corners] (2.5, 1.4) -- (2.5,1+4/50)--(3,1+4/50)--(3,0.1);
     \draw [->,rounded corners] (2.5,1.4) -- (2.5,1+9/50)--(5,1+9/50)--(5,0.1);
     \draw [->,rounded corners] (2.5,1.4)--(2.5,1+16/50)--(6,1+16/50)--(6,0.1);
     \draw (2.5,1.4) node[above] {$\color{magenta}{A=B}$};
   \end{tikzpicture}
   \caption{$A=B=\bigcup_{k=1}^\infty [k,k+2k^{-\alpha}]$}
   \label{fig4}
 \end{figure}
    因此, 相较于定理 \ref{thm-1}, 定理 \ref{thm-2} 给出了全新的两点时刻能观测不等式. 另一方面, 定理 \ref{thm-2} 中$A$和$B$ 被限制在半直线上, 这一限制条件来源于我们的证明中需要用到KdV方程在两点时刻半轴上的唯一延拓性. 目前我们还不知道去掉该条件后, 定理 \ref{thm-2} 是否仍然成立. 
    
    显然, 定理 \ref{thm-1} 和定理 \ref{thm-2} 都是下述KdV方程唯一延拓性的定量版本: 若 $u(0,\cdot)$ 在 $A^c$上恒为零并且 $u(t,\cdot),t\neq 0$在 $B^c$上恒为零, 则 $u(t,x)\equiv 0$. 类似的唯一延拓性结果见 Bingyu Zhang 的文章\cite{Zhang1992UniqueCF,Zhang1997Unique}. 另一种假设$u(0,x)$ 和 $u(t,x)$ 具有某种指数衰减的唯一延拓性, 在薛定谔方程情形下被 Escauriaza, Kenig, Ponce 和 Vega \cite{Escauriaza2007OnUP}证明.
    
    实际上, 证明上述两个主要结果的方法具有一般性, 它可用来处理其它不同类型的色散方程, 例如更高阶的KdV方程以及薛定谔方程. 

    \section{一般判据}
    本节中, 我们给出线性KdV方程两点时刻能观测不等式的一般判定方法. 
    设$S(t)=e^{-t\partial^3_x}$ 为线性KdV方程 \eqref{kdv-r} 生成的群. 则方程 \eqref{kdv-r} 的解可写作
    \begin{align}\label{equ-kdv-solu}
    u(t)=S(t)u_0=\int_\R G(t,x-y)u_0(y)\d y,
    \end{align}
    其中 $G$是线性KdV方程 \eqref{kdv-r} 的基本解, 其具体形式为
    \begin{align}\label{equ-kdv-ker}
    G(t,x)= \left\{
        \begin{array}{ll}
        \frac{1}{(3t)^{\frac{1}{3}}}\operatorname{Ai}(\frac{x}{(3t)^{\frac{1}{3}}}), & \hbox{ }t> 0, \\
        \delta(x), & \hbox{ }t=0.
        \end{array}
        \right.
    \end{align}
    这里, $\operatorname{Ai}(x)$是 Airy函数, 其具体形式为
    \begin{align*}
    \operatorname{Ai}(x)= \frac{1}{2\pi}\int^{\infty}_{-\infty}e^{i(xz+\frac{1}{3}z^3)}\d z.
    \end{align*}
    根据\cite[p.~330]{SteinShakarchi2010}, 存在常数 $C>0$ 使得
    \begin{align}\label{equ-Ai}
|\operatorname{Ai}(x)| \leq
\left\{
\begin{array}{ll}
C(1+|x|)^{-\frac{1}{4}}, &   x<0, \\
 Ce^{-\frac{2}{3}|x|^{\frac{3}{2}}}, &  x\geq 0.
\end{array}
\right.
\end{align}
本章的第2节和第3节将用到式 \eqref{equ-kdv-ker} -- 式 \eqref{equ-Ai}. 线性KdV方程 \eqref{kdv-r} 的解是$L^2$范数守恒的, 即 
\begin{align}\label{equ-conservation-law}
\|S(t)u_0\|_{L^2(\R)}=\|u(t,\cdot)\|_{L^2(\R)}=\|u_0\|_{L^2(\R)}, \quad t\in \R,
\end{align}
我们把这一性质称为线性KdV方程的守恒律.

设$A$和$B$为$\R$中的可测子集, 固定 $t\in \R$, 定义线性算子 $T:L^2(\R)\mapsto L^2(\R)$
\begin{align}\label{equ-T-def}
Tf = \chi_BS(t)\big( \chi_A f \big), \quad f\in L^2(\R).
\end{align}
在这里, 示性函数$\chi_A(x)$ 满足 $\chi_A(x)=1$ 若 $x\in A$ 及 $\chi_A(x)=0$ 若 $x\notin A$. 利用守恒律 \eqref{equ-conservation-law}, 易得
\begin{align}\label{equ-T-norm}
\|T\|_{L^2(\R)\to L^2(\R)}\leq 1.
\end{align}

\begin{proposition}\label{prop-T}
设 $\|T\|_{L^2(\R)\to L^2(\R)}< 1$. 则存在常数 $C>0$ 使得对方程 \eqref{kdv-r} 所有的解 $u(t,x)$ 都有
$$
    \int_\R |u_0|^2\d x \leq C\left( \int_{A^c}|u_0|^2\d x + \int_{B^c}|u(t,x)|^2\d x \right).
$$
\end{proposition}
\begin{proof}
假设 $\|T\|_{L^2(\R)\to L^2(\R)}=c_1$, 其中 $c_1$ 满足 $0\leq c_1<1$. 由定义式 \eqref{equ-T-def} 可得
$$
\|\chi_B(x)S(t)(\chi_Au_0)\|_{L^2(\R)}\leq c_1\|u_0\|_{L^2(\R)}, \quad \forall u_0\in L^2(\R).
$$
从而
 $$
\|\chi_B(x)S(t) \chi_Au_0\|_{L^2(\R)}\leq c_1\|\chi_Au_0\|_{L^2(\R)}=c_1\|S(t) (\chi_Au_0)\|_{L^2(\R)}, \quad \forall u_0\in L^2(\R),
$$
上述最后一步用到了  $\|\chi_Au_0\|_{L^2(\R)}=\|S(t) (\chi_Au_0)\|_{L^2(\R)}$ (利用守恒律 \eqref{equ-conservation-law}). 由此我们可以得到
\begin{align*}
\|S(t) (\chi_Au_0)\|_{L^2(\R)}&\leq \|\chi_B(x)S(t) (\chi_Au_0)\|_{L^2(\R)}+\|\chi_{B^c}(x)S(t) (\chi_Au_0)\|_{L^2(\R)}\\
&\leq c_1\|S(t) (\chi_Au_0)\|_{L^2(\R)}+\|\chi_{B^c}(x)S(t) (\chi_Au_0)\|_{L^2(\R)}.
\end{align*}
设  $c_2=1/(1-c_1)$, 则由上式进一步可得
\begin{align}\label{equ-5}
 \|S(t) (\chi_Au_0)\|_{L^2(\R)}\leq c_2\|S(t) (\chi_Au_0)\|_{L^2(B^c)}, \quad \forall u_0\in L^2(\R).
\end{align}
现在我们可以由式 \eqref{equ-5} 得
\begin{align*}
\|u_0\|_{L^2(\R)} &= \|S(t)u_0\|_{L^2(\R)}\leq \|S(t) (\chi_Au_0)\|_{L^2(\R)}+\|S(t) (\chi_{A^c}u_0)\|_{L^2(\R)}\\
&\leq c_2\|S(t) (\chi_Au_0)\|_{L^2(B^c)}+\|S(t) (\chi_{A^c}u_0)\|_{L^2(\R)}\\
&\leq c_2\|S(t)u_0\|_{L^2(B^c)}+(1+c_2)\|S(t) (\chi_{A^c}u_0)\|_{L^2(\R)}\\
&=c_2\|u(t,\cdot)\|_{L^2(B^c)}+(1+c_2)\|u_0\|_{L^2(A^c)}.
\end{align*}
证明完毕 .
\end{proof}



    \section{定理 \ref{thm-1} 的证明}
    由性质 \ref{prop-T} 可知, 线性KdV方程两点时刻能观测不等式可由不等式 $\|T\|_{L^2(\R)\to L^2(\R)}< 1$推出. 为了便于后面证明叙述, 我们先证明不等式
\begin{align}\label{equ-ST-0}
\|S(-t)T\|_{L^2(\R)\to L^2(\R)}<1,
\end{align}
再利用
\begin{align}\label{equ-ST}
\|T\|_{L^2(\R)\to L^2(\R)}=\|S(-t)T\|_{L^2(\R)\to L^2(\R)},
\end{align}
得到 $\|T\|_{L^2(\R)\to L^2(\R)}< 1$. 这里等式 \eqref{equ-ST} 再次用到了守恒律 \eqref{equ-conservation-law}. 

本节主要目的是证明在$A$ 和 $B$均为有限测度可测集的条件下, 式 \eqref{equ-ST-0} 成立. 我们采用 Amrein 和 Berthier 在 \cite{Amrein1977OnSP}中所用到的方法, 由性质 \ref{prop-T} 和式 \eqref{equ-ST} 易知, 定理  \ref{thm-1} 可归结为对下述性质的证明.
\begin{proposition}\label{prop-3}
设 $A$, $B$ 为 $\R$ 中具有有限测的可测集, 即 $|A|,|B|<\infty$.  再设 $S(t)$ 和 $T$ 分别由式 \eqref{equ-kdv-solu} 和式 \eqref{equ-T-def} 给出. 则对任意的 $t>0$ 我们有
$$
\|S(-t)T\|_{L^2(\R)\to L^2(\R)}<1.
$$
\end{proposition}
为了证明性质 \ref{prop-3}, 首先我们需要先证明一些引理.

\begin{lemma}\label{lem-T-comp}
对任意的 $t>0$, $T$ 是从 $L^2(\R^n)$ 到 $L^2(\R^n)$ 上的紧算子.
\end{lemma}
\begin{proof}
根据式 \eqref{equ-kdv-solu} 和式 \eqref{equ-T-def}, 算子$T$ 可以被重写为积分的形式:
$$
(Tf)(x)=\int_{\R }\chi_A(x)G(t,x-y)\chi_B(y)f(y)\d y:=\int_{\R } K(t,x,y)f(y)\d y.
$$
若对任意的 $t>0$, 都有
\begin{align}\label{equ-10}
 \int_{\R }\int_{\R } K^2(t,x,y)\d x \d y<\infty.
\end{align}
则算子 $T$ 是在 $L^2(\R )$上的 Hilbert-Schmidt 算子, 进而由\cite[p.~277]{Yosida1999}可得其为紧算子.

剩下的只需要说明式 \eqref{equ-10} 成立. 事实上, 对任意的 $t>0$, 由式 \eqref{equ-kdv-ker} 和 式\eqref{equ-Ai} 可得 $|G(t,x-y)|\leq C(t)$, 这里 $C(t)>0$ 是一个仅仅依赖于$t$ 的常数. 进而有
\begin{align*}
 \int_\R\int_\R K^2(t,x,y)\d x \d y\leq  C^2(t) \int_\R\int_\R \chi_A(x)\chi_B(y)d x \d y=C^2(t)|A||B|<\infty.
\end{align*}
由此我们便证明了式 \eqref{equ-10}.
\end{proof}

设 $f$ 为 $\R$ 上的可测函数, $A$ 为可测集. 若
$$
f(x)=0, \quad a.e. \, x\in A^c,
$$
则我们称支撑 $\mathrm{supp } \, f\subset A$. 

\begin{lemma}\label{lem-2}
设函数$f\in L^2(\R)$ 且$\|\chi_BS(t)(\chi_Af)\|_{L^2(\R)}=\|f\|_{L^2(\R)}$,  则有 $\mathrm{supp } \, f\subset A$
且 $\mathrm{supp } \, S(t)f \subset B$.
\end{lemma}
\begin{proof}
第一步先证明
\begin{align}\label{equ-425-1}
\mathrm{supp }  \, f\subset A.
\end{align}
这里我们用反证法, 假设
$$
\Big|\{x\in \R: |f(x)|>0\} \backslash A \Big|>0
$$
成立, 则
\begin{align}\label{equ-425-2}
\|f\|_{L^2(A^c)}>0.
\end{align}
根据引理假设可得
$$
\|f\|_{L^2(\R)}=\|\chi_BS(t)(\chi_Af)\|_{L^2(\R)}\leq \|\chi_Af\|_{L^2(\R)},
$$
从而 $\|f\|_{L^2(A^c)}=0$. 但这与式 \eqref{equ-425-2} 矛盾. 所以式 \eqref{equ-425-1} 成立.

第二步再证明
\begin{equation}\label{equ-4225-3}
    \mathrm{supp}\, S(t)f\subset B.
\end{equation}
利用式 \eqref{equ-425-1} 及引理假设, 可得
$$
\|\chi_BS(t)f\|_{L^2(\R)}=\|\chi_BS(t)(\chi_Af)\|_{L^2(\R)}=\|f\|_{L^2(\R)}=\|S(t)f\|_{L^2(\R)}.
$$
上式可推出式 \eqref{equ-4225-3}.
\end{proof}

设 $\lambda\in \R$ 和 $A\subset\mathbb{R}$, 我们对集合 $A$ 定义集合
$$A+\lambda=\{x\in \mathbb{R}\lvert x+\lambda \in A\}.$$
上述集合的定义表示对集合 $A$ 作大小为 $\lambda$ 的平移之后的新集合.
\begin{lemma}\label{lem-3}
设 $C_0$ 和 $C$ 是 $\R$ 中两个可测集, 并且满足 $0<|C_0|,|C|<\infty$ 和 $C_0\subset C$. 定义函数
$$
h_C(\lambda)=|C\cup (C_0+\lambda)|, \quad \lambda\geq 0.
$$
则 $h_C$ 是关于 $\lambda$ 的连续函数, $h(0)=|C|$ 且  $\lim_{\lambda\to \infty}\limits h_C(\lambda)=|C|+|C_0|$.
\end{lemma}
\begin{proof}
见 \cite[p.~99]{Havin2012}.
\end{proof}

我们还需要下述引理, 它关于一个集合的版本见 \cite[定理 1]{Amrein1977OnSP}.

\begin{lemma}\label{lem-4}
设 $C_0,C,D_0,D$为 $\R$ 上的四个可测集,且满足条件$C_0\subset C, D_0\subset D$ 和 $0<|C_0|, |C|$, $|D_0|, |D|<\infty$. 则对任意的 $\varepsilon>0$, 存在一个大小为 $\lambda>0$ 的平移 使得
\begin{align}\label{equ-426-1}
   |C\cup (C_0+\lambda)|\leq|C|+\varepsilon, \quad    |D\cup (D_0+\lambda)|\leq|D|+\varepsilon
\end{align}
成立, 并且
\begin{align}\label{equ-426-2}
  |C|<|C\cup (C_0+\lambda)|
\end{align}
和
\begin{align}\label{equ-426-3}
 |D|< |D\cup (D_0+\lambda)|
\end{align}
两者至少有一个严格不等式成立.
\end{lemma}
\begin{proof}
定义下述两个函数
$$
h_C=|C\cup (C_0+\lambda)|, \quad h_D=|D\cup (D_0+\lambda)|, \quad \lambda\geq 0.
$$
固定 $\varepsilon>0$, 不失一般性, 设 
$$0<\varepsilon<\min \lbrace|C|+|C_0|, |D|+|D_0|\rbrace.$$ 考虑集合
$$
\{h_C=|C|+\varepsilon\}=\{\lambda\geq 0: h_C(\lambda)=|C|+\varepsilon\}.
$$
由引理 \ref{lem-3} 可得, $h_C$ 是一个关于 $\lambda$ 的连续函数, 且满足$h(0)=|C|$ 和 $\lim_{\lambda\to \infty}\limits h_C(\lambda)=|C|+|C_0|$. 从而 $\{h_C=|C|+\varepsilon\}$ 是非空闭集. 令
$$
\lambda'=\min_{\lambda\in h_C=|C|+\varepsilon} \lambda.
$$
再次利用函数 $h_C$ 的连续性, 可得
\begin{align}\label{equ-426-4}
|C|\leq h_C(\lambda)\leq|C|+\varepsilon, \quad \forall \lambda\in[0,\lambda'].
\end{align}
接下来我们分成两种情形讨论.

{\bf 情形 1:} $h_D(\lambda')\leq |D|+\varepsilon$. 该情形下由 $h_C(\lambda')=|C|+\varepsilon$, 可得式 \eqref{equ-426-1} 和式 \eqref{equ-426-2} 在$\lambda=\lambda'$时成立.

{\bf 情形 2:} $h_D(\lambda')>|D|+\varepsilon$. 对函数 $h_D$ 用引理 \ref{lem-3}, 可得对某个 $\lambda''\in(0,\lambda')$ 有
$$
 h_D(\lambda'')=|D|+\varepsilon.
$$
 再利用式 \eqref{equ-426-4} 可得
$$
|C|\leq h_C(\lambda'')\leq|C|+\varepsilon.
$$
因此, 该情形下式 \eqref{equ-426-1} 和式 \eqref{equ-426-3} 在 $\lambda=\lambda''$ 时成立.
\end{proof}

有了上述引理, 我们便可以完成性质 \ref{prop-3} 的证明.\\
{\bf {性质 \ref{prop-3} 的证明}.} 
首先, 由守恒律 \eqref{equ-conservation-law} 和式 \eqref{equ-T-norm}, 可得
$$
\|S(-t)T\|_{L^2(\R)\to L^2(\R)}\leq 1.
$$
因此只要说明对任意的 $t>0$ 都有
\begin{align}\label{equ-4-23-1}
\|S(-t)T\|_{L^2(\R)\to L^2(\R)}\neq 1,
\end{align}
便可以完成性质 \ref{prop-3} 的证明.


为了证明式 \eqref{equ-4-23-1}, 下面我们使用反证法. 固定 $t>0$, 
假设我们有
\begin{equation}\label{assum}
    \|S(-t) T \|_{L^2(\R)\to L^2(\R)}=1.
\end{equation}
由 $\|S(-t)\|_{L^2(\R)\to L^2(\R)}=1$ 以及由引理 \ref{lem-T-comp} 得到的$T$ 是 $L^2(\R)$ 上的紧算子这一事实, 易得 $S(-t)T$ 也是  $L^2(\R)$上的紧算子. $S(-t)T$是紧算子的事实再加上反证假设 \eqref{assum}, 可得存在一个函数 $f\in L^2(\R)$ 使得
\begin{align}\label{equ-425-3}
\|S(-t)Tf\|_{L^2(\R)}=\|f\|_{L^2(\R)}=1
\end{align}
成立.
上式以及 $\|S(-t)Tf\|_{L^2(\R)}=\|Tf\|_{L^2(\R)}$可得 $\|Tf\|_{L^2(\R)}=\|f\|_{L^2(\R)}$. 进而再由引理 \ref{lem-2} 可得
\begin{align}\label{equ-425-4}
\mathrm{supp }\, f\subset A, \quad \mathrm{supp }\,  S(t)f\subset B.
\end{align}
设
$$
A_0=\mathrm{supp }\, f, \quad B_0=\mathrm{supp }\,  S(t)f.
$$
根据引理中 $A,B$ 测度有限的假设和式 \eqref{equ-425-4}, 易知
\begin{align}\label{equ-425-10}
0<|A_0|, |B_0|<\infty.
\end{align}
 由此我们可以利用引理 \ref{lem-4} 得到一组序列 $\{\lambda_i\}_{i=1}^\infty\subset(0,\infty)$ 以及集合序列 $\lbrace A_i\rbrace_{i=1}^\infty$ 和 $\lbrace B_i\rbrace_{i=1}^\infty$, 使其满足
\begin{align}\label{equ-425-5}
|A_{i-1}\cup (A_0+\lambda_i)|\leq |A_{i-1}|+\frac{1}{2^i}, \quad   |B_{i-1}\cup (B_0+\lambda_i)|<|B_{i-1}|+\frac{1}{2^i},
\end{align}
且有
\begin{align}\label{equ-425-6}
|A_{i-1}|< |A_{i-1}\cup (A_0+\lambda_i)|  \quad  \mbox{ 或 } \quad |B_{i-1}|< |B_{i-1}\cup (B_0+\lambda_i)|
\end{align}
二者之一成立. 其中 $i=1,2,\cdots$, 集合序列 $\lbrace A_i\rbrace_{i=1}^\infty$ 和 $\lbrace B_i\rbrace_{i=1}^\infty$ 通过下述的递归定义得到 
\begin{align}\label{equ-425-7}
A_i=A_{i-1}\cup  (A_0+\lambda_i), \quad B_i=B_{i-1}\cup (B_0+\lambda_i).
\end{align}
由式 \eqref{equ-425-5} -- 式 \eqref{equ-425-7} 可得
\begin{align}\label{equ-425-8}
|\bigcup_{i=0}^{\infty}A_i|\leq |A|+1,\quad |\bigcup_{i=0}^{\infty}B_i|\leq|B|+1.
\end{align}
令 $f_0=f$ 以及
$$
f_i=\mathcal{T}_{\lambda_i}f, \quad i=1,2,\cdots.
$$
这里$\mathcal{T}_\lambda$ 表示平移算子, 即
\begin{equation*}
    \mathcal{T}_{\lambda}f(x)=f(x-\lambda).
\end{equation*}
易知   $\mathrm{supp}\, f_\lambda=A_0+\lambda$ 且 $S(t)f_\lambda=\mathcal{T}_{\lambda}(S(t)f)$, 从而  $\mathrm{supp}\,S(t)f_\lambda= B_0+\lambda$. 进而对所有的 $i=1,2,\cdots$, 我们有
\begin{align}\label{equ-426-5}
\mathrm{supp }\,f_i=A_0+\lambda_i\subset A_i, \quad \mathrm{supp}\,S(t)f_i =B_0+\lambda_i\subset B_i.
\end{align}
令 $\mathcal {A}=\bigcup_{i=0}^{\infty}A_i$ 以及 $\mathcal {B}=\bigcup_{i=0}^{\infty}B_i$. 我们接下来说明下述三个结论:
\begin{itemize}
  \item [(i)] 序列 $\{ f_i\}_{i=0}^{\infty}$ 是线性无关的.
  \item [(ii)]  算子 $S(-t)\chi_\mathcal {B}S(t)\chi_\mathcal {A}$ 是 $L^2(\R)$ 上的紧算子.
  \item [(iii)]  对每一个 $i=0,1,\cdots,$ $f_i$ 是算子 $S(-t)\chi_\mathcal {B}S(t)\chi_\mathcal {A}$ 关于特征值 $1$ 的特征函数.
\end{itemize}

对于 (i), 首先固定 $m\in \mathbb{N}$. 由式 \eqref{equ-425-6} 和式 \eqref{equ-426-5} 可得 $\chi_{A_m\setminus A_{m-1}}f_m\neq 0$ 和 $\chi_{B_m\setminus B_{m-1}}S(t)f_m\neq 0$ 中至少有一个成立. 不论何种情况成立,  $f_m$ 都不可能是 $f_0,f_1,\cdots,f_{m-1}$ 的线性组合. 这说明对任意的 $m\in \N$, 序列 $\{f_i\}_{i=0}^{m}$ 是线性无关的, 所以 (i) 成立.

对于 (ii), 我们首先注意到 $|\mathcal {A}|,|\mathcal {B}|<\infty$. 再由引理 \ref{lem-T-comp} 可得算子 $S(-t)\chi_\mathcal {B}S(t)\chi_\mathcal {A}$ 为紧算子.

对于 (iii),  由式 \eqref{equ-426-5}, 以及 $\mathcal {A}$ 和 $\mathcal {B}$ 的定义, 可得对所有的$i=0,1,\cdots$都有
$$
\mathrm{supp }\, f_i\subset \mathcal {A}, \quad \mathrm{supp }\,  S(t)f_i\subset \mathcal {B}.
$$
因此
$$
S(-t)Tf_i=S(-t)\chi_{\mathcal {B}}S(t)(\chi_{\mathcal {A}}f_i)=S(-t)\chi_{\mathcal {B}}S(t)f_i=S(-t)S(t)f_i=f_i.
$$
亦即, 对任意的 $i$, $f_i$ 是算子 $S(-t)\chi_\mathcal {B}S(t)\chi_\mathcal {A}$ 关于特征值 $1$ 的特征函数. 所以 (iii) 成立.

最后, (i) 和 (iii) 说明算子 $S(-t)\chi_\mathcal {B}S(t)\chi_\mathcal {A}$  关于特征值 $1$ 有无穷多个线性无关的特征函数, 这和 (ii), 也就是算子的紧性相矛盾. 这说明假设  \eqref{assum} 不成立. 从而 式 \eqref{equ-4-23-1} 成立. $\hfill\Box$





    \section{定理 \ref{thm-2} 的证明}
    本节中我们会证明一个比定理 \ref{thm-2} 更加广泛的定理. 事实上, 若 $\alpha=\beta>\frac{5}{6}$, 则 $(\alpha,\beta)$ 满足不等式组
$$
  ({\bf H}) \qquad   \qquad \qquad   \begin{cases}
          \alpha+\beta>\frac{5}{3}&\\
          \alpha+3\beta>3&\\
          3\alpha+\beta>3&\\
          \alpha,\beta>\frac{1}{2}&
       \end{cases}.
        \qquad \qquad \qquad
$$
因此定理 \ref{thm-2} 是下述定理的直接推论.

\begin{theorem}\label{thm-4}
设 $(\alpha,\beta)$ 满足条件 $({\bf H})$. 再设  $A$ 和 $B$ 分别是具有密度 $|x|^{-\alpha}$ 和 $|x|^{-\beta}$的可测集, 并且对某个常数$c\in \R$ 有  $A,B\subset (c,\infty)$ 或者 $A,B\subset (-\infty,c)$. 则对任意 $t>0$, 存在常数 $C=C(t,A,B)>0$ 使得对所有满足KdV方程 \eqref{kdv-r} 的解$u(t,x)$都有下述连点时刻能观测不等式成立
$$
    \int_\R |u_0|^2\d x \leq C\left( \int_{A^c}|u_0|^2\d x + \int_{B^c}|u(t,x)|^2\d x \right).
$$
\end{theorem}

证明该定理的核心思想依然是利用性质 \ref{prop-T}, 即证明 $\|T\|_{L^2(\R)\to L^2(\R)}<1$, 其中
\begin{align}\label{equ-51-1}
Tf=\chi_BS(t)\chi_A f = \int_\R K(t,x,y)f(y)\d y, \quad f\in L^2(\R)
\end{align}
以及
\begin{align}\label{equ-51-2}
K(t,x,y)=\chi_B(x)G(t,x-y)\chi_A(y)f(y).
\end{align}
这里 $S(t)=e^{-t\partial_x^3}$, $G$ 由式 \eqref{equ-kdv-ker} 给出.

\begin{lemma}\label{lem-comp-2}
设 $(\alpha,\beta)$ 满足条件 $({\bf H})$. 再设集合  $A$ 和 $B$ 分别是具有密度 $|x|^{-\alpha}$ 和 $|x|^{-\beta}$的可测集. 则由式 \eqref{equ-51-2} 给出的核函数 $K$ 满足
    $$
       \int_{\mathbb{R}}\int_{\mathbb{R}}K^2(t,x,y)\d x\d y<\infty.
    $$
 \end{lemma}

\begin{proof}
由式 \eqref{equ-51-2}, 估计式 \eqref{equ-kdv-ker} 和 \eqref{equ-Ai} 可知, 我们只需要证明
    \begin{equation}
       \int_{\mathbb{R}}\int_{\mathbb{R}}\chi_B(x)\langle|x-y|\rangle^{-\frac{1}{2}}\chi_A(y)\d x\d y<\infty.\label{eqn1-1}
    \end{equation}
这里以及往后的论述中, $\langle x \rangle = 1+|x|$.
    将式 \eqref{eqn1-1} 左边改写为
    \begin{equation}
       \sum_{j\in \mathbb{Z}}\sum_{k\in\mathbb{Z}}\int_j^{j+1}\int_k^{k+1}\chi_B(x)\langle|x-y|\rangle^{-\frac{1}{2}}\chi_A(y)\d x\d y.\label{eqn1-2}
    \end{equation}
    然而当 $x\in [k,k+1]$, $y\in[j,j+1]$ 时, 有
    $$
       \langle|x-y|\rangle\asymp \langle|j-k|\rangle.
    $$
    现在对式 \eqref{eqn1-2} 的上界进行估计可得
    \begin{align}
       &\sum_{j\in \mathbb{Z}}\sum_{k\in\mathbb{Z}}\int_j^{j+1}\int_k^{k+1}\chi_B(x)\langle|j-k|\rangle^{-\frac{1}{2}}\chi_A(y)\d x\d y\notag\\
       =&\sum_{j\in\mathbb{Z}}\sum_{k\in\mathbb{Z}}|B\cap [k,k+1]|\cdot \langle|j-k|\rangle^{-\frac{1}{2}}\cdot |A\cap [j,j+1]|\notag\\
       \lesssim & 1+\sum_{j,k\in\mathbb{Z},|j|,|k|\ge L} |B\cap[k,k+1]|\cdot \langle|j-k|\rangle^{-\frac{1}{2}}\cdot|A\cap [j,j+1]|\notag\\
       \lesssim & \sum_{j,k\in\mathbb{Z},|j|,|k|\ge L} |j|^{-\alpha}|k|^{-\beta}\langle |j-k|\rangle^{-\frac{1}{2}}.\label{eqn1-3}
    \end{align}
在式 \eqref{eqn1-3} 的最后一行中, 我们用到了注 \ref{rem-1} 中的等价定义.
 将式 \eqref{eqn1-3} 的最后一行和式分为两部分
 \begin{align}\label{equ-426-10}
 \sum_{j,k\in\mathbb{Z},|j|,|k|\ge L} |j|^{-\alpha}|k|^{-\beta}\langle |j-k|\rangle^{-\frac{1}{2}}:= I+II,
 \end{align}
 其中 $I, II$ 分别取  $|j|\leq |k|$ 和 $|j|>|k|$ 两种情况下的和式.

{\bf 式 $I$ 的估计.} 设 $\delta\in (0,1)$, 它的值在稍后确定. 进一步把和式分解成
$I:=I_1+I_2$,  其中
 \begin{align}
I_1 &= \sum_{|j|,|k|\ge L, |k|-|k|^\delta\leq |j|\leq |k|} |j|^{-\alpha}|k|^{-\beta}\langle |j-k|\rangle^{-\frac{1}{2}},\label{equ-426-14}\\
I_2 &= \sum_{|j|,|k|\ge L, |j|> |k|-|k|^\delta} |j|^{-\alpha}|k|^{-\beta}\langle |j-k|\rangle^{-\frac{1}{2}}.\label{equ-426-15}
 \end{align}
对于式 $I_1$,  因为 $|k|-|k|^\delta\le |j|\le |k|$, 我们有
\begin{align}\label{equ-426-16}
       I_1       \le  \sum_{|k|\ge L,|k|-|k|^\delta\le|j|\le |k|}|j|^{-\alpha}|k|^{-\beta}
       \lesssim  \sum_{|k|\ge L}|k|^{\delta-\alpha}|k|^{-\beta}=\sum_{|k|\ge L}|k|^{\delta-\alpha-\beta}.
\end{align}
对于式 $I_2$, 因为 $|j|<|k|-|k|^\delta$, 我们有 $|k-j|\ge |k|-|j|\ge |k|^\delta$, 进而  $\langle |j-k| \rangle^{-1}\lesssim|k|^{-\delta}$. 在这里我们分成两种情况进行讨论:
     \begin{itemize}
        \item  $0<\alpha\leq 1$ 时, 我们有
     \begin{align}
        I_2&= \sum_{|k|\ge L,|j|<|k|-|k|^\delta}|j|^{-\alpha}|k|^{-\beta}\langle |j-k|\rangle^{-\frac{1}{2}}\lesssim \sum_{|k|\ge L,|j|<|k|-|k|^{\delta}}|j|^{-\alpha}|k|^{-\beta-\frac{1}{2}\delta}\nonumber\\
        &\lesssim
     \left\{
     \begin{array}{ll}
     \sum_{|k|\ge L }\limits|k|^{1-\alpha}|k|^{-\beta-\frac{1}{2}\delta}=\sum_{|k|\ge L}\limits|k|^{1-\frac{1}{2}\delta-\alpha-\beta}, \quad & 0<\alpha<1, \vspace{1ex}\\
      \sum_{|k|\ge L }\limits\ln|k|\cdot|k|^{-\beta-\frac{1}{2}\delta}, \quad & \alpha=1.
     \end{array}
     \right. \label{equ-426-17}
     \end{align}

       \item $\alpha>1$时, 我们有
    \begin{align}\label{equ-426-18}
          I_2&= \sum_{|k|\ge L,|j|<|k|-|k|^\delta}|j|^{-\alpha}|k|^{-\beta}\langle |j-k|\rangle^{-\frac{1}{2}}\lesssim \sum_{|k|\ge L,|j|<|k|-|k|^{\delta}}|j|^{-\alpha}|k|^{-\beta-\frac{1}{2}\delta}\nonumber\\
          &\lesssim \sum_{|k|\ge L }|L|^{1-\alpha}|k|^{-\beta-\frac{1}{2}\delta}\lesssim\sum_{|k|\ge L}|k|^{-\frac{1}{2}\delta-\beta}.
    \end{align}
     \end{itemize}
结合式 \eqref{equ-426-14} -- 式 \eqref{equ-426-18}, 我们可得
\begin{align*}
I\lesssim
\left\{
\begin{array}{ll}
\sum_{|k|\geq L}\limits|k|^{\delta-\alpha-\beta}+ |k|^{1-\frac{1}{2}\delta-\alpha-\beta}, &  \quad 0<\alpha<1,\vspace{1ex}\\
\sum_{|k|\geq L}\limits|k|^{\delta-\alpha-\beta}+ \ln|k|\cdot|k|^{-\beta-\frac{1}{2}\delta}, & \quad \alpha=1,\vspace{1ex}\\
\sum_{|k|\geq L}\limits|k|^{\delta-\alpha-\beta}+|k|^{-\frac{1}{2}\delta-\beta}, & \quad \alpha>1.
\end{array}
\right.
\end{align*}
为了使得 $I<\infty$ 成立, 只需要说明存在 $\delta\in(0,1)$ 使得下述三组不等式中的一组成立即可:
\begin{align}
\delta-\alpha-\beta<-1, 1-\frac{1}{2}\delta-\alpha-\beta<-1,0<\alpha<1; \label{equ-426-19}\\
\delta-\alpha-\beta<-1,-\beta-\frac{1}{2}\delta<-1,\alpha=1; \label{equ-426-20}\\
\delta-\alpha-\beta<-1, -\frac{1}{2}\delta-\beta<-1, \alpha>1. \label{equ-426-21}
\end{align}
一方面, 易知当 $\alpha,\beta,\delta$ 满足
$$
0<\alpha\leq 1, \quad \alpha+\beta>1+\delta, \quad \alpha+\beta>2-\frac{1}{2}\delta
$$
时, 式 \eqref{equ-426-19} -- 式 \eqref{equ-426-20} 成立. 
通过选取 $\delta=\frac{2}{3}$, 可得当
\begin{align} \label{equ-426-22}
0<\alpha\leq 1, \quad \alpha+\beta>\frac{5}{3},
\end{align}时, $I<\infty$ 成立.
另一方面, 若存在 $\delta\in(0,1)$ 使得
$$
\alpha>1, \quad \alpha+\beta-1>\delta>2(1-\beta),
$$
则式 \eqref{equ-426-20} 成立.
进而在
$$
\alpha>1, \quad \alpha+\beta-1>2(1-\beta), \quad 1>2(1-\beta)
$$
条件下式 \eqref{equ-426-20} 成立.
 这说明当 $\alpha,\beta$满足条件
\begin{align} \label{equ-426-23}
\alpha>1, \quad \beta>\frac{1}{2}, \quad \alpha+3\beta>3
\end{align}
时, $I<\infty$成立.

{\bf 式 $II$ 的估计.} 类似的, 我们可以用同样的论证说明当 $\alpha,\beta$ 满足条件
\begin{align} \label{equ-426-24}
0<\beta\leq 1, \quad \alpha+\beta>\frac{5}{3}
\end{align}
或者条件
\begin{align} \label{equ-426-25}
\beta>1, \quad \alpha>\frac{1}{2}, \quad \beta+3\alpha>3
\end{align}
时, $II<\infty$ 成立.



 为了让式 $I$ 和 $II$ 都有限, $\alpha,\beta$ 只需要满足下述四种情形之一即可:
 \begin{equation*}
 \begin{array}{ll}
\eqref{equ-426-22}+\eqref{equ-426-24}
 \begin{cases}
  0< \alpha,\beta\le 1  \\
  \alpha+\beta>\frac{5}{3}
 \end{cases},
 &
 \eqref{equ-426-22}+\eqref{equ-426-25}
 \begin{cases}
  \frac{1}{2}<\alpha\leq 1\\
  \beta>1 \\
  \alpha+\beta>\frac{5}{3}\\
  3\alpha+\beta>3
  \end{cases},\\

 \eqref{equ-426-23}+\eqref{equ-426-24}
 \begin{cases}
  \alpha>1\\
  \frac{1}{2}<\beta\leq 1 \\
  \alpha+\beta>\frac{5}{3}\\
  \alpha+3\beta>3
 \end{cases},
 &
 \eqref{equ-426-23}+\eqref{equ-426-25}
 \begin{cases}
  \alpha>1 \\
  \beta>1
 \end{cases}.
 \end{array}
\end{equation*}
这四种情况的并等价于 $(\alpha,\beta)$ 满足条件 $({\bf H})$, 见图 \ref{fig1}. 这样我们就完成了证明 \end{proof}

\begin{figure}
    \begin{tikzpicture}[scale=1.8]
       % axes
       \draw[very thick,->] (-0.3,0) -- (3.5,0) node[right] {$\alpha$};
       \draw[very thick,->] (0,-0.3) -- (0,3.5) node[above] {$\beta$};
       % regions and boundaries
       \foreach \x in {1,...,3} \draw (\x, 0.05) -- (\x, -0.05) node[below] {\tiny \x};
       \foreach \y in {1,...,3} \draw (-0.05, \y) node[left] {\tiny \y} -- (0.05, \y);
       \fill[red!30] (1,1) -- (1,3.5) -- (3.5,3.5) -- (3.5,1) --cycle;
       \fill[yellow!30] (1/2,3/2) -- (2/3,1) -- (1,1) -- (1,3.5) --(1/2,3.5)--cycle;
       \fill[green!30] (3/2,1/2) -- (1,2/3) -- (1,1) -- (3.5,1) --(3.5,1/2)--cycle;
       \fill[blue!30] (2/3,1) -- (1, 2/3) -- (1,1) --cycle;
       \draw (-0.2,-0.01) node[below] {\tiny $O$};
       \draw[yellow!80, very thick] (1,1) -- (1,3.5);
       \draw[green!50, very thick] (1,1) -- (3.5,1);
       \draw[blue!50, very thick] (2/3,1)--(1,1)--(1,2/3);
       % labels
       \fill[blue!30] (4,3.25-0.625)--(4.25,3.25-0.625)--(4.25,3.5-0.625)--(4,3.5-0.625);
       \draw (4.25,3.375-0.625) node[right] {\tiny$0<\alpha,\beta\le 1,\alpha+\beta>\frac{5}{3}$};
       \fill[yellow!30] (4,2.25-0.625) -- (4.25,2.25-0.625) -- (4.25, 2.5-0.625) -- (4,2.5-0.625);
       \draw (4.25, 2.375-0.625) node[right] {\tiny$\frac{1}{2}<\alpha\le 1,\beta>1,\alpha+\beta>\frac{5}{3},3\alpha+\beta>3$};
       \fill[green!30] (4,2.75-0.625) -- (4.25, 2.75-0.625) -- (4.25,3-0.625)--(4,3-0.625);
       \draw (4.25,2.875-0.625) node[right] {\tiny$\alpha>1,\frac{1}{2}<\beta\le 1,\alpha+\beta>\frac{5}{3},\alpha+3\beta>3$};
       \fill[red!30] (4,1.75-0.625) -- (4.25, 1.75-0.625) -- (4.25,2-0.625) -- (4,2-0.625);
       \draw (4.25,1.875-0.625) node[right] {\tiny $\alpha>1,\beta>1$};
       % lines and equations
       \draw[dashed] (-1/5,18/5)node[below left]{\tiny $3\alpha+\beta=3$} --(1.1,-0.3);
       \draw[dashed] (-0.3,1.1)  --(18/5,-1/5);
       \draw (-1/5,1.1) node[below left]{\tiny $\alpha+3\beta=3$};
       \draw[dashed] (-1/5,28/15) node[below left]{\tiny $\alpha+\beta=\frac{5}{3}$}-- (28/15,-1/5);
    \end{tikzpicture}
    \caption{ 确保 $\int _{\R^2}K^2\,\d x \mathrm{d}y<\infty$ 成立的 $(\alpha,\beta)$ 范围.}
    \label{fig1}
    \end{figure}
    
     为了得到定理 \ref{thm-4}, 我们需要下述的唯一延拓性结果.
\begin{lemma}\label{lem-ucp}
    设 $u(t,x)\in C(\R, L^2(\R))$ 为线性KdV方程 \eqref{kdv-r} 的解. 若存在两点时刻 $t_1\neq t_2$ 以及常数 $c\in\R$ 使得
    \begin{equation}
        \mathrm{supp }\, u(t_j,\cdot)\subset (-\infty,c),\quad j=1,2
    \end{equation}
    或者
    \begin{equation}
        \mathrm{supp }\, u(t_j,\cdot)\subset (c,\infty),\quad j=1,2
    \end{equation}
    成立, 则
    $$
        u(t,x)\equiv 0,\quad -\infty <t,x<\infty.
    $$
\end{lemma}
\begin{proof}
 见 \cite[p.~60]{Zhang1992UniqueCF} 中定理 3.1 的证明.
\end{proof}

{\bf {定理 \ref{thm-2}} 的证明} 固定 $t>0$. 类似定理 \ref{thm-1}, 只需要证明 $$
\|T\|_{L^2(\R)\to L^2(\R)}\neq 1
$$
即可.

我们仍然用反证法, 假设 $\|T\|_{L^2(\R)\to L^2(\R)}= 1$. 利用引理 \ref{lem-comp-2}, 可知 $T$ 是从 $L^2(\R)$ 到 $L^2(\R)$ 上的紧算子. 因此, 存在函数 $f\in L^2(\R)$ 使得
\begin{align}\label{equ-51-5}
\|Tf\|_{L^2(\R)}=\|\chi_BS(t)\chi_Af\|_{L^2(\R)}=\|f\|_{L^2(\R)}=1.
\end{align}
有了式 \eqref{equ-51-5}, 便可由引理 \ref{lem-2} 得到
$$\mathrm{supp }\, f\subset A\quad \text{ 和 }\quad \mathrm{supp }\, S(t)f \subset B.$$
令 $u(\tau,\cdot)=S(\tau)f$. 则 $u(\tau,\cdot)$ 是方程
$$
\partial_\tau u+\partial_x^3u=0
$$
的解.
由定理 \ref{thm-2} 关于 $A$ 和 $B$ 的假设: 对于 $\tau=0,t$, $\mathrm{supp }\,u(\tau,\cdot)\subset (-\infty,c)$ 或 $(c,\infty)$ 成立. 从而可以利用引理 \ref{lem-ucp}, 得到 $u(\tau,\cdot)\equiv 0$. 特别地, 这说明 $f=0$. 但这与 $\|f\|_{L^2(\R)}=1$ (见 \eqref{equ-51-5}) 矛盾. 这样我们便完成了定理的证明. $\hfill\Box$
%

   \chapter{周期边界情形的可观测性}
    
     本章考虑周期情形的线性KdV方程
    \begin{align}\label{kdv-l}
        \left\lbrace \begin{array}{ll}\partial_t u +\partial_x^3u=0, & (t,x)\in \R\times (0,2\pi), \\
        \partial_x^ku(t,0)=\partial_x^k u(t,2\pi), & 0\le k\le 2,\\
        u(0,x)=u_0(x), & x\in (0,2\pi),
        \end{array}\right.
    \end{align}
       并给出方程 \eqref{kdv-l} (1) 关于一个移动观测点的能观测不等式 (2) 关于两个移动观测点的能观测不等式及判定方法.
    定义该情形下 Sobolev 空间
    \begin{equation*}
        H^3_p:=\lbrace v\in H^3 (0,2\pi):\partial_x^i v(0)=\partial_x^i v(2\pi), i=0,1,2 \rbrace.
    \end{equation*}
    对任意的初始函数 $u_0\in H_p^3$, 存在唯一的弱解
    \begin{equation*}
        u\in C(\R, H^3_p)\cap C^1(\R,L^2(0,2\pi)).
    \end{equation*}
    进而, $u$ 可以表示为
    \begin{equation}
        u(t,x)=\sum_{k\in\Z}c_k e^{i(k^3t+kx)},
    \end{equation}
    其中 $c_k$ 是 $u_0$ 的傅里叶系数:
    \begin{equation*}
        u_0(x)=\sum_{k\in \Z} c_ke^{ikx}.
    \end{equation*}
    一般情形下, KdV 方程都是实值函数, 所以还额外满足 $\overline{c_{-k}}=c_k, \forall k\in \Z$. 接下来的讨论对实值函数与复值函数都适用, 并在一些地方额外加以说明.
    \section{Ingham不等式与Eisenstein整数}
    设 $I$ 为一区间, 且 $|I|=2\pi$, 则我们可以将著名的Parseval公式写作
    \begin{equation*}
        \frac{1}{|I|} \int_I \left| \sum_{k\in \Z} c_k e^{ikx}\right|^2 \d x = \sum_{k\in \Z} |c_k|^2.
    \end{equation*}
    该公式对于 $|I|=2k\pi, k\in \N$ 也正确, 由此易知, 对于任意满足 $2k\pi <|I|<(2k+2)\pi,k\in\N$ 的区间 $I$, 都有
    \begin{equation*}
        2k\pi \sum_{k\in \Z}|c_k|^2 \le \int _I \left| \sum_{k\in \Z}c_ke^{ikx} \right|^2\d x \le (2k+2)\pi  \sum_{k\in\Z }|c_k|^2. 
    \end{equation*}
    自然地, 给定一个区间 $I$, 以及一个实数集 $\Lambda$, 是否存在常数 $c,C>0$, 使得对任意的平方可和序列 $\left\{c_k\right\}_{k\in \Z}$都有关系式
    \begin{equation}
        c \sum_{k\in \Lambda} |c_k|^2 \le \int _I \left| \sum_{k\in \Lambda} c_k e^{ikx} \right|^2 \d x \le C \sum_{k\in \Lambda}|c_k|^2
    \end{equation}
    或其等价形式
    \begin{equation}\label{3-1-2}
        \int _I \left| \sum_{k\in \Lambda} c_k e^{ikx} \right|^2 \d x\asymp \sum_{k\in \Lambda}|c_k|^2,
    \end{equation}
    成立呢? 若成立, 则上式可看作Parseval公式在区间 $I$以及指数集合$\Lambda$ 上的推广. 
    
    为了建立上述Parseval公式的推广, Ingham \cite{Ingham1936} 证明了下述定理
    \begin{theorem}\label{thm-ingham}
    设 $\Lambda$ 为一族实数集, 且该实数族满足条件
    \begin{equation}\label{3-1-1}
        \gamma(\Lambda)=\inf\lbrace |\lambda_1-\lambda_2|: \lambda_1\neq \lambda_2, \lambda_1,\lambda_2\in \Lambda\rbrace>0,
    \end{equation}
     则对任意满足 $|I|>2\pi /\gamma$ 的区间 $I$都有关系式 \eqref{3-1-2} 成立. 实际上, 关系式的常数可以具体给出并写作
     \begin{equation}
         \frac{2\alpha(4\pi+\alpha)}{\pi(2\pi+\alpha)^2}|I|\sum_{k\in\Lambda}|c_k|^2\le \int_{I}\left|\sum_{k\in \Lambda}c_ke^{i\lambda_kx}\right|^2\d x\le \frac{20 |I|}{\min\left\{\pi,|I|\gamma(\Lambda)\right\}}\sum_{k\in\Lambda}|c_k|^2,
     \end{equation}
     其中 $\alpha=|I|\gamma_0-2\pi$.
     
      我们称满足条件 \eqref{3-1-1} 的实数集为一致分离集. 
    \end{theorem}
    
    条件 $|I|> 2\pi /\gamma$ 是对所有满足 \eqref{3-1-1} 实数族的一致最佳条件, 但在某些特殊情形下, 这一条件可以被改进. Haraux \cite{Haraux1989} 给出了下述定理.
    
    \begin{theorem}\label{thm-haraux}
    设 $\Lambda$ 为一致分离集, 且 $F\subset \Lambda$ 为有限子集, 则关系式 \eqref{3-1-2} 在 $|I|> \frac{2\pi}{\gamma(\Lambda \backslash F)}$ 的条件下也成立.
    \end{theorem}
    

    
   
    本章第三节对多条线段的可观测性研究中, 涉及到集合
    \begin{equation}\label{Gamma-set}
        \Gamma:=\lbrace n:\exists m,k\in \Z,m\neq k \text{ s.t. } n=m^2+mk+k^2 \rbrace
    \end{equation}
    中元素的性质. 而该集合在数论中与 Eisenstein 整数相关, 本节旨在介绍 Eisenstein 整数及其与文章内容相关的性质. 对下述提到的环, 数域, 整环, 唯一分解整环, 欧几里得整环等基本概念, 参见. 
    
\begin{definition}
    一个数 $\alpha \in \C$  称为代数数, 若其是一个多项式
    \begin{equation*}
        f(x)=a_nx^n+a_{n-1}x^{n-1}+\cdots+a_1x+a_0
    \end{equation*}
    的跟, 其中 $a_0,a_1,\cdots,a_{n-1},a_n\in \Q, a_n\neq 0$. 
    若, $\alpha$ 不是代数数, 则称其为超越数.
    
    特别地, 若存在多项式系数均为整数且 $a_n=1$, 使得 $\alpha$ 为它的根, 则称 $\alpha$ 为代数整数. 
\end{definition}

\begin{definition}
    若数域 $F$ 包含一个子数域 $k$, 则称 $F$ 是数域 $k$ 的域扩张, 记作 $F /k$. 若 $F$ 是 $k$ 的扩张, 则可以将$F$ 看作域 $k$ 上的线性空间, 该线性空间的维数被称为域扩张的度数并记作
    \begin{equation*}
        [F:k]=\text{线性空间 } F \text{ 在 } k \text{ 上的维数}.
    \end{equation*}
\end{definition}

\begin{definition}
    数域 $\C$ 中的子数域, 称为代数数域, 若其是包含代数数 $\alpha_1,\alpha_2,\cdots,\alpha_n$ 和 $\Q$ 的最小数域, 并记作 $\Q(\alpha_1,\alpha_2,\cdots,\alpha_n)$.
\end{definition}

代数整数推广了整数的概念, 例如有理数域 $\Q$ 中所有的代数整数是 $\Z$, 高斯数域 $\Q(i)$ 中所有代数整数是 $\Z[i]$.

\begin{theorem}
若 $K$ 是一个代数数域, 则存在一个代数数 $\theta$ 使得 $K=\Q(\theta) $.
\end{theorem}

\begin{definition}
    令 $\omega =e^{\frac{2\pi i}{3}}$, 则易知 $\Q (\omega)$ 中的所有代数整数构成的整环为 $Z[\omega]$. 整环 $Z[\omega]$ 称为 Eisenstein 整环, 其中的元素称为 Eisenstein 整数. 记 $\Z [\omega]$ 中的所有可逆元为 $U(\Z[\omega])$, 则
    \begin{equation*}
        U(\Z[\omega])=\lbrace\pm 1,\pm \omega, \pm \omega ^2\rbrace.
    \end{equation*}
\end{definition}
实际上, 我们有 $\Q(\omega)=\Q(\sqrt{-3})$ 以及 $\Z[\omega]=\lbrace a+b\omega:a,b\in \Z \rbrace.$
Eisenstein 整环是欧几里得整环, Eisenstein 素数之于 $\Z[\omega]$ 的作用, 与素数之于 $\Z$ 的作用基本相同. 这意味着对其有素因子概念以及唯一分解定理. 
\begin{definition}
    整环 $\Z[\omega]$ 中的不可约元素称为 Eisenstein 素数.
\end{definition}
\begin{theorem}
    对任意的 $a,b\in \Z[\omega]$, 若存在 $u\in U(\Z[\omega])$ 使得 $a=bu$, 则记 $a\sim b$. 对 $\Z [\omega]$ 中任意的元素 $m$, 都有素分解
    \begin{equation*}
        m=\prod_{i=1}^r p_i^{r_i}, 
    \end{equation*}
    其中 $p_i,i=1,2,\cdots, r$ 为 Eisenstein 素数, 且该分解在等价关系 $\sim$ 和不考虑顺序的情况下是唯一的.
\end{theorem}

\begin{theorem}\label{thm-eisenstein}
    设 $z=m+in\in \Z[\omega]$, 则 $z$ 为 Eisenstein 素数当且仅当存在素数 $p$ 使得其满足下述条件之一
    \begin{enumerate}
        \item[\rm{(1)}] $z\sim p, p\equiv 2\pmod{3}$.
        \item[\rm{(2)}] $|z|^2=m^2-mn+n^2=p, p\equiv 0 \text{ 或 } 1\pmod{3}$.
    \end{enumerate}
\end{theorem}

\begin{corollary}
    设 $p\in \Z$ 为素数, 则其可写作
    \begin{equation*}
        p=m^2+mn+n^2,\quad m,n\in \Z
    \end{equation*}
    当且仅当 $p\equiv 1\pmod{3}$ 或 $p=3$. 特别地, 当 $p\equiv 1 \pmod{3}$ 时, $m\neq n$, $p=3$ 时, $(a,b)$ 可取
    \begin{equation*}
        (1,1),(1,-2),(-1,-1),(-1,2).
    \end{equation*}
\end{corollary}
    
    \begin{corollary}\label{crc-3-2-2}
        设 $a\in \Z$ 为整数, 则其可写作
        \begin{equation*}
            a=m^2+mn+n^2,\quad m,n\in \Z
        \end{equation*}
    当且仅当 $a$ 满足下述的素数分解
    \begin{equation}\label{factorization}
        a=3^r \prod_{i=1}^s p_i^{s_i} \prod_{j=1}^t q_j^{t_j},
    \end{equation}
    其中 $p_i\equiv 1\pmod{3}, q_j\equiv 2\pmod{3}, 2|t_j,i=1,\cdots,s,j=1,\cdots,t$. 实际上, $a\in \Gamma$ 当且仅当 $a$ 满足上述分解.
    \end{corollary}
    
    \begin{proposition}
    对任意的实数 $a$, 定义
    \begin{equation*}
        \Pi_a:=\left\lbrace\begin{array}{ll}
           \lbrace k\in \Z:\exists m \neq k \text{ s.t. } a=m^2+mk+k^2\rbrace,  & a\in \Gamma  \\
            \emptyset, & a\notin \Gamma.
        \end{array}\right.
    \end{equation*}
    则易知 $|\Pi_a|<\infty$.
    \end{proposition}

\section{单个移动观测点的能观测不等式}

下述定理的证明思路源于 \cite{jaming2020moving} 中对薛定谔方程关于移动观测点能观测性的证明.
    \begin{theorem}\label{thm-3-3-1}
    设 $(t_0,x_0)\in \R\times \T$, $a\in \R$. 则方程 \eqref{kdv-l} 的解满足:
    \begin{enumerate}
        \item[\rm{(1)}] 对任意的 $T>0$, 存在常数 $C_1>0$ 使得不等式
        \begin{equation}\label{3-2-1}
            \int_0^T\left| u(t_0+t,x_0-at) \right|^2\d t \le C_1  \sum_{k\in \Z} |c_k|^2
        \end{equation}
        成立.
        \item[\rm{(2)}] 对任意的 $T>0$ 以及 $a\notin \Gamma$, 存在常数 $C_2>0$ 使得不等式
        \begin{equation}\label{3-2-2}
            \sum_{k\in\Z} |c_k|^2 \le C_2 \int_0^T |u(t_0+t,x_0-at)|^2 \d t
        \end{equation}
        成立.
        \item[\rm{(3)}] 若 $a\in \Gamma$, 则 $a>0$ 且
        \begin{equation}\label{3-2-3}
            \int_0^T |u(t_0+t,x_0-at)|^2\d t\asymp \sum_{k\in\Z}^\infty|d_k|^2,
        \end{equation}
        其中
        \begin{equation*}
            d_k=\sum_{\substack{
                m\in \Z,  \\
                k^2+mk+m^2 = a }} 
             c_me^{i(m^3t_0+mx_0)},\quad k\in \Z.
        \end{equation*}
        实际上, 当 $|k|> 2\left(\frac{a}{3}\right)^{1 /2}$ 时, 有 $d_k=c_k e^{i (k^3t_0+kx_0)}$.
        
        此时存在 $u\neq 0$ 使得
        \begin{equation*}
            u(t_0+t,x_0-at)=0, \quad\forall t\in \R.
        \end{equation*}
    \end{enumerate}
    \end{theorem}
    
    \begin{proof}
     (1) 固定 $a\in\R$, 简单计算可得
     \begin{equation}\label{3-2-4}
         u(t_0+t,x_0-at)=\sum_{k\in \Z} c_k e^{i(k^3(t_0+t)+k(x_0-at))}=\sum_{k\in \Z}c_k  e^{i (k^3t_0+kx_0)} e^{i(k^3-ak)t}.
     \end{equation}
    令 $\Lambda=\lbrace k^3-ak: k\in \Z\rbrace$, 若 $a<0$, 则有
    \begin{equation*}
        ((k+1)^3-a(k+1))-(k^3-ak)=3k^2+3k+1-a>1,\quad k\in \Z.
    \end{equation*}
    因此 $\Lambda$ 是一致分离集, 故由定理 \ref{thm-ingham} 知不等式 \eqref{3-2-1} 成立. 若 $a>0$, 设 $\sigma=\left(\frac{a}{3}\right)^{1 /2}$, 集合 $\Lambda$ 可分为下述三个集合的并:
    \begin{equation*}
        \Lambda_-=\lbrace k\in \Lambda : k<-\sigma\rbrace, \Lambda_0=\lbrace k\in \Lambda : -\sigma\le k\le \sigma\rbrace,\Lambda_+=\lbrace k\in\Lambda: k>\sigma\rbrace.
    \end{equation*}
    易知三个集合关于 $k$ 都是单调的, 所以三个集合都是一致分离集, 它们分别都满足不等式 \eqref{3-2-1}, 对着三个不等式求和, 再由不等式
    \begin{equation*}
        (x_1+x_2+\cdots+x_m)^2\le m (x_1^2+x_2^2+\cdots+x_m^2)
    \end{equation*}
    可得集合 $\Lambda$ 满足不等式 \eqref{3-2-1}.
    
    (2) 设 $a\notin \Gamma$, 则集合 $\Lambda$ 是一致分离的. 事实上, 对于两个不同的整数 $k$ 和 $m$, 我们有
    \begin{equation*}
        |(k^3-ak)-(m^3-am)|=|k-m||k^2+mk+m-a|\ge d(a,\Z):=\max(a-\lfloor a\rfloor,\lfloor a\rfloor +1-a). 
    \end{equation*}
    
    若 $a\le 0$, 取正整数 $N$, $k\neq m$ 且 $k,m\notin\lbrace-N,\cdots, N\rbrace$. 则利用等式
    \begin{equation*}
        (k^3-ak)-(m^3-am)=(k-m)(k^2+mk+m^2-a)
    \end{equation*}
    及简单的计算可得
    \begin{equation}\label{3-22-3}
        \left| (k^3-ak)-(m^3-am) \right|\ge \left\lbrace\begin{array}{ll}
            3N^2-a & k,m> N \text{ 或 } k,m <-N, \\
             2N^3-2aN  & km<0.
        \end{array}\right.
    \end{equation}
    所以
    \begin{equation*}
        \gamma\left(\Lambda \backslash \lbrace-N,\cdots,N\rbrace\right)\ge N^2.
    \end{equation*}
    令 $N\to \infty$, 则由定理 \ref{thm-haraux} 可得对任意的 $T>0$ 都有不等式 \eqref{3-2-2} 成立.
    
    若 $a>0$, 取 $N>\max\lbrace 2\sigma, a\rbrace$, $k\neq m, k,m\notin \lbrace-N,\cdots, N\rbrace$, 类似地我们也有估计 \eqref{3-22-3}. 同样令 $N\to \infty$ 可得对任意 $T>0$ 都有不等式 \eqref{3-2-2} 成立. 
    
    (3) 若 $a\in \Z$, 易知 $a>0$. 序列 $\lbrace k^3-ak\rbrace$ 在 $k<-\sigma$ 和 $k>\sigma$ 时单调递增, 在 $-\sigma\le k\le \sigma$ 时单调递减. 令
    \begin{equation*}
        \sigma^3-a\sigma=t^3-at,
    \end{equation*}
    解得较大的根 $t=2\left(\frac{a}{3}\right)^{1 /2}$,
    所以当 $|k|> 2\left(\frac{a}{3}\right)^{1 /2}$ 时, $k=m^3-am$ 的解 $m$ 只有一个. 这说明此时有 $d_k=c_k e^{i(k^3t_0+kx_0)}$. 
    则式 \eqref{3-2-4}可写作
    \begin{equation*}
        u(t_0+t,x_0-at)=\sum_{l\in \lbrace j^3-aj:j\in \Z  \rbrace}e_l e^{ilt}=\sum_{k\notin \Pi_a} d_k e^{i(k^3-ak)t}+\sum_{l\in \lbrace j^3-aj: j\in \Pi_a, j\in \Z\rbrace} e_l e^{ilt},
    \end{equation*}  
    其中
    \begin{equation*}
        e_l=d_k, \quad \forall k \text{ s.t. } k^3-ak=l.
    \end{equation*}
    因为 $\lbrace l: l=k^3-ak \rbrace \subset \Z $ 为一致分离集, 则由类似 (2) 的方法可得对任意的 $T$, 有关系式
    \begin{equation}\label{3-2-5}
        \int_0^T|u(t_0+t,x_0-t)|^2\d t\asymp \sum_{k\in \Pi_a}|d_k|^2+\sum_{l\in \lbrace j^3-aj: j \notin \Pi_a,j\in\Z\rbrace }|e_l|^2.
    \end{equation}
    由 $\lbrace k^3-ak:k\in \Z\rbrace$ 关于 $k$ 的单调性可知, 对任意一个 $l\in \lbrace k^3-ak; k\in \Z\rbrace$, 至多有 3 个 不同的 $k$, 且至少有一个 $k$  满足 $l=k^3-ak$. 这说明 对某个 $l$, $\sum_{k\in\Z} |d_k|^2$ 中 $d_k=e_l$ 的项至多有 3 个, 至少有一个, 则有关系式
    \begin{equation}\label{3-2-6}
        \sum_{k\in \Z}|d_k|^2\asymp  \sum_{k\in \Pi_a}|d_k|^2+\sum_{l\in \lbrace j^3-aj: j \notin \Pi_a,j\in\Z\rbrace }|e_l|^2.
    \end{equation}
    由式 \eqref{3-2-5} 和式 \eqref{3-2-6} 可得关系式 \eqref{3-2-3}.
    
    取固定的 $k\in \Pi_a$,  则存在 $m\neq k$ 且由 $a=m^2+mk+k^2$ 可知 $k^3-ak=m^3-am$. 此时只要取
        \begin{equation*}
            u_0(x)=e^{-i(k^3t_0+kx_0)}e^{ikx}-e^{-i(m^3t_0+mx_0)}e^{imx}.
        \end{equation*}
此时 $u_0\neq 0$ 且由 $u(t,x)=\sum_{k\in\Z}c_k e^{i(k^3 t+kx)}$ 可得
        \begin{align*}
            u(t_0+t,x_0-at) &= e^{-i(k^3t_0+kx_0)}e^{ik^3(t_0+t)}e^{ik(x_0-at)}\\
            &-e^{-i(m^3t_0+mx_0)}e^{im^3(t_0+t)}e^{im(x_0-at)}\\
            &=0.
        \end{align*}

    若要求解函数为实值函数, 则分下述两种情况:
    \begin{enumerate}
    \item [(i)] 若$a\neq k^2$, 设 $m\neq k$ 且 $k^3-ak=m^3-am$, 则易知 $m\neq -k$, 令
    \begin{align*}
        u_0(x)&= e^{-i(k^3t_0+kx_0)} e^{ikx} + e^{i(k^3t_0+kx_0)} e^{-ikx}\\
        &-e^{-i(m^3t_0+mx_0)}e^{imx}-e^{i(m^3t_0+mx_0)}e^{-imx},
    \end{align*}
    则 $u\neq 0$ 且
    \begin{equation*}
        u(t_0+t,x_0-at)=0,\quad\forall t\in \R.
    \end{equation*}
     \item [(ii)] 若$a=k^2$, 令
    \begin{equation*}
        u_0(x)=e^{-i(k^3t_0+kx_0)}e^{ikx} + e^{i(k^3t_0+kx_0)}e^{-ikx}-2,
    \end{equation*}
    则 $u\neq 0$ 且
    \begin{equation*}
        u(t_0+t,x_0-at)=0,\quad\forall t\in\R.
    \end{equation*}
    \end{enumerate}
    \end{proof}
    
    在上述的定理中, 我们证明了当 $a\notin \Gamma$ 时, 对所有方程 \eqref{kdv-l} 的解 $u(t,x)$ 有
    \begin{equation*}
        \int_{\T}|u_0|^2\d x\asymp \int_{0}^T|u(t_0+t,x_0-at)|^2\d t.
    \end{equation*}
    对于空间 $ H_p^s$ 中的情形, 我们有下述引理:
    \begin{lemma}
    设 $(t_0,x_0)\in \R\times \T, a\in \R,$ $u(t,x)$ 为方程 \eqref{kdv-l} 的解, 且 $u(0,x)=u_0(x)\in H_p^s$. 则我们有
    \begin{equation}
        \|u_0\|_{s}\asymp \|h\|_{\frac{s}{3}},
    \end{equation}
    其中 $h(t):=u(t_0+t,x_0-at)$, 且 $\|h\|_{\frac{s}{3}}:=\left( \sum_{k\in \Z}\langle k^3-ak\rangle^{2s/3}|c_k|^2 \right)^{\frac{1}{2}}$.
    \end{lemma}
    
    \section{多个移动观测点的能观测不等式}
    首先我们讨论两个移动观测点的情形, 设两个移动点的观测速度分别为 $a_1,a_2$.
    定义
    \begin{equation*}
        \begin{array}{ll}
                        d_{1,k}=\sum\limits_{\substack{
                m\in \Z,  \\
                k^2+mk+m^2 = a_1 }} 
             c_me^{i(m^3t_1+mx_1)}, & k\in \Z, \\
                        d_{2,k}=\sum\limits_{\substack{
                m\in \Z,  \\
                k^2+mk+m^2 = a_2 }} 
             c_me^{i(m^3t_2+mx_2)}, & k\in \Z. 
        \end{array}
    \end{equation*}
    由定理 \ref{thm-3-3-1} 可得
\begin{equation}\label{3-2-7}
    \int_0^T |u(t_1+t,x_1-a_1t)|^2\d t\asymp \sum_{k\in\Z}^\infty|d_{1,k}|^2,
\end{equation}
以及
\begin{equation}\label{3-2-8}
    \int_0^T |u(t_2+t,x_2-a_2t)|^2\d t\asymp \sum_{k\in\Z}^\infty|d_{2,k}|^2.
\end{equation}
式 \eqref{3-2-7} 和 式 \eqref{3-2-8}相加可得
\begin{equation}\label{3-2-9}
  \sum_{k\in\Z}|d_{1,k}|^2+\sum_{k\in\Z}|d_{2,k}|^2\asymp \int_0^T |u(t_1+t,x_1-a_1t)|^2 + |u(t_2+t,x_2-a_2t)|^2\d t.
\end{equation}
对于两个移动观测点, 以及任意的 $T>0$ 和初始位置 $(t_1,x_1),(t_2,x_2)$, 本节尝试建立关系式
\begin{equation*}
    \sum_{k\in \Z} |c_k|^2 \asymp \int_0^T |u(t_1+t,x_1-a_1t)|^2 + |u(t_2+t,x_2-a_2t)|^2\d t.
\end{equation*}
而由定理 \ref{thm-3-3-1} 的 (1) 可知只需要建立式
\begin{equation} 
    \sum\limits_{k\in \Z} |c_k|^2\lesssim\int_0^T |u(t_1+t,x_1-a_1t)|^2 + |u(t_2+t,x_2-a_2t)|^2\d t.
\end{equation}

\begin{lemma}\label{lma-3-3-1}
设 $a_1,a_2$ 为两个任意实数, 若 $\Pi_{a_1}\cap \Pi_{a_2}=\emptyset$, 则不等式
\begin{equation} 
    \sum\limits_{k\in \Z} |c_k|^2\lesssim\int_0^T |u(t_1+t,x_1-a_1t)|^2 + |u(t_2+t,x_2-a_2t)|^2\d t.
\end{equation}
成立.
\end{lemma}
\begin{proof}
 若 $a_1\notin \Gamma$, 则由定理 \ref{thm-3-3-1} 的 (2) 可知不等式成立. 同理 $a_2\notin \Gamma$ 时, 不等式也成立. 

进一步地, 若 $a_1,a_2\in \Gamma $ 且 $\Pi_{a_1}\cap \Pi_{a_2}=\emptyset$, 则对任意的 $k\in \Pi_{a_1}$, 必有 $k\notin \Pi_{a_2}$, 所以 $d_{2,k}=c_k e^{i(kx_0+k^3t_0)}$, 即 $|d_{2,k}|=|c_k|$. 同理对任意的 $k\in \Pi_{a_2}$, 必有 $|d_{1,k}|=|c_k|$. 这说明该情形下有
\begin{equation*}
    \sum_{k\in \Z }|c_k|^2\lesssim \sum_{k\in \Z }|d_{1,k}|^2+\sum_{k\in\Z}|d_{2,k}|^2
\end{equation*}
成立, 上式和式 \eqref{3-2-9} 可得所需不等式.
\end{proof}

由引理 \ref{lma-3-3-1} 可知, 我们只需要考虑 $a_1,a_2\in \Gamma$ 的情形. 为了后续定理叙述方便以及直观表述, 我们将 $V:=\Pi_{a_1}\cap \Pi_{a_2}$ 中的数视作节点, 并将两个节点按照下述规则相连:
\begin{itemize}
    \item 若 $V$ 中存在两个不同的节点 $k,k'\in V, k\neq k'$ 使得 $k^3-a_1k=k'^3-a_1k'$ (或等价地 $k^2+kk'+k'^2=a_1$), 则在两个节点之间连上红色的边.
    \item 若 $V$ 中存在两个不同的节点 $k,k'\in V, k\neq k'$ 使得 $k^3-a_1k=k'^3-a_2k'$ (或等价地 $k^2+kk'+k'^2=a_2$), 则在两个节点之间连上红色的边.
\end{itemize}
则所有的边和节点 $V$ 构成了一个边二染色图, 记作 $G(a_1,a_2)$.

由定义易知图 $G(a_1,a_2)$ 的任意两个节点 $k$ 与 $k'$ 之间 只有一条颜色的边相连, 这是因为
$k^2+kk'+k'^2=a_1$ 与 $k^2+kk'+k'^2=a_2$ 不可能同时成立.
\begin{example}
设 $\Pi_{a_1}\cap \Pi_{a_2}=\emptyset$, 则 $G(a_1,a_2)=\emptyset$ 为空图. 
\end{example}
\begin{example}\label{exa-important}
设 $a_1=7,a_2=13$, 则 $G(7,13)$ 见图 \ref{fig:G-7-13}.
\begin{figure}[htbp]
    \centering
     \begin{tikzpicture}
     \draw (0,0)node[circle, fill, inner sep=1.5pt]{};
     \draw (0,0)node[below left] {$-3$};
     \draw (0,2)node[circle, fill, inner sep=1.5pt]{};
     \draw (0,2)node[above left] {$1$};
     \draw (2,0)node[circle, fill, inner sep=1.5pt]{};
     \draw (2,0)node[below right] {$-1$};
     \draw (2,2)node[circle, fill, inner sep=1.5pt]{};
     \draw (2,2)node[above right] {$3$};
     \draw[red] (0,0)--(0,2);
     \draw[red] (2,0)--(2,2);
     \draw [blue] (0,0)--(2,0);
     \draw[blue] (0,2)--(2,2);
     \end{tikzpicture}
    \caption{边二染色图 $G(7,13)$}
    \label{fig:G-7-13}
\end{figure}
\end{example}

在图论中, 若我们从图中的某一点沿着边走下去能回到出发点, 则称走过的边和点构成的路径为闭路径. 为了叙述方便, 我们引入交错闭路径的概念.
\begin{definition}
    若图 $G(a_1,a_2)$  中存在红蓝交错排布的边构成闭路径, 则称该回路为图 $G(a_1,a_2)$ 的交错闭路径, 记作 $O(k_1,k_2,\cdots,k_{2n})$, 若 $k_1$ 与 $k_2$ 间连红色, 如图 \ref{fig:cross-circle} 所示.
    \begin{figure}[htbp]
        \centering
        \begin{tikzpicture}
        \draw (0,0)node[circle, fill, inner sep=1.5pt]{};
        \draw (1,2)node[circle, fill, inner sep=1.5pt]{};
        \draw (3,2)node[circle, fill, inner sep=1.5pt]{};
        \draw (4,0)node[circle, fill, inner sep=1.5pt]{};
        \draw (3,-2)node[circle, fill, inner sep=1.5pt]{};
        \draw (1,-2)node[circle, fill, inner sep=1.5pt]{};
        \draw[red] (0,0)--(1,2);
        \draw[blue] (1,2)--(3,2);
        \draw[red] (3,2)--(4,0);
        \draw[loosely dotted] (4,0)--(3,-2);
        \draw[red] (3,-2)--(1,-2);
        \draw[blue] (1,-2)--(0,0);
        \draw (0,0)node[left] {$k_1$};
        \draw (1,2)node[above left] {$k_2$};
        \draw (3,2)node[above right] {$k_3$};
        \draw (4,0)node[right] {$k_4$};
        \draw (3,-2)node[below right] {$k_{2n-1}$};
        \draw (1,-2)node[below left] {$k_{2n}$};
        \end{tikzpicture}
        \caption{交错路径 $O(k_1,k_2,\cdots,k_{2n})$}
        \label{fig:cross-circle}
    \end{figure}
    若交错闭路径是简单闭路径 (不存在 $k_i=k_j,i\neq j$), 则称其为简单交错闭路径.
\end{definition}

在例 \ref{exa-important} 中, $G(7,13)$ 有交错闭路径, 且该交错路径为简单交错闭路径.
\begin{lemma}\label{lma-3-3-2}
对任意的图 $G(a_1,a_2)$, 若存在三个不同的节点 $k,k',k''$ 使得其中有两条同色边, 则三点构成一个同色三角形. 特别地, 两两相连的边构成三角形必同色. 
\end{lemma}
\begin{proof}
不妨设 $k,k'$间的边与 $k,k''$ 间的边均为红色, 则由定义可知
\begin{align*}
    & k^2+kk'+k'^2=a_1,\quad  k^2+kk''+k''^2=a_1.
\end{align*}
由三个数各不相同, 第一个等式两边乘以 $k-k'$, 第二个等式两边乘以 $k-k''$, 则上述两个等式等价于
\begin{align*}
    k^3-k'^3=a_1(k-k'),\quad k^3-k''^3=a_1(k-k'').
\end{align*}
两个等式相减可得
\begin{equation*}
    k'^3-k''^3=a_1(k'-k'')\Longrightarrow k'^2+k'k''+k''^2=a_1. 
\end{equation*}
所以 $k',k''$ 间也有红色边相连.
\end{proof}

\begin{lemma} \label{lma-3-3-3}
对任意的图 $G(a_1,a_2)$, 下述三个命题等价:
\begin{enumerate}
    \item [\rm{(1)}]  $G(a_1,a_2)$ 有交错路径.
    \item [\rm{(2)}]  $G(a_1,a_2)$ 有简单交错路径.
    \item [\rm{(3)}]  $G(a_1,a_2)$ 有双色简单闭路径.
\end{enumerate}
\end{lemma}
\begin{proof}
由定义以及引理 \ref{lma-3-3-2} 可知 (3) $\Rightarrow$ (1) 与 (2) $\Rightarrow$ (3) 是显然的, 所以只要说明
(1) $\Rightarrow$ (2) 即可. 对于交错闭路径 $O(k_1,k_2,\cdots,k_{2n})$, 不妨设其第一个交错点为 $k_{i}, 1<i<2n$ 且 $i$ 为偶数. 由 $k_i$ 为交错点知存在 $j > i$ 使得 $k_j=k_i$. 实际上 $j>i+1$, 因为若 $j=i+1$, $k_i$ 与 $k_{i+1}$ 为同一节点, 矛盾. 则 $k_i$ 及其相邻元素有两种情况:
\begin{itemize}
    \item[(i)] $k_{j-1}$ 与 $k_j$ 连线为蓝色, 如图 \ref{fig:cross-graph-2-a}.
\begin{figure}[htbp]
    \centering
    \begin{subfigure}[b]{0.3\textwidth}
        \begin{tikzpicture}
    \draw (0,0) node[circle, fill, inner sep=1.5pt]{};
    \draw (0,0.2)node[left] {$k_i=k_j$};
    \draw (0,2) node[circle, fill, inner sep=1.5pt]{};
    \draw (0,2)node[left]{$k_{i-1}$};
    \draw (-2,-1) node[circle, fill, inner sep=1.5pt]{};
    \draw (-2,-1) node[left]{$k_{j-1}$};
    \draw (2,-1) node[circle, fill, inner sep=1.5pt]{};
    \draw (2,-1) node[right] {$k_{j+1}$};
    \draw[red] (0,0)--(0,2);
    \draw [blue] (-2,-1)--(0,0);
    \draw [red] (0,0)--(2,-1);
    \end{tikzpicture}
    \caption{节点$k_{j-1},k_j$ 间为蓝色边}
    \label{fig:cross-graph-2-a}
    \end{subfigure}
    \hspace{0.1\textwidth}
    \begin{subfigure}[b]{0.3\textwidth}
            \begin{tikzpicture}
    \draw (0,0) node[circle, fill, inner sep=1.5pt]{};
    \draw (0,0.2)node[left] {$k_i=k_j$};
    \draw (0,2) node[circle, fill, inner sep=1.5pt]{};
    \draw (0,2)node[left]{$k_{i-1}$};
    \draw (-2,-1) node[circle, fill, inner sep=1.5pt]{};
    \draw (-2,-1) node[left]{$k_{j-1}$};
    \draw (2,-1) node[circle, fill, inner sep=1.5pt]{};
    \draw (2,-1) node[right] {$k_{j+1}$};
    \draw[red] (0,0)--(0,2);
    \draw [red] (-2,-1)--(0,0);
    \draw [blue] (0,0)--(2,-1);
    \end{tikzpicture}
    \caption{ 节点$k_{j-1},k_j$ 间为红色边}
    \label{fig:cross-graph-2-b}
    \end{subfigure}
    \caption{节点 $k_i$ 及其相邻节点间的关系}
    \label{fig:cross-graph-2}
\end{figure}
则由引理 \ref{lma-3-3-2} 得 $k_{i-1}$ 与 $k_{j+1}$ 之间有红色边, 由此可得新的交错闭路径 $O(k_1,k_2,\cdots,k_{i-1} ,k_{j+1},\cdots,k_{2n})$.
\item [(ii)] $k_{j-1}$ 与 $k_j$ 连线为蓝色, 如图 \ref{fig:cross-graph-2-b}. 则由 $k_{i}$ 直接到 $k_{j+1}$ 也可以得到新的交错闭路径 $O(k_1,k_2,\cdots, k_{i-1},k_{j+1},\cdots ,k_{2n})$. 
\end{itemize}
在以上两种情形我们构造新的交错路径后, 自交点严格变少, 路径中包含的节点也严格变少, 持续上述步骤经有限步后, 必然不会再有自交点, 即得简单交错闭路径.
\end{proof}


\begin{theorem}\label{abstract-thm}
给定任意的 $T>0$, 考虑方程 \eqref{kdv-r} 的解. 设 $a_1,a_2$ 为两个不同的实数, $(t_1,x_1),(t_2,x_2)\in \R\times \T$.   则下述命题等价:
\begin{enumerate}
    \item[\rm{(1)}] 不等式 
    \begin{equation}\label{obs-two-points} 
    \sum\limits_{k\in \Z} |c_k|^2\lesssim\int_0^T |u(t_1+t,x_1-a_1t)|^2 + |u(t_2+t,x_2-a_2t)|^2\d t
\end{equation}
成立.
    \item [\rm{(2)}] 由
\begin{equation*}
    u(t_1+t,x_1-a_1t)=u(t_2+t,x_2-a_2t)=0, \forall t\in (0,T)
\end{equation*}
可推出 $u$ 是平凡解, 即 $u_0=0$.
   \item[\rm{(3)}]  $G(a_1,a_2)$ 中不存在长度大于 $3$ 的简单闭路径.
\end{enumerate}
\end{theorem}
\begin{proof}
\iffalse  不失一般性, 设 $G(a_1,a_2)$ 连通. 对 $G(a_1,a_2)$ 中的任意一个节点, 只有如图 \ref{fig:cross-graph-3} 所示的 $4$ 种情况. 
\begin{figure}[htbp]
    \centering
    \begin{tikzpicture}
        \draw (0,0) node[circle, fill, inner sep=1.5pt]{};
        \draw[red] (-1,0)--(1,0);
        \draw[blue] (0,-1)--(0,1);
        \draw (3,0) node[circle, fill, inner sep=1.5pt]{};
        \draw[blue] (3,0)--(3-0.5,1);
        \draw[red] (3,0)--(3+1,0);
        \draw[blue] (3,0)--(3-0.5,-1);
        \draw (3+3,0) node[circle, fill, inner sep=1.5pt]{};
        \draw[red] (3+3,0)--(3+3-0.5,1);
        \draw[blue] (3+3,0)--(3+1+3,0);
        \draw[red] (3+3,0)--(3+3-0.5,-1);
        \draw (3+3+3,0) node[circle, fill, inner sep=1.5pt]{};
        \draw[blue] (3+3+3,0)--(3+1+3+3,0);
        \draw[red]  (3+3+3,0)--(3-1+3+3,0);
    \end{tikzpicture}
    \caption{节点的四种情形}
    \label{fig:cross-graph-3}
\end{figure}
固定任意一点 $k_1$, 选择红色边前进到 $k_2$, 再由 $k_2$ 出发, 选择蓝色边前进到 $k_3$, 若 $k_3$ 与 $k_1$ 之间也有边, 则构成了边不同色的三角形, 这与 引理 \ref{lma-3-3-1} 矛盾. 所以按此步骤至少前 $3$ 个节点各不相同. 由于节点的四种情形均有两种颜色的边, 除非遇到先前经过的点, 否则该步骤可以一直持续下去. 由于节点个数的有限性, 必然存在第一个 $k_j$ 使得 $k_j=k_l$, $l<j-2$ (若 $l=j-2$, 则 $k_{j-1},k_{j-1},k_{j}$ 构成异色三角形), 则可得简单交错闭路径 $O(k_l,\cdots,k_j)$.
\fi
$(3)\Rightarrow (2)$. 若不存在长度大于 $3$ 的简单闭路径, 我们需要证明对任意的 $k\in \Pi_{a_1}\cap \Pi_{a_2}$ 有 $c_k=0$. 将给情形分为下述两种情况:
\begin{enumerate}
    \item [(i)] 不存在简单闭路径. 则 $G(a_1,a_2)$ 的连通部分是一个树. 若连通部分只有一个节点 $k$, 对满足 $m\neq k, a_1=m^2+mk+k^2$ 的 $m$, 必有 $m\in\Gamma_{a_1}$ 且 $m\notin \Gamma_{a_2}$, 所以由 $u(t_1+t,x_1-a_1t)=u(t_2+t,x_2-a_2t)=0$ 及定理 \ref{thm-3-3-1} 知 $c_{m}=0, d_{1,k}=0$. 而 $d_{1,k}$ 是所有满足 $m^2+mk+k^2=a_1$ 的数 $c_me^{i(m^3t_1+mx_1)}$ 的和, 且其中 $m\neq k$ 的每一项都为零, 所以 $c_{k}=0$. 所以接下来不妨设连通部分至少两个节点.
    
    若存在一个节点有三条边, 则由引理 \ref{lma-3-3-2} 可知两条同色边可构成同色三角形, 即简单闭路径, 矛盾, 所以每个节点的边数小于等于 $2$. 该情况下, 图 $G(a_1,a_2)$ 的每个连通部分均为一条交错路径, 如图 \ref{fig:cross-path} 所示.
    \begin{figure}[htbp]
        \centering
        \begin{tikzpicture}
            \draw (0,0) node[circle, fill, inner sep=1.5pt]{};
            \draw (1,0) node[circle, fill, inner sep=1.5pt]{};
            \draw (2,0) node[circle, fill, inner sep=1.5pt]{};
            \draw (4,0) node[circle, fill, inner sep=1.5pt]{};
            \draw (5,0) node[circle, fill, inner sep=1.5pt]{};
            \draw (6,0) node[circle, fill, inner sep=1.5pt]{};
            \draw (0,0) node[below] {$k_1$};
            \draw (1,0) node[below] {$k_2$};
            \draw (2,0) node[below] {$k_3$};
            \draw (4,0) node[below] {$k_{n-2}$};
            \draw (5,0) node[below] {$k_{n-1}$};
            \draw (6,0) node[below] {$k_n$};
            \draw [red] (0,0)--(1,0);
            \draw [blue] (1,0)--(2,0);
            \draw [red] (2,0) --(2.5,0);
            \draw [loosely dotted] (2.5,0)--(3.5,0);
            \draw [blue] (3.5,0) --(4,0);
            \draw [red] (4,0)--(5,0);
            \draw [blue] (5,0)--(6,0);
        \end{tikzpicture}
        \caption{交错路径}
        \label{fig:cross-path}
    \end{figure}
    考虑作为端点的节点 $k_1$, 对于任意满足 $m\neq k_1,a_2=m^2+mk_1+k_1^2$ 的 $m$, 因为 $m\in \Gamma_{a_2}$, 所以 $m\notin \Gamma_{a_1}$. 由 $u(t_1+t,x_1-a_1t)=u(t_2+t,x_2-a_2t)=0$ 及定理 \ref{thm-3-3-1} 知 $c_m=0, d_{2,k_1}=0$. 而 $d_{2,k_1}$ 是所有满足 $m^2+mk_1+k_1^2=a_2$ 的数 $c_m e^{i(m^3t_2+mx_2)}$ 的和, 且其中 $m\neq k_1$ 的每一项都为零, 所以 $c_{k_1}=0$. 考虑节点 $k_2$, 以及任意满足 $m\neq k_1,a_1=m^2+mk_1+k_1^2$ 的 $m$. 已知 $m=k_1$ 的情形下 $c_{k_1}=0$, 此时便可将其视作只有蓝色线的端点, 再由同 $k_1$ 一样的论证可得 $c_{k_2}=0$. 依次进行下去, 可得路径上的节点的值全为 $0$.
    \item [(ii)] 仅存在长度等于 $3$ 的闭路径, 由引理 \ref{lma-3-3-2} 知必构成同色三角形. 不妨假设图 $G(a_1,a_2)$ 本身连通. 
    
    我们对图的节点个数 $n$ 进行归纳. 当 $n=3$ 时, $G(a_1,a_2)$ 是一个同色三角形, 不妨设颜色为红色, 则对该三角形的任意一个节点 $k$, 由 $u(t_1+t,x_1-a_1t)=u(t_2+t,x_2-a_2t)=0$ 及定理 \ref{thm-3-3-1} 可知满足 $m\neq k, a_2=m^2+mk+k^2$ 的 $m$ 满足 $c_m=0$, 再由 $d_{2,k}=\sum\limits_{a_2=m^2+mk+k^2,m\in\Z}c_m e^{i(m^3t_2+mx_2)}$ 可推出 $c_k=0$.
    当 $n=4$ 时, 必然是一个同色三角形与一个节点 $l$ 相连, 由于 $l$ 仅有一条边, 所以 $c_l=0$, 则可以将该图视作只有同色三角形的情形. 假设节点数为 $n-2,n-1$ 时, 可推出所有节点的值为 $0$, 现在我们考虑节点个数为 $n$ 的情形. 若存在一个端点, 即一个节点且只有一条边与其相连, 同样地则可以将其去掉, 则图可以视为节点个数 $n-1$ 的情形, 再利用归纳假设即可. 若不存在端点, 则必然存在同色三角形仅与一条边相连 (否则会出现长度大于 $3$ 的简单闭路径), 如图  \ref{no-end-points} 所示.
    \begin{figure}[htbp]
        \centering
        \begin{tikzpicture}
            \draw (-0.5,1) node[circle, fill, inner sep=1.5pt]{};
            \draw (-0.5,1) node[left] {$k_1$};
            \draw (-0.5,-1) node[circle, fill, inner sep=1.5pt]{};
            \draw (-0.5,-1) node[left] {$k_2$};
            \draw (1,0) node[circle, fill, inner sep=1.5pt]{};
            \draw (1,0) node[above] {$k_3$}; 
            \draw (2,0) node[circle, fill, inner sep=1.5pt]{};
            \draw (2,0) node [above] {$k_4$};
            \draw[red] (-0.5,-1)--(-0.5,1)--(1,0)--(-0.5,-1);
            \draw [blue] (1,0)--(2,0);
            \draw [red] (2,0) -- (2.5,0);
            \draw [loosely dotted] (2.5,0)--(3,0);
        \end{tikzpicture}
        \caption{不存在端点的情形}
        \label{no-end-points}
    \end{figure}
    由 $k_1$ 没有蓝色边可知, 对于任意满足 $m\neq k_1,a_2=m^2+mk_1+k_1^2$ 的 $m$, 由 $u(t_1+t,x_1-a_1t)=u(t_2+t,x_2-a_2t)=0$ 及定理 \ref{thm-3-3-1} 知 $c_m=0,d_{2,k_1}=0$. 再由 $d_{2,k_1}=\sum\limits_{a_2=m^2+mk_1+k_1^2,m\in\Z} c_me^{i(m^3t_2+mx_2)}$ 可推出 $c_{k_1}=0$. 同理可推出 $k_2=0$. 此时可将 $k_1,k_2$ 从图中去除, 则变成了节点个数为 $n-2$ 的图, 再利用归纳假设即可.   
\end{enumerate}

(2) $\Rightarrow$ (3). 考虑证明其逆否命题. 若存在长度大于 $3$ 的简单闭路径, 则其必为双色. 若不然, 则由引理 \ref{lma-3-3-2} 可知存在四个节点 $k_i,i=1,2,3,4$ 两两之间由相同颜色相连, 不妨设为红色, 则 $k_i^3-a_1 k_i=k_j^3-a_1k_j,\forall i,j$, 但这是不可能的. 再由引理 \ref{lma-3-3-3} 知存在简单交错闭路径, 设该交错闭路径为 $O(k_1,k_2,\cdots,k_{2n}), k_i\neq k_j, \forall i\neq j$. 不失一般性设 $k_1$ 与 $k_2$ 间的边为红色. 不妨令 $(t_0,x_0)=(t_1,x_1)=(t_2,x_2)$, 若不然, 则设 $(t_0,x_0)$ 为 $x=x_1-a_1(t-t_1)$ 与 $x=x_2-a_2(t-t_2)$ 的交点. 令 
\begin{equation*}
    u_0(x)=\sum_{j=1}^{2n}(-1)^je^{-i(k_j^3t_0+k_jx_0)}e^{ik_jx}.
\end{equation*}
则
\begin{equation*}
    u(t,x)=\sum_{j=1}^{2n}(-1)^j e^{-i(k_j^3t_0+k_jx_0)}e^{ik_j^3 t}e^{ik_jx} 
\end{equation*}
将 $(t,x)$ 由 $(t_0+t,x_0-a_1t)$ 代替可得对任意的 $t\in \R$ 有
\begin{align*}
    u(t_0+t,x_0-a_1t)&=\sum_{j=1}^{2n}(-1)^j e^{-i(k_j^3t_0+k_jx_0)}e^{ik_j^3 (t_0+t)}e^{ik_j(x_0-a_1t)} \\
    &=\sum_{j=1}^{2n}(-1)^j e^{i(k_j^3-a_1)t}\\
    &=\left(-e^{i(k_1^3-a_1 k_1)t)}+e^{i(k_2^3-a_2k_2)t)}\right)+\cdots\\
    &+\left(-e^{i(k_{2n-1}^3-a_1k_{2n-1})t)}+e^{i(k_{2n}^3-a_2k_{2n})t)}\right)\\
    &=0.
\end{align*}
上式最后一步用到了路径颜色的交错性.
同理对任意的 $t\in \R$, 有 
\begin{equation*}
    u(t_0+t,x_0-a_2t)=0.
\end{equation*}
然而 $\sum_{k\in \Z} |c_k|^2=2n$, 所以由解函数在两条线段上为零不可推出 $u_0=0$.

由 (1) $\Rightarrow$ (2) 显然, 所以最后只需要证明 (2) $\Rightarrow$ (1) 即可. 实际上, 这等价于证明
\begin{equation*}
    \sum_{k\in \Z} |c_k|^2\lesssim \sum_{k\notin \Pi_{a_1}\cap \Pi_{a_2}}|c_k|^2+\sum_{k\in \Pi_{a_1}\cap \Pi_{a_2}} |d_{1,k}|^2+|d_{2,k}|^2.
\end{equation*}
上式可化简为
\begin{equation}\label{final}
    \sum_{k\in \Pi_{a_1}\cap \Pi_{a_2}}|c_k|^2\lesssim \sum_{k\in \Pi_{a_1}\cap \Pi_{a_2}} |d_{1,k}|^2+|d_{2,k}|^2.
\end{equation}
我们使用反证法, 假设存在序列 $c_k^{(i)},i=1,2,\cdots,$ 使得当 $i\to \infty$ 时有
\begin{equation*}
    \frac{\sum_{k\in \Pi_{a_1}\cap \Pi_{a_2}} |d^{(i)}_{1,k}|^2+|d^{(i)}_{2,k}|^2}{ \sum_{k\in \Pi_{a_1}\cap \Pi_{a_2}}|c^{(i)}_k|^2}\longrightarrow 0.
\end{equation*}
由式 \ref{final} 的其次性, 可不失一般性, 设 $\sum_{k\in \Pi_{a_1}\cap \Pi_{a_2}}|c_k^{(i)}|^2=1, i=1,2,\cdots$. 由 $\Pi_{a_1}\cap \Pi_{a_2}$ 中元素只有有限个可知当 $i\to \infty$ 时 $d^{(i)}_{1,k}\to 0, d^{(i)}_{2,k}\to 0,\forall k\in \Pi_{a_1}\cap \Pi_{a_2}$. 设极限分别为 $c^{(0)}_k, d^{(0)}_{1,k}=d^{(0)}_{2,k}=0, \forall k\in \Pi_{a_1}\cap \Pi_{a_2}$, 且 
\begin{equation}\label{final-1}
    \sum_{k\in \Pi_{a_1}\cap \Pi_{a_2}}|c_k^{(0)}|^2=1.
\end{equation}
令 $u_0=\sum_{k\in \Pi_{a_1}\cap \Pi_{a_2}}c^{(0)}_k e^{ikx}$, 
从而由式 \eqref{final}可得 $c_k=0,\forall k\in \Pi_{a_1}\cap \Pi_{a_2}$, 这与式 \eqref{final-1} 相矛盾. 所以不等式 \eqref{obs-two-points} 成立.
\end{proof}
上面的定理是用边二染色图来刻画出能观测性, 由该定理可知, 给定任意两个不同速率的移动观测点, 能够建立相应能观测不等式当且仅当其在两条线段轨迹上有唯一延拓性. 另一方面, 为了验证其可观测性, 可通过作图 $G(a_1,a_2)$ 并尝试找到其长度大于 3 的简单闭回路. 

实际操作中, 定理 \ref{abstract-thm} 的刻画比较复杂, 接下来我们给出两个相对具体的判定方法. 第一个是利用 $a_1,a_2$ 的素因式分解, 第二个是利用函数 $y=x^3-ax$ 的增减性. 这两种方法在判别两个移动观测点的可观测性时更加简单.

\begin{theorem}\label{thm-use-prime}
给定任意的 $T>0$, 考虑方程 \eqref{kdv-r} 的解. 设 $a_1,a_2$ 为两个不同的正整数, $(t_1,x_1),(t_2,x_2)\in \R\times \T$. 若存在素数 $p\equiv 2\pmod{3}$ 使得 $\mathrm{deg}_p(a_1)\neq \mathrm{deg}_p(a_2)$, 则由
\begin{equation*}
    u(t_1+t,x_1-a_1t)=u(t_2+t,x_2-a_2t)=0,\quad \forall t\in (0,T)
\end{equation*}
可推出 $u$ 是平凡解, 即 $u_0=0$, 并且在该情形下存在不等式 \eqref{obs-two-points}.
\end{theorem}
\begin{proof}
不妨令 $\mathrm{deg}_p(a_1)>\mathrm{deg}_p(a_2),a_1,a_2\in \Gamma$, 由推论 \ref{crc-3-2-2} 可设 $\mathrm{deg}_p(a_1)=2e,e\in\N_{+}$, 则 $p^{2e}|a_1$. 令 $a_1=k^2+km+m^2,k\neq m$, 则 $a_1=(k-m\omega)(k-m\overline{\omega})$, 因为 $p\equiv 2\pmod{3}$, 由定理 \ref{thm-eisenstein} 知 $p$ 为 Eisenstein 素数, 则由唯一分解性质可得 $p^e | k-m\omega,p^e|k-m\overline{\omega}$, 进而有 $p^e|k, p^e|m$. 设 $k\in \Pi_{a_2}$, 则存在 $l\neq k$ 使得 $a_2=k^2+kl+l^2$ 成立. 由 $p^{2e}\nmid a_2$ 可得 $p^e\nmid k-l\omega,p^e\nmid k-l\overline{\omega}$, 再由 $p^e|k$ 可得 $p^e\nmid l$, 所以 $l\notin \Gamma_{a_1}$. 所以连接点 $k\in \Gamma_{a_1}$ 的边只能是红色边, 即 $k$ 是个孤立点, 孤立线段或孤立的红色三角形的一部分. 既然红色边都以这种形式存在, 则蓝色边也只会以孤立点, 孤立线段或孤立蓝色三角形的形式存在. 再由定理 \ref{abstract-thm} 的 (3) 可得结论. 
\end{proof}

\begin{theorem}\label{a24a1}
给定任意的 $T>0$, 考虑方程 \eqref{kdv-r} 的解. 设 $a_1,a_2$ 为两个不同的正整数, 且满足 $a_2>4a_1>0$, $(t_1,x_1), (t_2,x_2)\in \R\times \T$. 则由 
  \begin{equation*}
        u(t_1+t,x_1-a_1t)=u(t_2+t,x_2-a_2t)=0,\quad \forall t\in (0,T)
  \end{equation*}
可推出$u$ 是平凡解, 即 $u_0=0$, 并且在该情形下存在不等式 \eqref{obs-two-points}.
    \end{theorem}
    \begin{proof}
假设 $k\in  \Pi_{a_1}$. 此时只需要考虑 $k\in \left[0, 2\left(\frac{a_1}{3}\right)^{1 /2}\right]$ 的情形, 如图 \ref{fig6} 所示.
          \begin{figure}[htbp]
        \centering
         \begin{tikzpicture}[scale=1.2]
           % axes 
           \draw[very thick,->] (-3.5,0) -- (3.5,0) node[right] {$x$};
           \draw[very thick,->] (0,-4) -- (0,4) node[above] {$y$};
           \draw (-0.2,-0.01) node[below] {\small $O$};
           % functions
         \draw[red] plot[domain=-1.8:1.8, smooth] (\x, \x*\x*\x-1*\x);
         \draw[blue] plot[domain=-2.1:2.1, smooth] (\x, \x*\x*\x-4.2*\x);
         \draw[dashed] (0.7,0) node[above] {$k$} -- (0.7,-2.62);
         \draw[dashed] (-1.8,-2.62) -- (1.6,-2.62);
         \draw[dashed] (1.6,-2.62) -- (1.6,2.53);
         \draw (1.6,0) node[above right] {$k'$};
         \draw (1.9,4) node[right] {$y=x^3-a_1x$};
         \draw (2.2,0.5) node[right] {$y=x^3-a_2x$};
        \end{tikzpicture}
        \caption{函数$y=x^3-a_1x$ 和 $y=x^3-a_2 x$ 的图像.}
        \label{fig6}
    \end{figure}
     若 $k\in \Pi_{a_2} $, 则必然存在另一个不同的 $k'$ 使得 $k^3-a_2k={k'}^3-a_2k'$, 但是由函数图像可知 $|k'|\ge\left(\frac{a_2}{3}\right)^{1 /2}>2\left(\frac{a_2}{3}\right)^{1 /2}$, 所以 $k'\notin \Pi_{a_1}$. 由定义式
     \begin{equation*}
         d_{2,k}=\sum\limits_{\substack{k'\neq k\in \Z,\\ k^2+mk+m^2=a_2}} c_{k'}e^{i(k'^3t_2+k'x_0)}+c_k e^{i(k^3t_2+kt_2)}
     \end{equation*}
     以及条件 $u(t_1+t,x_1-a_1t)=u(t_2+t,x_2-a_2t)=0,\forall t\in (0,T)$ 可得 $$\sum\limits_{\substack{k'\neq k\in \Z,\\ k^2+mk+m^2=a_2}} c_{k'}e^{i(k'^3t_2+k'x_0)}=0$$ 以及 $d_{2,k}=0$, 进而 $c_k=0$. 所以 $u_0=0$. 再由定理 \ref{abstract-thm} 可知不等式 \eqref{obs-two-points} 成立.
      \end{proof}
\iffalse      
\begin{theorem}
给定任意的 $T>0$ , 考虑方程 \eqref{kdv-r} 的解. 设 $n\ge 3$ 个不同的实数 $a_1<a_2<\cdots <a_n$, $(t_i,x_i)\in \R\times \T, i=1,2,\cdots,n$. 则由
\begin{equation*}
    u(t_1+t,x_1-a_1t)=u(t_2+t,x_2-a_2t)=\cdots=u(t_n+t,x_n-a_nt)=0,\quad \forall t\in (0,T)
\end{equation*}
可推出 $u$ 是平凡解, 即 $u_0=0$, 并且在该情形下不等式 
\begin{equation}
    \sum_{k\in\Z}|c_k|^2\lesssim \int_{0}^T\sum_{i=1}^n|u(t_i+t,x_i-a_i t)|\d t
\end{equation}
成立.
\end{theorem}
\begin{proof}
不妨令 $a_i\in \Gamma, a_i>0, \forall i=1,2,\cdots,n$. 若 $a_n>4a_1$, 则可由定理 \ref{a24a1} 直接推出结论. 

设 $a_n\le 4a_1$, 则由定理 \ref{thm-use-prime} 可知对任意的素数 $p\equiv 2\pmod{3}$, $\mathrm{deg}_p(a_i)=\mathrm{deg}_p(a_j),\forall i\neq j$. 所以不妨令所有的 $a_i$ 都不存在模 $3$ 余 $2$ 的素因子. 
\end{proof}

\fi

   


   \chapter{偏微分方程控制论中的应用}
本章主要介绍第二章和第三章中的结果在偏微分方程控制理论中的应用.

本章第一节主要利用第二章与第三章的结果, 得到KdV方程的可控性定理. 本章第二节是对周期情形下非线性结果的进一步讨论, 并得到非线性情形下关于的可控性定理.

\section{基本概念与HUM方法}
设 $\mathbb{K}$ 为 $\C$ 或 $\R$, $\mathcal{H}(\Omega)$ 为定义域在 $\Omega\subset \R^n$ 上的函数构成的希尔伯特空间, 其内积和范数分别表示为 $\langle \cdot,\cdot \rangle_{\mathcal{H}(\Omega)}$ 和 $\|\cdot\|_{\mathcal{H}(\Omega)}$. 设 $U$ 为 $\mathcal{H}$ 上的线性算子. 


考虑下述线性方程:
\begin{equation}
    \left\lbrace\begin{array}{ll}
        \partial_t \Phi+U\Phi=0, & \Phi=\Phi(t)\in C([0,T];\mathcal{H}(\Omega)), \\
         \Phi(0)=\Phi_0,& \Phi_0\in \mathcal{H}(\Omega). 
    \end{array}
    \right.\label{controllability-abstract-equation}
\end{equation}
对于不同的初始值 $\Phi_0$, 可以得到相应的 $\Phi(T)$. 在偏微分方程的控制理论中, 我们通常对上述方程加一个外界控制, 使得其能在 $T$ 时刻得到给定的 $\Phi(T)$. 对给定初始值 $\Phi_0$ 和目标函数 $\Phi_1$ 的方程 \eqref{controllability-abstract-equation}, 若存在这样的外界控制使得 $\Phi(T)=\Phi_1$, 则说明方程具有某种可控性. 为了更加清晰明确地描述方程的可控性, 下面给出相关定义. 
\begin{definition}
 考虑下述带有控制项 $f\in C([0,T];\mathcal{H}(\Omega))$ 的方程
\begin{equation}
        \partial_t \Phi+P\Phi=f, 
    \label{controllability-abstract-equation-nonhomogeneous}  
\end{equation}
$P=P(u,\partial_x u,\cdots,\partial_x^m u)$ 为线性或非线性的偏微分算子. 若对任意 $\Phi_0,\Phi_1\in \mathcal{H}(\Omega)$, 都可以选取合适的 $f$ 使得 $\Phi(0)=\Phi_0, \Phi(T)=\Phi_1$, 则我们称方程 \eqref{controllability-abstract-equation-nonhomogeneous} 精确可控. 若固定 $\Phi_1=0$, 即对任意的 $\Phi_0$ 都存在控制函数 $f$ 使得 $\Phi(T)=0$, 则称方程 \eqref{controllability-abstract-equation-nonhomogeneous} 零值可控. 
\end{definition}

\begin{definition}
 我们称算子 $U$ 为斜自伴随, 如果其满足
\begin{equation*}
    \langle Uf,g\rangle = - \langle f,Ug\rangle,\quad \forall f,g\in \mathcal{H}(\Omega).
\end{equation*}
\end{definition}

\begin{proposition}\label{null-equiv-exact}
对于带有控制项 $f\in C([0,T];\mathcal{H}(\Omega))$ 的线性方程
\begin{equation}
    \partial_t\Phi+U\Phi=f,\label{control-fukk}
\end{equation}
若 $U$ 在 $\mathcal{H}(\Omega)$ 上生成自同构 $e^{-tU}$ 构成的强连续群, 则其零值可控性与精确可控性等价.
\end{proposition}
\begin{proof}
由定义知精确可控性包含零值可控性, 我们只要说明零值可控性能够推出精确可控性. 任给 $\Phi_0,\Phi_1\in \mathcal{H}(\Omega)$, 考虑方程
\begin{equation}
    \left\lbrace
    \begin{array}{ll}
        \partial_t\Psi+U\Psi=0, &   \\
         \Psi(T)=\Phi_1.& 
    \end{array}
    \right.
\end{equation}
由 $U$ 的性质可知上述方程有解, 记作 $\Psi(t)$. 由方程的零值可控性知, 存在控制函数 $f$ 使得初值为 $\Phi_0-\Psi(0)$ 的方程 \eqref{control-fukk} 的解函数 $\Phi^{(1)}$ 满足 $\Phi^{(1)}(T)=0$. 进而 $\Phi(t)=\Psi(t)+\Phi^{(1)}(t)$ 即是方程 \eqref{control-fukk} 的解, 且 $\Phi_(0)=\Phi_0,\Phi(T)=\Phi_1$. 
\end{proof}

希尔伯特唯一性方法 (Hilbert Uniqueness Method, 简称HUM方法), 是一种由能观测不等式得到线性方程可控性的方法. 具体来说, 考虑初值为 $\Phi_0$ 的线性方程 \eqref{controllability-abstract-equation}
的解 $\Phi(t)$, 假设存在常数 $C>0$ 使得能观测不等式
\begin{equation}
    \|\Phi_0\|_{\mathcal{H}(\Omega)}^2\le C\int_0^T\langle \Phi(t),\chi_{\omega} \Phi(t)\rangle_{\mathcal{H}(\Omega)}\d t 
\end{equation}
成立, 其中 $\omega\subset \Omega$. 设方程 \eqref{controllability-abstract-equation} 的对偶方程为
\begin{equation}
    \left\lbrace
    \begin{array}{ll}
        \partial_t \Psi+U\Psi=\chi_{\omega}\Phi, &  \\
         \Psi(T)=0.& 
    \end{array}
    \right.\label{tired}
\end{equation}
将方程 \eqref{tired}两边同时与 ${\Phi}$ 作内积, 并在 $[0,T]$ 上积分可得
\begin{equation*}
    \int_0^T\langle \Phi,\partial_t \Psi\rangle+\langle \Phi,U\Psi\rangle =\int_0^T\langle \Phi,\chi_{\omega}\Phi\rangle \d t.
\end{equation*}
上式左边经分部积分以及 $U$ 的斜自伴随性可得
\begin{equation*}
    -\langle \Phi_0,\Psi_0\rangle=\int_0^T\langle \Phi,\chi_{\omega}\Phi\rangle\ge \frac{1}{C}\|\Phi_0\|^2_{\mathcal{H}(\Omega)},
\end{equation*}
上式第二个关系用到能观测不等式. 令 $\Psi_0=-S\Phi_0$, 则有
\begin{equation}
    \frac{1}{C}\|\Phi_0\|^2_{\mathcal{H}(\Omega)}\le\langle \Phi_0, S\Phi_0\rangle \le T\|\phi_0\|^2_{\mathcal{H}(\Omega)}, 
\end{equation}
这里用到了条件 $\langle \Phi,\chi_{\omega}\Phi\rangle\le \|\Phi\|^2_{\mathcal{H}(\Omega)}$, 这对于许多常见的希尔伯特空间是正确的.
上式说明我们所定义的线性映射 $S$ 是有界且可逆映射, 这说明对于任意的 $\Psi_0$, 都存在控制 $\Phi_0=-S^{-1}\Psi_0$ 使得方程 \eqref{tired} 成立, 即具有零值可控性. 若 $U$ 还在 $\mathcal{H}(\Omega)$ 上构成强连续群, 则由性质 \ref{null-equiv-exact} 可得精确可控性.

\section{脉冲控制函数下的方程可控性}
首先, 我们考虑全空间 $\R$ 中线性KdV方程 \eqref{kdv-r}, 由第二章的主要结果可知, 对某些满足条件的可测集 $A$ 和 $B$, 我们可以建立能观测不等式 \eqref{kdvobs-2}, 为叙述方便, 这里将该不等式重新写作
     \begin{equation}
         \int_{\R} |u_0|^2 \d x\le C\left(\int_{A^c}|u(t_1,x)|^2\d x+\int_{B^c}|u(t_2,x)|^2\d x\right),\label{kdv-r-4}
     \end{equation}
     其中 $0\le t_1<t_2\le T$.
在本章的第一节中, 我们详细介绍了HUM方法, 在那里我们用到的能观测不等式建立在一段时间上. 同样地, 利用HUM方法, 我们也 可以由两点时刻能观测不等式得到相应的可控性定理. 我们首先以全空间中的KdV方程为例, 给出其完整的证明, 从证明中可以看到观测时间从一段时间变成两点时刻对证明步骤没有本质的影响. 

\begin{theorem}
考虑方程
\begin{equation}
    \left\lbrace\begin{array}{ll}
        \partial_t u +\partial_x^3u=\delta(t-t_1)\chi_{A^c}h_1(x)+\delta(t-t_2)\chi_{B^c}h_2(x), & (t,x)\in (0,T)\times \R \\
        u(0,x)=u_0(x),& x\in \R,
    \end{array}\right.\label{kdv-r-dual}
\end{equation}
其中 $u_0(x),h_1(x),h_2(x)\in L^2(\R)$. 记 $u(t,x;u_0,h_1,h_2)$ 为方程 \eqref{kdv-r-dual} 的解. 设 $A$ 与 $B$ 是 $\R$ 中满足能观测不等式 \eqref{kdv-r-4} 的可测集, $0\le t_1<t_2\le T$. 
则对任意的 $u_0(x)\in L^2(\R)$ 和 $u_1(x)\in L^2(\R)$, 存在一对控制函数 $(h_1,h_2)\in L^2(\R)\times L^2(\R)$ 使得
\begin{equation*}
    u(T,x;u_0,h_1,h_2)=u_1(x), \quad x\in \R,
\end{equation*}
且
\begin{equation}
    \|h_1\|^2_{L^2(\R)}+\|h_2\|^2_{L^2(\R)}\le 2C \|u_1-e^{-\partial_x^3 T}u_0\|^2_{L^2(\R)},
\end{equation}
其中常数 $C$ 与不等式 \eqref{kdv-r-4} 中常数相同.
\end{theorem}
\begin{proof}
考虑方程 \eqref{kdv-r} 的对偶方程
\begin{equation}
        \left\lbrace\begin{array}{ll}
        \partial_t v +\partial_x^3v=\delta(t-t_1)\chi_{A^c}u(t,x)+\delta(t-t_2)\chi_{B^c}u(t,x), & (t,x)\in (0,T)\times \R \\
        v(T,x)=0,& x\in \R,
    \end{array}\right.\label{kdv-r-dual-2}
\end{equation}
记 $v(0,x)=-S u_0(x)$. 若能够证明 $S$ 为双射, 则对任意的 $v(0,x)$, 都可取 $u_0(x)=S^{-1}v(0,x)$, 且 $ h_1(x)=u(t_1,x),h_2(x)=u(t_2,x)$, 使得方程 \eqref{kdv-r-dual-2} 成立. 换句话说, 对方程 \eqref{kdv-r-dual} 以及任意的初始函数 $u_0(x)$, 存在一对控制函数 $(h_1,h_2)\in L^2(\R)\times L^2(\R)$ 使得
\begin{equation*}
    u(T,x;u_0,h_1,h_2)=0,\quad x\in\R
\end{equation*}
且
\begin{equation*}
     \|h_1\|^2_{L^2(\R)}+\|h_2\|^2_{L^2(\R)}\le 2 \|S^{-1}u_0\|^2_{L^2(\R)}.
\end{equation*}
对于 $u(T,x;u_0,h_1,h_2)=u_1(x)\neq 0$ 的情形, 则可令 $\tilde{u}(t,x)=u(t,x)-u(T,x)e^{-\partial_x^3(t-T)}$, 新的函数 $\tilde{u}(t,x)$ 满足方程 \eqref{kdv-r-dual-2} 且 $\tilde{u}(T,x)=0$.

由上述分析可知, 我们只需要证明 $S$ 是双射且 $\|S^{-1}\|\le C$. 方程 \eqref{kdv-r-dual-2} 两边同时乘以 $\overline{u}$ 并积分, $u$ 为满足方程 \eqref{kdv-r} 的解, 可得
\begin{equation*}
    \int_0^T\int_{\R} \overline{u}(\partial_t v+\partial_x^3 v)\d x\mathrm{d}t=\int_{A^c}|u(t_1,x)|^2\d x+\int_{B^c}|u(t_2,x)|^2 \d x.
\end{equation*}
等式右边经分部积分可得
\begin{equation*}
    -\int_{\R} \overline{u_0}(x)v(0,x)\d x =\int_{A^c}|u(t_1,x)|^2\d x\mathrm{d}t+\int_{B^c}|u(t_2,x)|^2 \d x\mathrm{d}t.
\end{equation*}
进而由上式以及能观测不等式 \eqref{kdv-r-4}可得
\begin{equation*}
    2\|u_0\|^2_{L^2(\R)}\ge\langle u_0,Su_0 \rangle_{L^2(\R)}\ge C^{-1}\|u_0\|^2_{L^2(\R)}.
\end{equation*}
所以 $S$ 必为双射且 $\|S^{-1}\|\le C$.
\end{proof}

\begin{remark}
上述定理可以理解为: 对任意的 $u_0,u_1\in L^2(\R)$, 都存在一对控制函数 $(h_1,h_2)\in L^2(\R)\times L^(\R)$ 使得方程 \eqref{kdv-r-dual} 的解从 $u_0$ 经时间 $T$ 到达 $u_1$. 实际上, 根据方程右边的示性函数可知, 我们只需要给出控制 $(h_1,h_2)$ 在 $A^c\times B^c$ 上的分布即可. 该控制仅在时间 $t_1$ 和 $t_2$ 时刻以 脉冲的形式出现.
\end{remark}


由第三章定理 \ref{thm-3-3-1} 知, 对于周期边界情形下的线性KdV方程, 若 $a\notin \Gamma$, 则对任意固定的 $\tau>0$, 我们有能观测不等式 
 \begin{equation}
     \int_{\T}|u_0|^2\d x\le C\int_{0}^\tau |u(t_0+t,x_0-at)|^2\d t,\label{holyshit}
 \end{equation}
 其中 $C=C(\tau, a)$ 是依赖于 $\tau$ 和 $a$ 的常数.
 
同样地, 我们也可以利用该能观测不等式以及HUM方法建立相应的可控性定理:
\begin{theorem}
 考虑方程
 \begin{equation}
     \left\lbrace\begin{array}{ll}
         \partial_t u+\partial_x^3u=\delta (x-x_0+at) \chi_{[t_0,t_0+\tau]}(t) h(t), & (t,x)\in (0,T)\times \T\\
         u(0,x)=u_0(x), & x\in \T,
     \end{array}\right.\label{kdv-l-control}
 \end{equation}
 其中 $u_0(x),h(t)\in L^2(\R)$, $t_0$ 和 $x_0$ 分别是两个任意给定的时刻与位置. 记 $u(t,x;u_0,h)$ 为方程 \eqref{kdv-l-control} 的解. 设 $a\notin \Gamma$ {{\rm{(}}即满足定理 \rm{\ref{thm-3-3-1} (2))}}, $0<\tau\le T$. 则对任意 $u_0(x)\in L^2(\T)$ 和 $u_1(x)\in L^2(\T)$, 以及任意的 $\tau$ 存在控制函数 $h\in L^2([t_0,t_0+\tau];\C)$ 使得 
 \begin{equation}
     u(T,x;u_0,h)=u_1(x),\quad x\in \T,
 \end{equation}
 且 
 \begin{equation}
     \|h\|^2_{L^2([t_0,t_0+\tau];\C)}\le C\| u_1-W(T)u_0\|^2_{L^2(\T)},
 \end{equation} 其中 $W(t) $ 是由 $-\partial_x^3$ 生成的 $L^2(\T)$ 上的强连续群.
\end{theorem}



\iffalse    \section{能观测不等式}
    首先我们给出关于一个移动观测点的能观测不等式成立的条件, 下述定理的证明遵循了 \cite{jaming2020moving} 中对薛定谔方程关于移动观测点能观测性的证明.
    \begin{theorem}\label{thm3-2-1}
    设 $(t_0,x_0)\in \R\times \T$, $a\in \R$. 则方程 \eqref{kdv-l} 的解满足:
    \begin{enumerate}
        \item[\rm{(1)}] 对任意的 $T>0$, 存在常数 $C>0$ 使得不等式
        \begin{equation}\label{3-2-1}
            \int_0^T\left| u(t_0+t,x_0-at) \right|^2\d t \le C  \sum_{k\in \Z} |c_k|^2
        \end{equation}
        成立.
        \item[\rm{(2)}] 定义集合
        \begin{equation*}
            \Gamma=\lbrace n:\exists m\neq k, m,k\in \Z \text{ s.t. }n=k^2+mk+m^2 \rbrace.
        \end{equation*}
        则对任意的 $T>0$ 以及 $a\notin \Gamma$, 存在常数 $C>0$ 使得不等式
        \begin{equation}\label{3-2-2}
            \sum_{k\in\Z} |c_k|^2 \le C \int_0^T |u(t_0+t,x_0-at)|^2 \d t
        \end{equation}
        成立.
        \item[\rm{(3)}] 若 $a\in \Gamma$, 其中 $\Gamma$ 定义同上, 则 $a>0$ 且
        \begin{equation}\label{3-2-3}
            \int_0^T |u(t_0+t,x_0-at)|^2\d t\asymp \sum_{k\in\Z}^\infty|d_k|^2,
        \end{equation}
        其中
        \begin{equation*}
            d_k=\sum_{\substack{
                m\in \Z,  \\
                k^2+mk+m^2 = a }} 
             c_me^{i(m^3t_0+mx_0)},\quad k\in \Z.
        \end{equation*}
        实际上, 当 $|k|> 2\left(\frac{a}{3}\right)^{1 /2}$ 时, 有 $d_k=c_k e^{i (k^3t_0+kx_0)}$.
        
        此时存在 $u\neq 0$ 使得
        \begin{equation*}
            u(t_0+t,x_0-at)=0, \quad\forall t\in \R.
        \end{equation*}
    \end{enumerate}
    \end{theorem}
    
    \begin{proof}
     (1) 固定 $a\in\R$, 简单计算可得
     \begin{equation}\label{3-2-4}
         u(t_0+t,x_0-at)=\sum_{k\in \Z} c_k e^{i(k^3(t_0+t)+k(x_0-at))}=\sum_{k\in \Z}c_k  e^{i (k^3t_0+kx_0)} e^{i(k^3-ak)t}.
     \end{equation}
    令 $\Lambda=\lbrace k^3-ak: k\in \Z\rbrace$, 若 $a<0$, 则有
    \begin{equation*}
        ((k+1)^3-a(k+1))-(k^3-ak)=3k^2+3k+1-a>1,\quad k\in \Z.
    \end{equation*}
    因此 $\Lambda$ 是一致分离集, 故由定理 \ref{thm-ingham} 知不等式 \eqref{3-2-1} 成立. 若 $a>0$, 设 $\sigma=\left(\frac{a}{3}\right)^{1 /2}$, 集合 $\Lambda$ 可分为下述三个集合的并:
    \begin{equation*}
        \Lambda_-=\lbrace k\in \Lambda : k<-\sigma\rbrace, \Lambda_0=\lbrace k\in \Lambda : -\sigma\le k\le \sigma\rbrace,\Lambda_+=\lbrace k\in\Lambda: k>\sigma\rbrace.
    \end{equation*}
    易知三个集合关于 $k$ 都是单调的, 所以三个集合都是一致分离集, 它们分别都满足不等式 \eqref{3-2-1}, 对着三个不等式求和, 再由不等式
    \begin{equation*}
        (x_1+x_2+\cdots+x_m)^2\le m (x_1^2+x_2^2+\cdots+x_m^2)
    \end{equation*}
    可得集合 $\Lambda$ 满足不等式 \eqref{3-2-1}.
    
    (2) 设 $a\notin \Gamma$, 则集合 $\Lambda$ 是一致分离的. 事实上, 对于两个不同的整数 $k$ 和 $m$, 我们有
    \begin{equation*}
        |(k^3-ak)-(m^3-am)|=|k-m||k^2+mk+m-a|\ge d(a,\Z):=\max(a-\lfloor a\rfloor,\lfloor a\rfloor +1-a). 
    \end{equation*}
    
    若 $a\le 0$, 取正整数 $N$, $k\neq m$ 且 $k,m\notin\lbrace-N,\cdots, N\rbrace$. 则利用等式
    \begin{equation*}
        (k^3-ak)-(m^3-am)=(k-m)(k^2+mk+m^2-a)
    \end{equation*}
    及简单的计算可得
    \begin{equation}\label{3-22-3}
        \left| (k^3-ak)-(m^3-am) \right|\ge \left\lbrace\begin{array}{ll}
            3N^2-a & k,m> N \text{ 或 } k,m <-N, \\
             2N^3-2aN  & km<0.
        \end{array}\right.
    \end{equation}
    所以
    \begin{equation*}
        \gamma\left(\Lambda \backslash \lbrace-N,\cdots,N\rbrace\right)\ge N^2.
    \end{equation*}
    令 $N\to \infty$, 则由定理 \ref{thm-haraux} 可得对任意的 $T>0$ 都有不等式 \eqref{3-2-2} 成立.
    
    若 $a>0$, 取 $N>\max\lbrace 2\sigma, a\rbrace$, $k\neq m, k,m\notin \lbrace-N,\cdots, N\rbrace$, 类似地我们也有估计 \eqref{3-22-3}. 同样令 $N\to \infty$ 可得对任意 $T>0$ 都有不等式 \eqref{3-2-2} 成立. 
    
    (3) 若 $a\in \Z$, 易知 $a>0$. 序列 $\lbrace k^3-ak\rbrace$ 在 $k<-\sigma$ 和 $k>\sigma$ 时单调递增, 在 $-\sigma\le k\le \sigma$ 时单调递减. 令
    \begin{equation*}
        \sigma^3-a\sigma=t^3-at,
    \end{equation*}
    解得较大的根 $t=2\left(\frac{a}{3}\right)^{1 /2}$,
    所以当 $|k|> 2\left(\frac{a}{3}\right)^{1 /2}$ 时, $k=m^3-am$ 的解 $m$ 只有一个. 这说明此时有 $d_k=c_k e^{i(k^3t_0+kx_0)}$. 记
    \begin{equation}\label{pia}
        \Pi_a:=\lbrace k\in \Z : \exists m\neq k,m\in \Z \text{ s.t. } k^2+km+m^2=a \rbrace.
    \end{equation}
    则式 \eqref{3-2-4}可写作
    \begin{equation*}
        u(t_0+t,x_0-at)=\sum_{l\in \lbrace j^3-aj:j\in \Z  \rbrace}e_l e^{ilt}=\sum_{k\notin \Pi_a} d_k e^{i(k^3-ak)t}+\sum_{l\in \lbrace j^3-aj: j\in \Pi_a, j\in \Z\rbrace} e_l e^{ilt},
    \end{equation*}  
    其中
    \begin{equation*}
        e_l=d_k, \quad \forall k \text{ s.t. } k^3-ak=l.
    \end{equation*}
    因为 $\lbrace l: l=k^3-ak \rbrace \subset \Z $ 为一致分离集, 则由类似 (2) 的方法可得对任意的 $T$, 有关系式
    \begin{equation}\label{3-2-5}
        \int_0^T|u(t_0+t,x_0-t)|^2\d t\asymp \sum_{k\in \Pi_a}|d_k|^2+\sum_{l\in \lbrace j^3-aj: j \notin \Pi_a,j\in\Z\rbrace }|e_l|^2.
    \end{equation}
    由 $\lbrace k^3-ak:k\in \Z\rbrace$ 关于 $k$ 的单调性可知, 对任意一个 $l\in \lbrace k^3-ak; k\in \Z\rbrace$, 至多有 3 个 不同的 $k$, 且至少有一个 $k$  满足 $l=k^3-ak$. 这说明 对某个 $l$, $\sum_{k\in\Z} |d_k|^2$ 中 $d_k=e_l$ 的项至多有 3 个, 至少有一个, 则有关系式
    \begin{equation}\label{3-2-6}
        \sum_{k\in \Z}|d_k|^2\asymp  \sum_{k\in \Pi_a}|d_k|^2+\sum_{l\in \lbrace j^3-aj: j \notin \Pi_a,j\in\Z\rbrace }|e_l|^2.
    \end{equation}
    由式 \eqref{3-2-5} 和式 \eqref{3-2-6} 可得关系式 \eqref{3-2-3}.
    
    固定 $k\in \Pi_a$, 
    \begin{enumerate}
    \item [(i)] 解函数为实值函数的情形. 若 $a\neq k^2$, 设 $m\neq k$ 且 $k^3-ak=m^3-am$, 则易知 $m\neq -k$, 令
    \begin{align*}
        u_0(x)&= e^{-i(k^3t_0+kx_0)} e^{ikx} + e^{i(k^3t_0+kx_0)} e^{-ikx}\\
        &-e^{-i(m^3t_0+mx_0)}e^{imx}-e^{i(m^3t_0+mx_0)}e^{-imx},
    \end{align*}
    则 $u\neq 0$ 且
    \begin{equation*}
        u(t_0+t,x_0-at)=0,\quad\forall t\in \R.
    \end{equation*}
    若 $a=k^2$, 令
    \begin{equation*}
        u_0(x)=e^{-i(k^3t_0+kx_0)}e^{ikx} + e^{i(k^3t_0+kx_0)}e^{-ikx}-2,
    \end{equation*}
    则 $u\neq 0$ 且
    \begin{equation*}
        u(t_0+t,x_0-at)=0,\quad\forall t\in\R.
    \end{equation*}
    \item [(ii)] 解函数为复值函数. 则函数的选取更加简单, 可在 $a\neq k^2$ 时取 $$u_0(x)= e^{-i(k^3t_0+kx_0)} e^{ikx}-e^{-i(m^3t_0+mx_0)}e^{imx}$$以及在 $a=k^2$ 时取 $$u_0(x)=e^{-i(k^3t_0+kx_0)}e^{ikx} -1.$$
    \end{enumerate}
    \end{proof}
    

    式 \eqref{3-2-2} 可等价地写作
    \begin{equation*}
        \int_{\T}|u_0(x)|^2\d x\le C\int_0^T\int_{\T}\delta(x-at)|u(t,x)|^2\d x\d t.
    \end{equation*}
    该式可看作KdV方程 \eqref{kdv-l} 的解关于观测点随时间移动的能观测不等式. 例如, 令 $a=0\notin \Gamma$, 则有
    \begin{equation*}
        \int_{\T}|u_0(x)|^2\d x\le C\int_{t_0}^{t_0+T} |u(t,x_0-at)|^2\d t.
    \end{equation*}
 

 {\bf 集合 $\Gamma$ 的刻画}. 根据 $\Gamma$ 的定义, $a\in\Gamma$ 当且仅当 $x^2+xy+y^2=a$ 有不在对角线上的整数解. 实际上, 若正整数 $a$ 可以写作 $a=m^2+m\cdot m+m^2=3m^2$, 则其一定可以写作 $a=m^2+m\cdot (-2m)+(-2m)^2$, 所以 $a\in\Gamma$ 当且仅当 $x^2+xy+y^2=a$ 有整数解. 而对于 $x^2+xy+y^2=a$ 这样的不定方程的解的问题.
 
 实际上, 设 $\omega=e^{\frac{2\pi i}{3}}$, 对任意满足 $p\equiv 1\pmod{3}$ 的素数 $p$, 可将其表示为 $p=(m-n\omega)(m-n\overline{\omega})$. 由 $\Z$ 和 $\omega$ 扩张成的环 $\Z[\omega]$ 称作 Eisenstein 整环, 它是一个主理想整环, 其中的素因子称为 Eisenstein 素数. Fermat 等人证明了, $p=m^2+mn+n^2,m,n\in\Z$ 当且仅当素数 $p$ 是 Eisenstein 素数 $m-n\omega$的模方, 即$p=(m-n\omega)(m-n\overline{\omega})$. 我们可以将其概括为下述定理.
 
  \begin{theorem}\label{thm-prime}
 素数 $p$ 可写作
 \begin{equation*}
     p=m^2+mn+n^2,\quad m,n\in\Z,
 \end{equation*}
 当且仅当 $p\equiv 1\pmod{3}$ 或 $p=3$, 且这样的分解在 $m\ge n\ge 0$ 时是唯一的.
 \end{theorem}
 
 由定理 \ref{thm-prime}易得下述推论.
 \begin{corollary}\label{crc-1}
 对任意的正整数 $a$, 它可以写作
 \begin{equation*}
     a=m^2+mn+n^2,\quad m,n\in\Z,
 \end{equation*}
  当且仅当其满足下述的素数分解
 \begin{equation}\label{prime-decomp}
     a=3^r\prod_{i=1}^{r_1}p_i^{s_i}\prod_{j=1}^{r_2}q_i^{t_j},
 \end{equation}
 其中 $p_i\equiv 1\pmod{3}, q_j\equiv 2\pmod{3}, 2|t_j,i=1,\cdots,r_1,j=1,\dots,r_2$.
 \end{corollary}

 \begin{corollary}
 $a\in\Gamma$ 当且仅当 $a\equiv 1\pmod{3}$ 或 $a=3^rm,r\in \N_{+},m\equiv 1 \pmod{3}$.
 \end{corollary}
 \begin{corollary}
 若 $a\in \Gamma$, 则
 \begin{equation}
     |\Pi_a|=\cdot 2^r\prod_{i=1}^r3^{s_i},
 \end{equation}
 其中 $r,s_i$ 同推论 \ref{crc-1}.
 \end{corollary}
 \begin{proof}
 首先我们考虑 $a$ 为素数 $p$ 的情形, 并设 $m^2+mn+n^2=p$ 的整数解 $(m,n)$ 的第一个分量 $m\ge 0$.
 \begin{enumerate}
     \item [(i)] $p=3$. 则满足 $m^2+mn+n^2=3$ 的全部整数解有
     \begin{equation*}
         (1,1),(1,-2),(-1,-1),(-1,2).
     \end{equation*}
     所以 $|\Pi_3|=4$.
     \item [(ii)] $p\equiv 1\pmod{3}$. 设 $(m,n)$ 为满足 $m^2+mn+n^2=p,m> n\ge 0$的一个整数解, 则全部整数解为
     \begin{equation*}
         \begin{array}{cc}
           (m,n),(m,-(m+n)),(-m,-n),(-m,m+n),(-(m+n),m),(m+n,-m)   &  \\
            (n,m),(n,-(m+n)),(-n,-m),(-n,m+n),(-(m+n),n),(m+n,-n).  & 
         \end{array}
     \end{equation*}
     所以 $m,n,-m,-n,m+n,-(m+n)\in \Pi_p$ $|\Pi_p|=6$.
 \end{enumerate}
 若 $a$ 可写作推论 \ref{crc-1} 中的形式, 则由整环 $\Z [\omega]$ 的唯一分解性质可得
 $$|\Pi_a|=2(r+1)\times 6\prod_{i=1}^{r_1}\left(s_i+1\right).$$
 \end{proof}

    
定理 \ref{thm3-2-1} 是KdV方程 \eqref{kdv-l} 关于一个线段的能观测性, 接下来我们讨论两个线段的可观测性. 当 $a\le 0$ 以及 $a$ 不为整数 ($\Gamma$ 中的元素一定是大于 $0$ 的整数) 时, 线段的能观测性已经被定理 \ref{thm3-2-1} 证明, 在后续对两个线段的讨论中, 我们只需要考虑 $ a>0 $ 且为正整数, 或者更准确地说, $a\in \Gamma$ 的情形. 
    
\begin{definition}
设 $a\in \Z$, $p$ 为素数, 定义 $e_p(a)$ 为 $a$ 的素数分解中 $p$ 的指数, 它满足
\begin{equation*}
    p^{e_p(a)}|a,p^{e_p(a)+1}\nmid a.
\end{equation*}
\end{definition}
 \begin{theorem}\label{thm-3-2-2}
 给定任意的 $T>0$, 考虑方程 \eqref{kdv-r} 的解. 设 $(t_1,x_1),(t_2,x_2)\in \R\times \T$, $a_1,a_2$ 为两个不同的正整数. 
 \begin{enumerate}
     \item[\rm{(1)}] 若存在 $p\equiv 2\pmod{3}$ 以及正整数 $e\in \N_{+}$ 使得 $e_p(a_1)\neq e_p(a_2)$, 则由 
    \begin{equation*}
        u(t_1+t,x_1-a_1t)=u(t_2+t,x_2-a_2t)=0,\quad \forall t\in (0,T))
    \end{equation*}
    可推出$u$ 是平凡解, 即 $u_0=0$. 更确切地说, 存在一个常数 $C>0$ 使得不等式   
    \begin{equation}
        \sum_{k\in\Z} |c_k|^2\le C\int_0^T|u(t_1+t,x_1-a_1t)|^2+|u(t_2+t,x_2-a_2t)|^2 \d t \label{3-2-10}
    \end{equation}
    \item [\rm{(3)}] 存在 $a_1,a_2\in \Gamma$ 使得 $u\neq 0$, 且
     \begin{equation*}
        u(t_1+t,x_1-a_1t)=u(t_2+t,x_2-a_2t)=0,\quad \forall t\in (0,T).
    \end{equation*}
 \end{enumerate}
 \end{theorem}
 \begin{proof}
    定义
    \begin{equation*}
        \begin{array}{ll}
                        d_{1,k}=\sum\limits_{\substack{
                m\in \Z,  \\
                k^2+mk+m^2 = a_1 }} 
             c_me^{i(m^3t_1+mx_1)}, & k\in \Z, \\
                        d_{2,k}=\sum\limits_{\substack{
                m\in \Z,  \\
                k^2+mk+m^2 = a_2 }} 
             c_me^{i(m^3t_2+mx_2)}, & k\in \Z. 
        \end{array}
    \end{equation*}
    由定理 \ref{thm3-2-1} 可得
\begin{equation}\label{3-2-7}
    \int_0^T |u(t_1+t,x_1-a_1t)|^2\d t\asymp \sum_{k\in\Z}^\infty|d_{1,k}|^2,
\end{equation}
以及
\begin{equation}\label{3-2-8}
    \int_0^T |u(t_2+t,x_2-a_2t)|^2\d t\asymp \sum_{k\in\Z}^\infty|d_{2,k}|^2.
\end{equation}
式 \eqref{3-2-7} 和 式 \eqref{3-2-8}相加可得
\begin{equation}\label{3-2-9}
  \sum_{k\in\Z}^\infty|d_{1,k}|^2+\sum_{k\in\Z}^\infty|d_{2,k}|^2\asymp \int_0^T |u(t_1+t,x_1-a_1t)|^2 + |u(t_2+t,x_2-a_2t)|^2\d t
\end{equation}
 
(1) 不妨令 $e_p(a_1)>e_p(a_2)$, 我们只需证明 $k\in \Pi_{a_1}$ 时 $k\notin \Pi_{a_2}$ 即可推出 $u=0$. 令 $a_1=k^2+km+m^2,k\neq m$, 因为 $a_1\in \Gamma$, 所以可令 $e_p(a_1)=2e$, 则$p^{2e} |a_1$. 因为 $p\equiv 1 \pmod{3}$,  其为Eisenstein 素数, 所以在整环 $Z[\omega]$ 中 $p^{e}|k+m\omega$ 且 $p^e | k-m\omega$. 下面我们用反证法, 假设 $k\in \Pi_{a_2}$, 则存在 $l\neq k$ 使得 $a_2=k^2+kl+l^2$ 成立. 由 $p^{2e} \nmid a_2$ 可知 $p^e \nmid k$, 这与 $p^e| k$ 矛盾, 所以 $k\notin \Pi_{a_2}$. 

上述证明说明对任意整数 $k$ 必有 $|c_k|=|d_{1,k}|$ 或 $|c_k|=|d_{2,k}|$, 所以  
  \begin{equation*}
      \sum_{k\in \Z}|c_k|^2\le \sum_{k\in \Z}|d_{1,k}|^2+\sum_{k\in\Z}|d_{2,k}|^2.
  \end{equation*}
  上式和式 \eqref{3-2-9} 可得不等式 \eqref{3-2-10} 成立.

(2) 首先我们考虑可能的最简单情形, 即存在 $m,n,l,s\in \Z$ 满足不定方程组 
\begin{equation}\label{dio}
    \left\lbrace\begin{array}{ll}
        m^2+ms+s^2=a_1, & m\neq s, \\
        n^2+nl+l^2=a_1, & n\neq l,\\
        m^2+mn+n^2=a_2, & m\neq n,\\
        s^2+sl+l^2=a_2, & s\neq l,
    \end{array}\right.
\end{equation}
 设$x=x_1+a_1 t$ 与 $x=x_2+a_2t$ 相交于 $(t_0,x_0)$.此时只要令
\begin{align*}
    u_0(x) &= e^{-imx_0+m^3t_0}e^{imx}-e^{-i(sx_0+s^3t_0)}e^{isx}-e^{-i(nx_0+n^3t_0}e^{inx}+e^{-i(lx_0+l^3t_0)}e^{ilx}.
\end{align*}
则
\begin{align*}
    u(t,x) &=e^{-i(mx_0+m^3t_0)}e^{imx}e^{im^3t}-e^{-i(sx_0+s^3t_0)}e^{isx}e^{is^3t}\\
    &-e^{-i(nx_0+n^3t_0}e^{inx}e^{in^3t}+e^{i(lx_0+l^3t_0)}e^{ilx}e^{il^3t}.
\end{align*}
带入 $t_0+t,x_0-at$ 可得 
\begin{align*}
    u(t_0+t,x_0-at) &= e^{-i(mx_0+m^3t_0)}e^{im(x_0-at)}e^{im^3(t_0+t)}-e^{-i(sx_0+s^3t_0)}e^{is(x_0-at)}e^{is^3(t_0+t)}\\
    &-e^{-i(nx_0+n^3t_0}e^{in(x_0-at)}e^{in^3(t_0+t)}+e^{i(lx_0+l^3t_0)}e^{il(x_0-at)}e^{il^3(t_0+t)}\\
    &= e^{i(m^3-am)t} -e^{i(s^3-as)t}-e^{i(n^3-an)t}+e^{i(l^3-al)t}.
\end{align*}
当 $a=a_1$ 时, 由其满足的不定方程组 \eqref{dio} 前两式可知上式最后一行第一项与第二项相消, 第三项与第四项相消. 同理, 当 $a=a_2$ 时, 由不定方程组的后两式可知上式最后一行第一项与第三项相消, 第二项与第四项相消.

接下来便是找到满足不定方程组 \eqref{dio} 的整数 $a_1,a_2,m,s,n,l$. 简单的计算可知, 当
\begin{equation*}
    a_1=7,a_2=13,m=1,s=-3,n=3,l=-1
\end{equation*}
时, 是不定方程组 \eqref{dio} 的解. 若要求函数为实值函数, 取初始函数为 $u_0+\overline{u_0}$ 即可.
 \end{proof}
 
 对于 $a_1,a_2\in \Gamma$ 的情形, 我们只需要讨论 $\Pi_{a_1}\cap \Pi_{a_2}\neq \emptyset$ 中的 $k$ 以及相应 $c_k$ 的取值情形. 为了后续定理叙述的方便, 我们将 $\Pi_{a_1}\cap \Pi_{a_2}$ 中的数视作节点, 若其中存在两个不同的节点 $k$ 和 $k'$ 使得 $k^3-a_1k=k'^3-a_1k$ (或 $k^2+kk'+k'^2=a_1$), 我们在两个节点之间用红色的线段连接. 类似地, 若 $\Pi_{a_1}\cap \Pi_{a_2}$ 存在两个不同的节点 $l$ 和 $l'$ 使得 $l^3-a_2l=l'^3-a_2l$, 我们在两个节点之间用蓝色线段连接. 这些节点和线段就构成了节点个数为 $|\Pi_{a_1}\cap \Pi_{a_2}|$ 且具有二染色边的图, 记作 $G(a_1,a_2)$. 例如定理 \ref{thm-3-2-2} 中, 节点 $1,-3,3,-1$ 之间的连线可表示为图 \ref{red-blue-example}.
 
 \begin{figure}[htbp]
     \centering
     \begin{tikzpicture}
     \draw (0,0)node[circle, fill, inner sep=1.5pt]{};
     \draw (0,0)node[below left] {$-3$};
     \draw (0,2)node[circle, fill, inner sep=1.5pt]{};
     \draw (0,2)node[above left] {$1$};
     \draw (2,0)node[circle, fill, inner sep=1.5pt]{};
     \draw (2,0)node[below right] {$-1$};
     \draw (2,2)node[circle, fill, inner sep=1.5pt]{};
     \draw (2,2)node[above right] {$3$};
     \draw[red] (0,0)--(0,2);
     \draw[red] (2,0)--(2,2);
     \draw [blue] (0,0)--(2,0);
     \draw[blue] (0,2)--(2,2);
     \end{tikzpicture}
     \caption{图 $G(7,13)$ 中节点 $1,-3,3,-1$ 构成的子图}
     \label{red-blue-example}
 \end{figure}
 对于一个图 $G(a_1,a_2)$, 若存在红蓝交错的圈, 且圈中每个节点只被圈经过一次, 则称该圈为交错简单圈. 图 \ref{red-blue-example} 就是一个交错简单圈. 由此受到启发, 我们有下述关于两个线段能观测性的充要判定条件.
 
\begin{theorem}
给定任意的 $T>0$, 考虑方程 \eqref{kdv-r} 的解. 设 $(t_1,x_1),(t_2,x_2)\in \R\times \T$, $a_1,a_2$ 为两个不同的正整数且 $a_1,a_2\in \Gamma$, $\Pi_{a_1}\cap \Pi_{a_2}\neq \emptyset$. 则下述两个命题等价:
\begin{enumerate}
    \item[\rm{(1)}] 由 
    \begin{equation*}
        u(t_1+t,x_1-a_1 t)=u(t_2+t,x_2-a_2t)=0,\quad \forall t\in \R
    \end{equation*}
    可推出 $u$ 是平凡解,  且不等式 \eqref{3-2-10} 成立.
    \item [\rm{(2)}] 图 $G(a_1,a_2)$ 中有交错简单圈.
\end{enumerate}
\end{theorem}



 
 上述讨论是基于 $a_1$ 和 $a_2$ 在 $\Z [\omega]$ 中的唯一分解. 实际上, 对 $\Z [\omega]$ 中的大数进行唯一分解是一件困难的事情. 接下来我们将给出一个更加容易的判定准则. 
    
在定理 \ref{thm3-2-1} 中, 我们对序列 $\lbrace k^3-ak\rbrace$ 进行过讨论, 为方便后续的讨论, 我们将函数 $y=x^3-ax$ 的性质整理如下:
    \begin{enumerate}
        \item [(1)] 该函数有两个极值点, 分别是 $x=-\left(\frac{a}{3}\right)^{1 /2}$ 和 $x=\left(\frac{a}{3}\right)^{1 /2}$, 极值分别是 $2\left(\frac{a}{3}\right)^{3 /2}$ 和 $-2\left(\frac{a}{3}\right)^{3 /2}$. 
        \item[(2)] 该函数与 $y=2\left(\frac{a}{3}\right)^{3 /2}$ 轴有两个交点, 分别是 $\left(-\left(\frac{a}{3}\right)^{1 /2},2\left(\frac{a}{3}\right)^{3 /2}\right)$ 和 $\left(2\left(\frac{a}{3}\right)^{1 /2},2\left(\frac{a}{3}\right)^{3 /2}\right)$.
        \item[(3)] 该函数与 $y=-2\left(\frac{a}{3}\right)^{3 /2}$ 轴有两个交点, 分别是 $\left(-2\left(\frac{a}{3}\right)^{1 /2},-2\left(\frac{a}{3}\right)^{3 /2}\right)$ 和 $\left(\left(\frac{a}{3}\right)^{1 /2},-2\left(\frac{a}{3}\right)^{3 /2}\right)$.
    \end{enumerate}
    \begin{figure}[ht]
        \centering
        \begin{tikzpicture}[scale=1.3]
           % axes 
           \draw[very thick,->] (-4,0) -- (4,0) node[right] {$x$};
           \draw[very thick,->] (0,-3) -- (0,3) node[above] {$y$};
           \draw (-0.2,-0.01) node[below] {\small $O$};
           % function
           \draw plot[domain=-2.1:2.1, smooth] (\x, \x*\x*\x-3*\x);
           \draw[dashed] (-1,2)  -- (2,2) node[right] {$\left(2\left(\frac{a}{3}\right)^{1 /2}, 2\left(\frac{a}{3}\right)^{3 /2}\right)$};
           \draw[dashed] (-2,-2) -- (1,-2);
           \draw[dashed] (-1,0) -- (-1,2);
           \draw[dashed] (1,0)  -- (1,-2);
           \draw[dashed] (-2,-2)  -- (-2, 0);
           \draw[dashed] (2,2) -- (2,0);
           \draw (2.2,0) node[below]{$2\left(\frac{a}{3}\right)^{1 /2}$};
           \draw (1.2,0) node[above] {$\left(\frac{a}{3}\right)^{1 /2}$};
           \draw (0,2)node[above right] {$2\left(\frac{a}{3}\right)^{3 /2}$};
        \end{tikzpicture}
        \caption{函数 $y=x^3-ax$, $a>0$ 的图像.}
        \label{fig5}
    \end{figure}
    上述性质具体可见图 \ref{fig5}. 根据这些性质, 我们可以得到下述判定定理.
        \begin{theorem}\label{thm-3-2-3}
    给定任意的 $T>0$, 考虑方程 \eqref{kdv-l} 的解. 设 $(t_1,x_1), (t_2,x_2)\in \R\times \T$, $a_1,a_2$ 为两个不同的正整数, 且满足 $a_2>4a_1>0$. 则由 
    \begin{equation*}
        u(t_1+t,x_1-a_1t)=u(t_2+t,x_2-a_2t)=0,\quad \forall t\in (0,T)
    \end{equation*}
    可推出$u$ 是平凡解, 即 $u_0=0$. 更确切地说, 存在一个常数 $C>0$ 使得不等式\eqref{3-2-10}
    成立.
    \end{theorem}
    \begin{proof}
    首先我们注意到, 对任意给定的整数 $k$, 若对所有方程 \eqref{kdv-l} 的解都满足定理条件可推出 $c_k=0$, 则一定也可以推出 $c_{-k}=0$. 事实上, 若存在 $c_{-k}\neq 0$ 且满足定理条件的解 $u(t,x)$, 令 $v(t,x)=\overline{u(t,x)}$, 则可得 $v(t,x)$ 的第 $k$ 个傅里叶系数不为零.
    所以, 我们只需要考虑 $k\ge 0$ 的情形.


    
      我们需要证明对任意的 $k\in\Z$, 都有 $c_k=0$. 固定 $k\in \Z$, 若 $k\notin \Pi_{a_1}\cap \Pi_{a_2}$, 则由定理 \ref{thm3-2-1} 可知 $c_k=0$. 因此在接下来的证明中我们设 $k\in  \left(\Pi_{a_1}\cap \Pi_{a_2}\right)$. 此时只需要考虑 $k\in \left[0, 2\left(\frac{a_1}{3}\right)^{1 /2}\right]$ 的情形, 见图 \ref{fig6}.
          \begin{figure}[ht]
        \centering
         \begin{tikzpicture}[scale=1.3]
           % axes 
           \draw[very thick,->] (-3.5,0) -- (3.5,0) node[right] {$x$};
           \draw[very thick,->] (0,-4) -- (0,4) node[above] {$y$};
           \draw (-0.2,-0.01) node[below] {\small $O$};
           % functions
         \draw plot[domain=-1.8:1.8, smooth] (\x, \x*\x*\x-1*\x);
         \draw plot[domain=-2.1:2.1, smooth] (\x, \x*\x*\x-4.2*\x);
         \draw[dashed] (0.7,0) node[above] {$k$} -- (0.7,-2.62);
         \draw[dashed] (-1.8,-2.62) -- (1.6,-2.62);
         \draw[dashed] (1.6,-2.62) -- (1.6,2.53);
         \draw (1.6,0) node[above right] {$k'$};
        \end{tikzpicture}
        \caption{函数$y=x^3-a_1x$ 和 $y=x^3-a_2 x$ 的图像.}
        \label{fig6}
    \end{figure}
     因为 $k\in \Pi_{a_2} $, 所以必然存在另一个不同的 $k'$ 使得 $k^3-a_2k={k'}^3-a_2k'$, 但是这样的 $|k'|\ge\left(\frac{a_2}{3}\right)^{1 /2}>2\left(\frac{a_2}{3}\right)^{1 /2}$, 所以 $k'\notin \Pi_{a_1}$. 这说明使得 $k^3-a_2k=k'^3-a_2k'$ (或等价地, $k^2+k'k+k'^2=a_2$) 的 $k'$ 都有 $c_{k'}=0$, 而根据定理 \ref{thm3-2-1} 有
     \begin{equation*}
         d_{2,k}=\sum\limits_{\substack{k'\neq k\in \Z,\\ k^2+mk+m^2=a_2}} c_{k'}e^{i(k'^3t_2+k'x_0)}+c_k e^{i(k^3t_2+kt_2)}=0
     \end{equation*}
     上式的第一部分由 $c_{k'}=0$ 可得 $\sum_{\substack{k'\neq k\in \Z,\\ k^2+mk+m^2=a_2}} c_{k'}e^{i(k'^3t_2+k'x_0)}=0$, 进而 $c_k=0$. 所以 $u_0=0$.
  
  由上面的论证可知, 对任意的 $k\in \Z$, 有 $d_{1,k}=c_k^{i(k^3t_1+kx_1)}$ 或 $d_{2,k}=c_k^{i(k^3t_2+kx_2)}$ 成立, 所以 $|c_k|=|d_{1,k}|$ 或 $|c_k|=|d_{2,k}|$, 所以
  \begin{equation*}
      \sum_{k\in \Z}|c_k|^2\le \sum_{k\in \Z}|d_{1,k}|^2+\sum_{k\in\Z}|d_{2,k}|^2.
  \end{equation*}
  上式和式 \eqref{3-2-9} 可得不等式 \eqref{3-2-10} 成立.   
      \end{proof}
    
  \iffalse  \begin{theorem}
    给定任意的 $T>0$, 考虑方程 \eqref{kdv-l} 的解. 设 $(t_1,x_1), (t_2,x_2)\in \R\times \T$, $0<a_1<a_2$ 为两个不同的正整数, 且满足下述条件之一  
\begin{itemize}
    \item [\rm{(1)}] $a_2>4a_1$.
    \item [\rm{(2)}] $a_2-a_1<L$, 其中 $L$ 是椭圆 $x^2+xy+y^2=a_1$ 外的整点离椭圆的最小距离.
\end{itemize}
 则由 
    \begin{equation*}
        u(t_1+t,x_1-a_1t)=u(t_2+t,x_2-a_2t)=0,\quad \forall t\in (0,T)
    \end{equation*}
    可推出$u$ 是平凡解, 即 $u_0=0$. 更确切地说, 存在一个常数 $C>0$ 使得不等式
    \begin{equation}\label{3-2-10}
        \sum_{k\in\Z} |c_k|^2\le C\int_0^T|u(t_1+t,x_1-a_1t)|^2+|u(t_2+t,x_2-a_2t)|^2 \d t
    \end{equation}
    成立.
    \end{theorem}
    \begin{proof}
    首先我们注意到, 对任意给定的整数 $k$, 若对所有方程 \eqref{kdv-l} 的解都满足定理条件可推出 $c_k=0$, 则一定也可以推出 $c_{-k}=0$. 事实上, 若存在 $c_{-k}\neq 0$ 且满足定理条件的解 $u(t,x)$, 令 $v(t,x)=\overline{u(t,x)}$, 则可得 $v(t,x)$ 的第 $k$ 个傅里叶系数不为零.
    所以, 我们只需要考虑 $k\ge 0$ 的情形.
    
    定义
    \begin{equation*}
        \begin{array}{ll}
                        d_{1,k}=\sum\limits_{\substack{
                m\in \Z,  \\
                k^2+mk+m^2 = a_1 }} 
             c_me^{i(m^3t_1+mx_1)}, & k\in \Z, \\
                        d_{2,k}=\sum\limits_{\substack{
                m\in \Z,  \\
                k^2+mk+m^2 = a_2 }} 
             c_me^{i(m^3t_2+mx_2)}, & k\in \Z. 
        \end{array}
    \end{equation*}
    由定理 \ref{thm3-2-1} 可得
\begin{equation}\label{3-2-7}
    \int_0^T |u(t_1+t,x_1-a_1t)|^2\d t\asymp \sum_{k\in\Z}^\infty|d_{1,k}|^2,
\end{equation}
以及
\begin{equation}\label{3-2-8}
    \int_0^T |u(t_2+t,x_2-a_2t)|^2\d t\asymp \sum_{k\in\Z}^\infty|d_{2,k}|^2.
\end{equation}

式 \eqref{3-2-7} 和 式 \eqref{3-2-8}相加可得
\begin{equation}\label{3-2-9}
  \sum_{k\in\Z}^\infty|d_{1,k}|^2+\sum_{k\in\Z}^\infty|d_{2,k}|^2\asymp \int_0^T |u(t_1+t,x_1-a_1t)|^2 + |u(t_2+t,x_2-a_2t)|^2\d t
\end{equation}
    
     (1) $a_2>4a_1$. 我们需要证明对任意的 $k\in\Z$, 都有 $c_k=0$. 固定 $k\in \Z$, 若 $k\notin \Pi_{a_1}\cap \Pi_{a_2}$, 则由定理 \ref{thm3-2-1} 可知 $c_k=0$. 因此在接下来的证明中我们设 $k\in  \left(\Pi_{a_1}\cap \Pi_{a_2}\right)$. 此时只需要考虑 $k\in \left[0, 2\left(\frac{a_1}{3}\right)^{1 /2}\right]$ 的情形, 见图 \ref{fig6}.
          \begin{figure}[ht]
        \centering
         \begin{tikzpicture}[scale=1.3]
           % axes 
           \draw[very thick,->] (-3.5,0) -- (3.5,0) node[right] {$x$};
           \draw[very thick,->] (0,-4) -- (0,4) node[above] {$y$};
           \draw (-0.2,-0.01) node[below] {\small $O$};
           % functions
         \draw plot[domain=-1.8:1.8, smooth] (\x, \x*\x*\x-1*\x);
         \draw plot[domain=-2.1:2.1, smooth] (\x, \x*\x*\x-4.2*\x);
         \draw[dashed] (0.7,0) node[above] {$k$} -- (0.7,-2.62);
         \draw[dashed] (-1.8,-2.62) -- (1.6,-2.62);
         \draw[dashed] (1.6,-2.62) -- (1.6,2.53);
         \draw (1.6,0) node[above right] {$k'$};
        \end{tikzpicture}
        \caption{函数$y=x^3-a_1x$ 和 $y=x^3-a_2 x$ 的图像.}
        \label{fig6}
    \end{figure}
     因为 $k\in \Pi_{a_2} $, 所以必然存在另一个不同的 $k'$ 使得 $k^3-a_2k={k'}^3-a_2k'$, 但是这样的 $|k'|\ge\left(\frac{a_2}{3}\right)^{1 /2}>2\left(\frac{a_2}{3}\right)^{1 /2}$, 所以 $k'\notin \Pi_{a_1}$. 这说明使得 $k^3-a_2k=k'^3-a_2k'$ (或等价地, $k^2+k'k+k'^2=a_2$) 的 $k'$ 都有 $c_{k'}=0$, 而根据定理 \ref{thm3-2-1} 有
     \begin{equation*}
         d_{2,k}=\sum\limits_{\substack{k'\neq k\in \Z,\\ k^2+mk+m^2=a_2}} c_{k'}e^{i(k'^3t_2+k'x_0)}+c_k e^{i(k^3t_2+kt_2)}=0
     \end{equation*}
     上式的第一部分由 $c_{k'}=0$ 可得 $\sum_{\substack{k'\neq k\in \Z,\\ k^2+mk+m^2=a_2}} c_{k'}e^{i(k'^3t_2+k'x_0)}=0$, 进而 $c_k=0$. 所以 $u_0=0$.
  
  由上面的论证可知, 对任意的 $k\in \Z$, 有 $d_{1,k}=c_k^{i(k^3t_1+kx_1)}$ 或 $d_{2,k}=c_k^{i(k^3t_2+kx_2)}$ 成立, 所以 $|c_k|=|d_{1,k}|$ 或 $|c_k|=|d_{2,k}|$, 所以
  \begin{equation*}
      \sum_{k\in \Z}|c_k|^2\le \sum_{k\in \Z}|d_{1,k}|^2+\sum_{k\in\Z}|d_{2,k}|^2.
  \end{equation*}
  上式和式 \eqref{3-2-9} 可得不等式 \eqref{3-2-10} 成立.   
  
  (2) $a_2-a_1<L^2$. 给定椭圆 $l_1:x^2+xy+y^2=a_1$ 和椭圆 $l_2:x^2+xy+y^2=a_2$, 若 $k\in \Pi_{a_1}$, 则存在 $k'\neq k,k\in \Z$ 使得 $k^2+kk'+k'^2=a_1$, 所以 $(k,k')$ 是椭圆 $l_1$ 的整数解.
  
       \begin{figure}[ht]
        \centering
        \begin{tikzpicture}[scale=1.3]
           % axes 
           \draw[very thick,->] (-4,0) -- (4,0) node[right] {$x$};
           \draw[very thick,->] (0,-4) -- (0,4) node[above] {$y$};
           \draw (-0.2,-0.01) node[below] {\small $O$};
           \draw[rotate=-45] (0,0) ellipse (4 cm and 2.3094 cm);
           \draw[rotate=-45] (0,0) ellipse (4.5 cm and 2.598076 cm);
           \draw[dashed] (1.5,-4) -- (1.5,4);
           \draw[dashed] (0,1.73)--(1.5,1.73);
           \draw[dashed] (0,2.17)--(1.5,2.17);
           \draw (1.5,0)node[below right] {$k$};
           \draw (0,1.73)node[below right] {$k'$};
           \draw (0,2.17)node[below right] {$k''$};
        \end{tikzpicture}
        \caption{椭圆 $x^2+xy+y^2=a_1$ 和 $x^2+xy+y^2=a_2$.}
        \label{fig7}
    \end{figure}
  
  对任意的$x\in \left[2\left(\frac{a_1}{3}\right)^{1 /2}-,2\left(\frac{a_1}{3}\right)^{1 /2}\right]$, 反解等式得上半支函数 $y=\frac{-x+\sqrt{4a_1-3x^2}}{2}$ 和 $y=\frac{-x+\sqrt{4a_1-3x^2}}{2}$, 作差可得
  \begin{equation}
      d(x):=\frac{\sqrt{4a_2-3x^2}-\sqrt{4a_1-3x^2}}{2}.
  \end{equation}
  该函数是点 $(x,y)\in l_1$ 与 $(x,y')\in l_2, yy'>0$ 的距离, 易知 $d(x)$ 最大值为 $\sqrt{a_2-a_1}$. 当 $\sqrt{a_2-a_1}<L$ 时, 说明对任意的整点 $(k,k')\in l_1$, 满足 $(k,k'')\in l_2$ 的点都不是整点, 即 $k\notin \Pi_{a_2}$ (见图 \ref{fig7}). 不等式 \eqref{3-2-10} 的证明同 (1).
      \end{proof}
      
\begin{theorem}
给定任意的 $T>0$, 考虑方程 \eqref{kdv-l} 的解. 设 $(t_1,x_1), (t_2,x_2),(t_3,x_3)\in \R\times \T$, $0<a_1<a_2<a_3$ 为三个不同的实数, 且满足 $a_3 - a_1 <1$. 则由 
\begin{equation*}
    u(t_1+t,x_1-a_1t)=u(t_2+t,x_2-a_2 t)=u(t_3+t,x_3-a_3t)=0, \quad \forall t\in (0,T)
\end{equation*}
可推出 $u$ 是平凡解, 即 $u_0=0$. 更确切地说, 存在一个常数 $C>0$ 使得不等式
\begin{equation}
            \sum_{k\in\Z} |c_k|^2\le C\int_0^T\sum\limits_{i=1}^3|u(t_i+t,x_i-a_it)|^2 \d t
\end{equation}
成立.
\end{theorem}

\begin{proof}
给定椭圆 $l_i:x^2+xy+y^2=a_i,i=1,2,3$. 任取 $k\in \left[-2\left(\frac{a}{3}\right)^{1 /2},2\left(\frac{a}{3}\right)^{1 /2}\right]$, 思路同前一定理, 只要证明对任意的 $k$ 都存在一个 $i$ 使得 $k\notin \Pi_{a_i}$ 即可. 

我们用反证法,假设 $k\in \left(\Pi_{a_1}\cap\Pi_{a_2}\cap \Pi_{a_3}\right)$, 过点 $(k,0)$ 作竖直线交三个椭圆于 $E,F,G,H,I,J$, 如图 \ref{fig8} 所示.
      \begin{figure}[ht]
        \centering
        \begin{tikzpicture}[scale=1.3]
           % axes 
           \draw[very thick,->] (-4.5,0) -- (4.5,0) node[right] {$x$};
           \draw[very thick,->] (0,-4.5) -- (0,4.5) node[above] {$y$};
           \draw (-0.2,-0.01) node[below] {\small $O$};
           \draw[rotate=-45] (0,0) ellipse (4 cm and 2.3094 cm);
           \draw[rotate=-45] (0,0) ellipse (4.5 cm and 2.598076 cm);
           \draw[rotate=-45] (0,0) ellipse (5 cm and 2.88675 cm);
           \draw (1.5,-4.5) -- (1.5,4.5);
           %\draw (-4,1.75)--(4,1.75);
           %\draw (-4,2.16)--(4,2.16);
           %\draw (-4, 2.53) -- (4, 2.53);
           %\draw (-4,-3.27) -- (4,-3.27);
           %\draw (-4,-3.65)--(4,-3.65);
           %\draw (-4, -4.04) -- (4,-4.04);
           \draw (1.5,1.75)node[below left] {$E$};;
           \draw (1.5,2.16)node[below left] {$F$};
           \draw (1.5,2.53)node[below left] {$G$};
           \draw (1.5,0)node[below right] {$k$};
           \draw (1.5,-3.27)node[below left]{$H$};
           \draw (1.5,-3.65)node[below left]{$I$};
           \draw (1.5,-4.04)node[below left]{$J$};
          \draw (1.5,1.75)node[circle, fill, inner sep=1.5pt]{};
          \draw (1.5,2.16)node[circle, fill, inner sep=1.5pt]{};
          \draw (1.5,2.53)node[circle,fill, inner sep=1.5pt]{};
          \draw (1.5,-3.27)node [circle,fill,inner sep =1.5pt]{};
          \draw (1.5,-3.65)node[circle,fill, inner sep=1.5pt]{};
          \draw (1.5,-4.04)node [circle,fill,inner sep =1.5pt]{};
        \end{tikzpicture}
        \caption{椭圆 $l_1,l_2$ 和 $l_3$.}
        \label{fig8}
    \end{figure}
     由 $k\in \Pi_{a_1}$ 知, $E$ 和 $H$ 至少有一个是整点, 同理由 $k\in \Pi_{a_2}$ 知 $F$ 和 $I$ 至少有一个整点, 由 $k\in \Pi_{a_3}$ 知 $G$ 和 $J$ 至少有一个整点. 但由于 $|EF|,|GH|$ 的长度都小于 $1$ (利用条件 $a_3-a_1<1$ ), 所以 $E$ 和 $F$ 不可能同时为整点, $H$ 和 $I$ 也不可能同时为整点. 所以整点只可能只有 $E$ 和 $I$ 或 $F$ 和 $G$ 两种情况. 不妨设整点是 $E$ 和 $I$. 因为 $I$ 为整点, $|IJ|<1$, 所以 $J$ 不可能为整点, 所以 $G$ 必为整点. 这样我们就得到了 $E$ 和 $G$ 为整点. 这与 $|EG|<1$ 矛盾, 所以假设不成立, 即至少有一个 $P_{a_i}$ 使得 $k\notin \Pi_{a_i}$.
\end{proof}
   \fi 
{   \color{red}
\section{常数估计}
精确计算出Ingham不等式的常数
\section{经典结果的新证明}
利用前面的结果给出开区域的能观测不等式以及给出显式的常数以来
\chapter{可控制性}
\section{HUM方法}
\section{精确可控性}
 }    
 \fi 

    
   % \section{常数估计}
    %\section{经典结果的新证明}
    
    
    %\chapter{控制论中的应用}
    %\section{基本定义}
    %admissible, observable
    %\section{HUM方法}
    %\section{能观测不等式到可控性}
    
    %在本节中, 我们给出线性KdV方程两点可观测性的一般判据. 该判据源于希尔伯特空间方法在傅里叶变换下不确定性原理的应用.
    
    %设 $S(t)=e^{-t\partial_x^3}$ 为线性KdV方程的算子群,
    
    % 更多内容
    %\subsection{这是二级标题}
    % 更多内容

    \backmatter
    \chapter{致谢}
    % 致谢内容
    \cugthesisbib{refs}

    \appendix
    \chapter{这是附录A}
    % 这里是附录内容
\end{document}
